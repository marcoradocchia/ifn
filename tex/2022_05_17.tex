%LTeX: language=it
\chapter{2022-05-17}
\section{Formula semi-empirica di massa (formula di Weizsäcker)}
\subsection{Nuclei stabili}
% TODO: copia dalle dispense

\subsection{Energia di legame dei nuclei}
L'\textit{energia di legame per nucleone} è data da:
\begin{equation}
	\eps = B
\end{equation}
% TODO: grafico (energia di legame per nucleone: energia su numero atomico)
\begin{figure}[ht]
	\centering
	\incfig[1]{./img/2022_05_17}{binding_energy}
	\caption{Energia di legame per nucleone}
	\label{fig:binding_energy}
\end{figure}


\begin{note}[]
	Questa formula si può pensare in analogia con una \textit{goccia di liquido}.
	Infatti un liquido è un fluido con una \textit{densità fissata}.
\end{note}

Ricordiamo che il \textit{raggio} nucleare può essere approssimato da:
\begin{equation}
	R_N \simeq \SI{12}{\femto\m} A ^{\nicefrac{1}{3}}
\end{equation}
Quindi il \textit{volume} occupato da un nucleo è circa:
\begin{equation}
	V_N \simeq \frac{4}{3} \pi R_N^3 = \frac{4}{3} \pi \mrb{\SI{12}{\femto\m} A}
\end{equation}
Quindi abbiamo che:
\begin{equation}
	\mdv{N}{V} = \frac{A}{V_N} = \frac{\cancel{A}}{\frac{4}{3} \pi
		\mrb{\SI{12}{\femto\m} \cancel{A}}}
\end{equation}

Penseremo il potenziale d'interazione fra nucleoni all'interno di un nucleo
come il potenziale d'interazione fra particelle che costituiscono un
liquido\footnote{
	Il prof. disegna il grafico del \textit{potenziale di Lennard-Jones}
}.

% TODO: riascolta che c'è troppa carne al fuoco
\begin{equation}
	V_\text{coulomb} = \frac{Z_1 Z_2 e^2}{4 \pi \eps_0} \frac{1}{r}
\end{equation}

% TODO: riascolta che c'è troppa carne al fuoco
\begin{equation}
	\textbf{Potenziale di Yukawa}\qquad
	\boxed{
		V = \frac{g}{r} e^{- \frac{r}{r_T}}
	}
\end{equation}
$g$ è la \textit{costante di accoppiamento} dell'interazione e $r_T$
rappresenta la \textit{lunghezza di taglio} dell'interazione ($\mu$ il suo
inverso).
\begin{note}[]
  Il pione $\pi$ è il mediatore dell'interazione forte nucleare.
\end{note}
\begin{note}[]
  La costante $\mu$ ha le dimensioni di una lunghezza e si può scrivere come:
  $\mu = \frac{m c^2}{\hbar c}$.
\end{note}
\begin{note}[]
	La carica nelle \textit{interazioni forti} si chiama \textit{\textbf{colore}}.
	L'effetto delle interazioni forti si osserva quando le particelle sono portate
	a distanze paragonabili con le loro dimensioni lineari.
\end{note}

\begin{equation}
	B \sim \mrb{\mdv{N}{V} \frac{4}{3} \pi r_T^3} \eps_\text{in} A
\end{equation}
dove $\mdv{N}{V} \frac{4}{3} \pi r_T^3 = N_\text{???}$
Quindi se dividiamo per $A$:
\begin{equation}
	\eps \equiv \frac{B}{A} \sim \mrb{\mdv{N}{V} \frac{4}{3} \pi r_T^3}
	\eps_\text{in} \frac{A}{A} = \text{const}
\end{equation}

% Simmetria di isospin (scambio elettrone???)

Abbiamo il \textit{termine di volume}:
\begin{equation}
	B = a_V A
\end{equation}
quindi:
\begin{equation}
	\eps = \frac{B}{A} = a_V
\end{equation}

\subsubsection{Termine di superficie}
Introduciamo ora una \textit{correzione}. Fino ad ora abbiamo assunto che tutti
i nucleoni interagiscono fra loro allo stesso modo. È facile capire però che un
nucleone alla superficie del nucleo interagisce con un numero minore di altri
nucleoni\footnote{
	Oltrepassata la superficie del nucleo non troviamo ulteriori nucleoni!
}.
Il raggio e la superficie del nucleo sono rispettivamente proporzionali ad $A$:
\begin{equation}
	R_N \propto A ^{\nicefrac{1}{3}}
	\mthen
	S_N \propto A ^{\nicefrac{2}{3}}
\end{equation}
Il numero di nucleoni contenuti all'interno di un \textit{guscio superficiale
	di raggio $r_T$}, quindi:
\begin{equation}
	N_n ^\text{superf.} \sim \mdv{N}{V} S_N r_T \propto A ^{\nicefrac{2}{3}}
\end{equation}
(ricorda che $\mdv{N}{V}$ rappresenta la densità di nuceloni nel nucleo).
Quindi:
\begin{align*}
	B = a_V A - a_S A ^{\nicefrac{2}{3}}
	\\
	\Rightarrow \eps = \frac{B}{A} = a_V - a_S A ^{-\nicefrac{1}{3}}
\end{align*}

\paragraph{Termine Coulombiano}
Se fosse solo così, però, in natura dovrebbero essere privilegiati i nuclei ad
$A$ molto grande. L'interazione forte rispetterà quanto appena detto, per per
l'interazione Coulombiana, che è un'\textit{interazione a lungo raggio}, non
possiamo dire che i protoni (che sono gli unici che risentono dell'effetto
coulombiano) interagiscano solo con i primi vicini. Il \textit{potenziale
	d'interazione coulombiana totale} fra le coppie di protoni in un nucleo sarà:
\begin{equation}
	\mrb{V _\text{coulomb}}_\text{tot} \simeq \frac{Z \mrb{Z - 1}}{2} \frac{e^2}{4
		\pi \eps_0} \frac{1}{R_N}
\end{equation}
dove $\nicefrac{Z \mrb{Z - 1}}{2}$ è il \textit{numero di coppie di protoni
	all'interno del nucleo}. In termini della \textit{costante di struttura fine}
$\alpha$:
\begin{equation}
	\mrb{V _\text{coulomb}}_\text{tot} \simeq \frac{Z \mrb{Z - 1}}{2}
	\frac{e^2}{4 \pi \eps_0} \frac{\hbar c}{\hbar c} \frac{1}{R_N} = \frac{Z
		\mrb{Z - 1}}{2} \frac{\alpha \mrb{\hbar c}}{R_N}
\end{equation}
Chiaramente questa è una stima approssimata, poiché per avere una stima più
accurata dovremmo almeno avere informazioni sulla distribuzione dei protoni
all'interno del nucleo.

Detto questo, quindi, possiamo scrivere che l'\textit{energia di legame per
	nucleone} diventa:
\begin{equation}
	B
	= a_V A - a_S A ^{\nicefrac{2}{3}} - a_C \frac{Z^2}{A^{\nicefrac{1}{3}}}
\end{equation}
Quindi otteniamo:
\begin{equation}
	\eps
	= \frac{B}{A}
	= a_V - a_S A^{-\nicefrac{1}{3}} - a_C \frac{Z^2}{A^{\nicefrac{4}{3}}}
\end{equation}

Quindi possiamo dire che ad alti valori di $A$, la decrescita dell'energia di
legame per nucleone è dovuta dalla \textit{\textbf{repulsione coulombiana}} che
diventa prevalente. Per questo motivo \textit{non possiamo avere nuclei con
	numero arbitrario di nucleoni} (oltre certe soglie, infatti, i nuclei
cominciano a fare fissione).

\begin{note}[]
	Attenzione! Mancano delle considerazioni, perché così facendo avremmo che
	\textit{i nuclei più fortemente legati sarebbero quelli privi di protoni}!
\end{note}

\subsubsection{Termine di Asimmetria}
La natura sceglie $Z = N$, perché \textit{neutroni e protoni
	\textbf{sono fermioni}}, quindi non possiamo porli tutti nello stesso livello
di energia! Nella disposizione nei livelli energetici dovremo soddisfare quindi
il \textit{principio di esclusione di Pauli}.

Descriviamo il \textit{nucleo come una buca di potenziale} di volume $V$; la
\textit{\textbf{densità di stati nello spazio delle fasi}}:
\begin{equation}
	\frac{\mathrm{d}^{3} n_\text{stati}}{\mathrm{d}^{3} p}
	= \frac{V}{\mrb{2 \pi}^{3} \hbar^3}
\end{equation}
\begin{note}[]
  Il volume minimo occupato da una particella quantistica nello spazio delle
  fasi si ottiene dal \textit{principio d'indeterminazione}.
\end{note}
Lo stato di \textit{minima energia} si ottiene occupando tutti gli stati di
energia più bassa, \textit{compatibilmente con il principio di esclusione di
	Pauli}.
\begin{equation}
	Z
	= N_p
	= 2 \int_{\abs{\vec{p}}\leq p_{F,p}} \mathrm{d}^{3}
	p\,\frac{\mathrm{d}^{3} n_\text{stati}}{\mathrm{d}^{3} p}
	= 2 \mint{0}{p_{F,p}}{p}{4 \pi p^2 \frac{V}{\mrb{2 \pi}^{3} \hbar^3}}
	= \frac{8 \pi}{3} p _{F,p}^{3} \frac{V}{\mrb{2 \pi}^3 \hbar^3}
\end{equation}
\begin{equation}
	A - Z
	\equiv N_n
	= 2 \mint{0}{p _{F,n}}{p}{4 \pi p^2 \frac{V}{\mrb{2 \pi}^{3} \hbar^3}}
  = \frac{8 \pi}{3} p _{F,n}^{3} \frac{V}{\mrb{2 \pi}^{3} \hbar^3}
  = \frac{V p _{F, n}^3}{3 \pi^2 \hbar^3}
\end{equation}
dove $p_{F,p}$ e $p_{F,p}$ sono gli \textbf{impulsi di fermi} relativi
a protoni e neutroni.
Le energie saranno\footnote{
  Nel calcolo abbiamo assunto la distribuzione degli impulsi isotropa, per cui
  si può sostituire $\md[3]{p} = 4 \pi p^2 \md[]{p}$
}:
\begin{equation}
	E_p = 2 \int_{\abs{\vec{p}}\leq p_{F,p}} \mathrm{d}^{3}
	p\,\frac{\mathrm{d}^{3} n_\text{stati}}{\mathrm{d}^{3} p} \frac{p^{2}}{2 m_p}
	= 2 \mint{0}{p_{F,p}}{p}{\frac{4 \pi p^4}{2 m_p} \frac{V}{\mrb{2 \pi}^{3}
			\hbar^3}}
	= \frac{8 \pi}{5} \frac{p _{F,p}^{5}}{2 m_p} \frac{V}{\mrb{2 \pi}^3 \hbar^3}
\end{equation}
\begin{equation}
	E_n = 2 \mint{0}{p _{F,n}}{p}{\frac{4 \pi p^4}{2 m_n} \frac{V}{\mrb{2 \pi}^3
			\hbar^3}} = \frac{8 \pi}{5} \frac{p _{F,n}^{5}}{2 m_n} \frac{V}{\mrb{2
			\pi}^{3} h^3}
\end{equation}
L'energia cinetica totale è data da:
\begin{equation}
	E _\text{tot}
  = E_p + E_n
  = \frac{8 \pi}{5} \frac{V}{\mrb{2 \pi}^{3} \hbar^3}
	\frac{1}{2 m_n} \msb{p _{F,p}^{5} + p _{F, n}^{5}}
\end{equation}
ricordando che $m_p \simeq m_n \simeq m_N \simeq \SI{1}{\GeV \per c^2}$.

