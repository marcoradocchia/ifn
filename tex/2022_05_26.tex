%LTeX: language=it
\chapter{2022-05-26}
\section{Instabilità dei nuclei (reprise)}
\subsection{Decadimenti $\alpha$ (reprise)}
\subsubsection{Descrizione fenomenologica del decadimento $\alpha$ (reprise)}
\paragraph{Effetto tunnel (reprise)}
La \textit{\textbf{probabilità di tunneling}} sarà data da:
\[
  P = \abs{t}^{2} = \frac{16 \delta^2}{\mrb{1 + \delta^2}^{2}} e^{-2 K L}
  = \frac{16 \mrb{\frac{k}{K}}^{2}}{\msb{1 + \mrb{\frac{k}{K}}^{2}}^{2}} e^{-2
  K L} \simeq e^{-2 K L}
\]
se $\nicefrac{k}{K} \simeq 1$.

\paragraph{Barriera di potenziale arbitraria}
% TODO: riascolta
% TODO: figura
\[
  P_{T} = P_1 P_2 \dots P_N = e^{-2 K_1 \Delta x} e^{-2 K_2 \Delta x} \dots
  e^{-2 K_N \Delta x} = e^{-\msum{i}{} K_i \Delta x}
\]
nel limite in cui mandiamo $\Delta x \to 0$, la somma diventa un integrale ed
otteniamo così la probabilità di attraversare interamente la barriera di
potenziale arbitraria:
\[
  P_T = e^{- 2 \mint{A}{B}{x}{K \mrb{x}}}
\]
\begin{note}[]
  Questo è un ragionamento approssimato. Stiamo componendo le probabilità
  (mentre dovremmo comporre le ampiezze di probabilità, causa interferenza),
  quindi stiamo implicitamente assumento che il potenziale non vari molto
  rapidamente rispetto alla funzione d'onda.
\end{note}
Consideriamo che, come avevamo detto:
\[
  \hbar K = \sqrt{2m V \mrb{x} - E}
\]
e otteniamo che la \textit{probabilità di tunneling attraverso una barriera di
potenziale arbitraria}:
\[
  P_T = \exp{-\frac{2}{\hbar} \mint{A}{B}{x}{\sqrt{2m V \mrb{x} - E}}} = e^{-G}
\]
dove $G$ è il \textbf{fattore di penetrazione di Gammon}.

\paragraph{Tunneling attraverso barriera coulombiana}
Indichiamo con $R$ una \textit{\textbf{lunghezza di taglio}} sotto la quale le
\textit{interazioni forti} sono dominanti ed al di sopra della quale sono
invece \textit{trascurabili}.
% TODO: figura
La \textit{probabilità di tunneling della particella $\alpha$} può essere
calcolata, dato il \textit{fattore di penetrazione di Gammon}:
\[
  G = \frac{2}{\hbar} \mint{R}{b}{r}{\sqrt{2 m \mrb{\frac{zZ e^2}{4 \pi \eps_0}
  \frac{1}{r} - E}}}
\]
dove $m$ è la \textit{massa della particella $\alpha$}, $z = 2$ il
\textit{numero atomico della particella $\alpha$}, $Z$ il \textit{numero
atomico della particella figlio} ed $E = T _{\alpha} \simeq Q _{\alpha}$.
Otteniamo:
\[
  G = \frac{2}{\hbar} \sqrt{2 m} \sqrt{\frac{z Z e^2}{4 \pi \eps_0}}
  \mint{R}{b}{r}{\mrb{\frac{1}{r} - \frac{1}{b}}^{\nicefrac{1}{2}}}
\]
dove il seguente passaggio intermedio:
\[
  G = \frac{2}{\hbar} \sqrt{2 m} \sqrt{\frac{z Z e^2}{4 \pi \eps_0}}
  b^2 \frac{b}{???} \mint{R}{b}{r}{\frac{1}{b} \mrb{\frac{1}{r} -
  \frac{1}{b}}^{\nicefrac{1}{2}}} % TODO: ???
\]
quindi:
\[
  G = \frac{2}{\hbar} \sqrt{\frac{m zZ e^2 b}{2 \pi \eps_0}} \mcb{\arccos
  \mrb{\sqrt{\frac{R}{b}}} - \sqrt{\frac{R}{b} \mrb{1 - \frac{R}{b}}}}
\]
In \textit{approssimazione di \textbf{barriera spessa}}, i.e. $\frac{R}{b} \ll
1$, si ottiene:
\[
  G = \frac{2}{\hbar} \sqrt{\frac{m zZ e^2}{2 \pi \eps_0}} \mcb{\frac{\pi}{2} -
  2 x ^{\nicefrac{1}{2}}}
\]
dove $x = \nicefrac{R}{b}$, tenendo conto anche dell'espressione di
$\arccos{\sqrt{x}}$.

Sappiamo che $R ~ \si{\femto\m}$, mentre $b$ dipende dall'energia cinetica
della particella, ovvero:
\[
  E = \frac{1}{2} m v^2 = \frac{zZ e^2}{4 \pi \eps_0} \frac{1}{b}
\]
con:
\[
  b =\frac{zZ e^2}{4 \pi \eps_0}  \frac{\hbar c}{\hbar c} \frac{1}{\frac{1}{2}
  m v^2} = \frac{zZ \alpha \hbar c}{\frac{1}{2} m v^2} \simeq
  \frac{\SI{1.4}{\MeV}}{\frac{1}{2} m v^2} zZ\, \si{\femto\m}
\]

In conclusione otteniamo:
\[
  G = \frac{2}{\hbar} \sqrt{\frac{mzZ e^2 b}{2 \pi \eps_0}} \mcb{\frac{\pi}{2}
  - 2 x ^{\nicefrac{1}{2}}} = \frac{4 \mrb{zZ \alpha}}{\frac{v}{c}}
  \mcb{\frac{\pi}{2} - 2 x ^{\nicefrac{1}{2}}}
\]
dove nel primo termine del prodotto, $\nicefrac{v}{c} \propto T
_{\alpha}^{\nicefrac{1}{2}}$.
Il valore del \textit{fattore di penetrazione di Gammon}:
\[
  G = \frac{2 \pi zZ \alpha}{\frac{v}{c}} -4 \sqrt{\frac{2 \mrb{mc}^{2} zZ
  \alpha R}{\hbar c}}
\]
Osserviamo che:
\begin{itemize}
  \item $\frac{2 \pi zZ \alpha}{\frac{v}{c}}$ \textit{non dipende da $R$}!
    Quindi i dettagli della buca di potenziale non sono importanti per
    calcolare la probabilità di tunneling, nel limite di barriera spessa;
  \item $4 \sqrt{\frac{2 \mrb{mc^2} zZ \alpha R}{\hbar c}}$ \textit{dipende
    da $R$}.
\end{itemize}

\begin{note}[]
  È molto più facile far interagire, con \textit{interazioni forti}, nuclei
  leggeri, rispetto a nuclei pesanti, a causa dell'altezza della barriera
  coulombiana.
\end{note}

Quindi, ricapitolando, per \textit{barriera coulombiana} in limite di
\textit{barriera spessa}, l'effetto tunneling di particelle $\alpha$ è
descritto, approssimativamente, da:
\begin{align*}
  \boxed{
    \begin{dcases}
      P_T = e^{-G} &\textit{Probabilità di tunneling}
      \\
      G = \frac{2 \pi zZ \alpha}{\frac{v}{c}} - 4 \sqrt{\frac{2 \mrb{mc^2} zZ
      \alpha R}{\hbar c}} &\textit{Fattore di penetrazione di Gammon}
    \end{dcases}
  }
\end{align*}

\paragraph{Rate di decadimento}
% TODO: riascolta la descrizione che è utile a capire
Per determinare il \textit{rate di decadimento}:
\[
  \omega = F P_T
\]
dove $\msb{\omega} = t^-1$ e $\msb{F} = t^-1$, con:
\[
  F \sim \frac{v_0}{2R}
\]
dove $v_0$ è la \textit{velocità della particella nella buca di potenziale}.
Notiamo che:
\[
  \ln \mrb{\nicefrac{\omega}{\si{s^{-1}}}} = \ln
  \mrb{\nicefrac{F}{\si{s^{-1}}}} + \ln \mrb{P_T}
  = \ln \mrb{\nicefrac{F}{\si{s^{-1}}}} - G 
  = \ln \mrb{\nicefrac{F}{\si{s^{-1}}}} + 4 \sqrt{\frac{2 \mrb{mc^2} zZ \alpha
  R}{\mrb{\hbar c}}} - \frac{2 \pi zZ \alpha}{\frac{v}{c}}
\]
indicando con:
\[
  f = \ln \mrb{\nicefrac{F}{\si{s^{-1}}}} + 4 \sqrt{\frac{2 \mrb{mc^2} zZ \alpha
  R}{\mrb{\hbar c}}} 
\]
allora:
\[
  \ln \mrb{\nicefrac{\omega}{\si{s^{-1}}}} = f - g \frac{Z}{\sqrt{T _{\alpha}}}
\]
con $T _{\alpha} = \frac{1}{2} m v^2 \simeq Q _{\alpha}$.
La costante $g$ è data da:
\[
  g = 4 \pi \alpha \sqrt{\frac{mc^2}{2}} \simeq \SI{3.97}{\MeV
  ^{\nicefrac{1}{2}}}
\]
Quindi:
\[
  \ln \mrb{\nicefrac{\omega}{\si{s^{-1}}}} = f - \SI{3.97}{\MeV
  ^{\nicefrac{1}{2}}} \frac{Z}{\sqrt{T _{\alpha}}}
\]

Quindi vediamo che $\omega$ è una \textit{funzione fortemente crescente di $T
_{\alpha}$}:
\[
  \boxed{
    \omega \propto \exp \mcb{-3.97 Z \mrb{\frac{T
    _{\alpha}}{\SI{1}{\MeV}}}^{-\frac{1}{2}}}
  }
\]
Quindi aumentando l'energia cinetica avremo che la probabilità di tunneling
decresce esponenzialmente. Questo spiega un po' il significato della
\textit{legge di Geiger-Nuttal}.
\begin{note}[]
  Per fare queste considerazioni abbiamo considerato il \textit{\textbf{nucleo
  filglio non rinculante}} (i.e. $M_D \gg m$).
\end{note}

\paragraph{Tunneling attraverso barriera coulombiana con rinculo del nucleo}
Vediamo ora il caso più generale di \textit{nucleo rinculante}. Introducendo la
\textit{massa ridotta}:
\[
  \mu = \frac{m_D m _{\alpha}}{m_D + m _{\alpha}}
\]
e trasformando $v$ nella \textit{velocità relativa}.
Il \textit{fattore di penetrazione di Gammon} diventa:
\[
  G = \frac{2 \pi \alpha zZ}{\frac{v}{c}} -4 \sqrt{\frac{2 \mrb{\mu c^2} zZ
  \alpha R}{\hbar c}}
\]
In questo caso avremo che $T _{\alpha} \neq Q _{\alpha}$. Quindi:
\[
  Q _{\alpha} = \frac{1}{2} \mu v^2
  \mthen
  2 Q _{\alpha} = \mu v^2
  \mthen
  \frac{v}{c} = \frac{1}{c} \sqrt{\frac{2 Q _{\alpha}}{\mu}}
  = \frac{1}{c} \sqrt{\frac{2 Q _{\alpha}}{m} \frac{m}{\mu}}
\]
Utilizzando queste relazioni otteniamo:
% TODO: copia
\[
  \ln \mrb{\nicefrac{\omega}{\si{s^{-1}}}} = \ln
  \mrb{\nicefrac{F}{\si{s^{-1}}}} - G = \ln
  \mrb{\nicefrac{F}{\si{s^{-1}}}} + 4\sqrt{\frac{2 \mrb{mc^2} zZ \alpha
  R}{\hbar c}} - \frac{2 \pi zZ \alpha}{\frac{v}{c}}
\]
dove, inndicando con $f'$:
\[
  f' = \ln \mrb{\nicefrac{F}{\si{s^{-1}}}} + 4\sqrt{\frac{2 \mrb{mc^2} zZ
  \alpha R}{\hbar c}}
\]

\begin{note}[]
  Noi abbiamo considerato il problema come unidimensionale, ma il processo
  reale è in realtà tridimensionale.
  % TODO: riascolta riguardo conservazione del momento angolare
\end{note}

\paragraph{Ruolo del momento angolare}
Dato il processo di \textit{decadimento $\alpha$}:
\[
  \mrb{Z, A} \rightarrow \mrb{Z - 2, A - 4} + \ch{^{4}He}
\]
abbiamo che necessariamente deve valere il \textit{principio di conservazione
del momento angolare}, quindi consideriamo:
\begin{itemize}
  \item $\vec{J}_{P}$ lo \textit{psin del nucleo padre};
  \item $\vec{J}_{D}$ lo \textit{spin del nucldeo filgio} (\textit{daughter});
  \item $\vec{J}_{\alpha} = 0$ lo \textit{spin della particella $\alpha$};
  \item $\vec{L}$ il \textit{momento angolare orbitale dello stato finale};
\end{itemize}
quindi dovrà essere:
\[
  \vec{J}_{P} = \vec{J}_{D} + \cancel{\vec{J}_{\alpha}} + \vec{L} = \vec{J}_{D}
  + \vec{L}
\]
con:
\[
  \abs{J_D - l} \leq J _\text{TOT} \leq J_D + l
\]
La relazione precedente implica che se $J_P \neq J_D$, allora $l \neq 0$:
\[
  \abs{J_P - J_D} \leq l \leq \abs{J_P + J_D}
\]
con $l = 0, 1, 2, \dots$
Se $l = 0$ dobbiamo modificare la trattazione fatta fino ad ora.
La \textit{funzione d'onda che descrive il moto relativo} soddisfa l'equazione
di Schrodinger:
\[
  - \frac{\hbar^2}{2 \mu} \nabla \psi \mrb{\vec{r}\,} + V \mrb{\vec{r}\,} \psi
  \mrb{\vec{r}\,} = Q _{\alpha} \psi \mrb{\vec{r}\,}
\]
dove $V \mrb{\vec{r}\,}$ è un \textit{potenziale che tiene conto della
repulsione coulombiana e delle interazioni forti}. L'ipotesi è che questo sia
un \textit{potenziale centrale}:
\[
  V \mrb{\vec{r}\,} = V \mrb{\abs{\vec{r}\,}}
\]
La funzione d'onda può essere scomposta in \textit{armoniche sferiche}:
\[
  \psi \mrb{\vec{r}\,} = R \mrb{r} Y _{lm} \mrb{\theta, \varphi}
\]
con $l = 0, 1, 2, \dots$ e $-l \leq m \leq l$.
Definendo $\rho \mrb{r} = r R \mrb{r}$, si ottiene la seguente equazione per
$\rho \mrb{r}$:
\[
  - \frac{\hbar^2}{2 \mu} \mdv[2]{}{r} \rho \mrb{r} + \mcb{V \mrb{r} + \frac{l
  \mrb{l + 1} \hbar^2}{2 \mu r^2}} \rho \mrb{r} = Q _{\alpha} \rho \mrb{r}
\]
dove $V \mrb{r}$ è il \textit{potenziale effettivo centrifugo}.
\[
  \frac{l \mrb{l + 1} \hbar^2 c^2}{2 \mu r^2 c^2} = \frac{l \mrb{l + 1}
  \mrb{\SI{200}{\MeV \femto\m}^{2}}}{\SI{8e3}{\MeV} r^2} \sim \frac{l \mrb{l +
  1} \SI{5}{\MeV}}{\mrb{\nicefrac{r}{\SI{1}{\femto\m}}}^{2}}
\]
Per $l \neq 0$ otteniamo un \textit{innalzamento della barriera di potenziale},
quindi un effetto tunneling meno probabile. In più:
\begin{itemize}
  \item anche se sotto-dominante, il termine aggiuntivo può modificare
    sostanzialmente il valore di $P_T$;
  \item \quot{\textit{effetto centrifugo}} presente anche in assenza di
    repulsione coulombiana (importante per reazioni nucleari).
\end{itemize}
