%LTeX: language=it
\chapter{2022-05-31}
\section{Reazioni nucleari}
\begin{definition}[Processo di scattering]
  Parliamo di un \textit{\textbf{processo di scattering}} se la particella
  incidente è presente nei prodotti di reazione:
  \[
    a + A \longrightarrow b + B
    \qquad
    \textit{se } a = b \Rightarrow \textit{scattering}
  \]
  Parliamo di \textit{scattering \textbf{elastico}} se $a = b$ e $A = B$,
  mentre parliamo di \textit{scattering \textbf{inelastico}} se $A \neq B$.
\end{definition}

\begin{note}[]
  Ricordiamo che deve valere la \textit{conservazione del \textbf{numero
  barionico}} (per quanto visto i nucleoni sono barioni).
\end{note}

\begin{note}[]
  Noi qui faremo l'ipotesi che l'interazione avvenga fra nuclei
  \textit{non-realtivistiche}.
\end{note}

\subsection{Cineamtica di un processo di collisione (non-relavitisca)}
\[
  a + A \longrightarrow b + B
\]

\paragraph{Sistema del laboratorio}
Nel \textit{sistema del laboratorio} $\vec{v}_{A} = 0$.
\begin{align*}
  T_A = \frac{1}{2} m_a v_{a}^{2} = 0
  \\
  \vec{p}_{A} = m_a \vec{v}_{a} = 0
\end{align*}
Energia cinetica ed impulso totali:
\begin{align*}
  T_\text{TOT} = T_a + T_A = \frac{1}{2} m_a \vec{v}_{a}^2
  \\
  p_\text{TOT} = \vec{p}_{a} + \vec{p}_{A} = m \vec{v}_{a}
\end{align*}
\begin{note}[]
  Il sistema del centro di massa è definito come il sitema di riferimento in
  cui l'impulso totale delle particelle risultanti è nullo.
\end{note}
La velocità del \textit{CdM} nel sistema del laboratorio sarà:
\[
  \vec{r}_\text{CM} = \frac{m_a \vec{r}_{a} + m_A \vec{r}_{A}}{m_a + m_A}
  \mthen
  v _\text{CM} = \frac{m_a \vec{v}_{a} + m_A \vec{v}_{A}}{m_a + m_A} =
  \frac{m_a}{m_a + m_A} \vec{v}_{a}
\]

\paragraph{Sistema del centro di massa}
Nel sistema del centro di massa:
\[
  \begin{dcases}
    \vec{\tilde p}_\text{TOT} = \vec{\tilde p}_{a} + \vec{\tilde p}_{A} =
    \vec{0}
    \\
    \vec{\tilde v}_\text{CM} = \vec{0}
  \end{dcases}
\]
e abbiamo che:
\[
  \vec{\tilde p}_{a} = -\vec{\tilde p}_{A}
  \mthen
  \abs{\vec{\tilde p}_{a}} = \abs{\vec{\tilde p}_{A}} = \tilde p
\]
Per le velocità:
\[
  m_a \vec{\tilde v}_{a} = - m_A \vec{\tilde v}_{A}
  \mthen
  \vec{\tilde v}_{A} = - \frac{m_a}{m_A} \vec{\tilde v}_{a}
\]
Per l'energia cinetica:
\begin{align*}
  \tilde T_a = \frac{1}{2} m_a \tilde v_a^2 = \frac{\tilde p^2}{2 m_a}
  \\
  \tilde T_A = \frac{1}{2} m_A \tilde v_A^2 = \frac{\tilde p^2}{2 m_A}
\end{align*}

Consideriamo che:
\[
  \vec{\tilde v}_{a} = \vec{v}_{a} - \vec{v}_\text{CM} = \mrb{1 -
  \frac{m_a}{m_a + m_A}} \vec{v}_{a} = \frac{m_A}{m_a + m_A} \vec{v}_{a}
\]
ciò implica che:
\[
  \vec{\tilde p}_{A} = - \vec{\tilde p}_{a} = - \frac{m_a}{m_a + m_A}
  \vec{p}_{A}
\]
quindi:
\[
  \vec{\tilde p}_{A} = - \vec{\tilde p}_{a} = - \frac{m_a}{m_a + m_A}
  \vec{P}_{A}
\]
L'\textit{energia cinetica del moto relativo}:
\[
  \tilde T = \tilde T_a + \tilde T_A = \frac{\tilde p^2}{2 m_a} + \frac{\tilde
  p^2}{2 m_A} = \frac{\tilde p^2}{2} \mcb{\frac{1}{m_a} + \frac{1}{m_A}} =
  \frac{\tilde p^2}{2} \frac{1}{\mu}
\]
dove $\mu$ è la \textit{massa ridotta}:
\[
  \mu = \frac{m_a m_A}{m_a + m_A}
\]
Quindi:
\[
  \tilde T = \frac{1}{2 \mu} \tilde p^2 = \frac{1}{2 \mu} \mrb{\frac{m_a}{m_a +
  m_A}}^{2} \mrb{\frac{p_a^2}{2 m_a}} 2 m_a = \frac{m_A}{m_a + m_A} T
\]

In tutti i processi fisici dovranno essere conservati \textit{energia} e
\textit{impulso}:
\begin{align*}
  \textit{Conservazione dell'energia} \qquad \mrb{E _\text{tot}}_\text{ini} = E
  \mrb{ _\text{tot}}_\text{fin}
  \\
  \textit{Conservazione impulso} \qquad \mrb{\vec{p}_\text{tot}}_\text{ini} =
  \mrb{\vec{p}_\text{tot}}_\text{fin}
\end{align*}
Ricordando che stiamo considerando una reazione del tipo:
\[
  a + A \longrightarrow b + B
\]
Nel sitema del centro di massa:
\[
  \mrb{\vec{p}_\text{tot}}_\text{ini} = \mrb{\vec{p}_\text{tot}}_\text{fin} =
  \vec{0}
\]
quindi:
\[
  \begin{rcases}
    \abs{\vec{\tilde p}_{a}} = \abs{\vec{\tilde p}_{A}} = \tilde p_\text{ini}
    \\
    \abs{\vec{\tilde p}_{b}} = \abs{\vec{\tilde p}_{B}} = \tilde p_\text{fin}
  \end{rcases}
  \mthen
  \begin{dcases}
    \vec{p}_{a} = - \vec{p}_{A}
    \\
    \vec{p}_{b} = - \vec{p}_{B}
  \end{dcases}
\]
implementa la \textit{conservazione dell'impulso}.

La \textit{conservazione dell'energia implica che}:
\[
  m_a c^2 + \tilde T_a + m_A c^2 + \tilde T_A = m_b c^2 + \tilde T_b + m_B c^2
  + \tilde T_B
\]
quindi definiamo il \textit{$Q$-value} della reazione:
\[
  Q = \mrb{m_a c^2 + m_A c^2} - \mrb{m_b c^2 + m_B c^2} = \mrb{\tilde T_b +
  \tilde T_B} - \mrb{\tilde T_a + \tilde T_A} \equiv \Delta \tilde T
\]
dove:
\begin{align*}
  \begin{dcases}
    m_a c^2 + m_A c^2 \Rightarrow \textit{Somma delle masse dello stato ini.}
    \\
    m_b c^2 + m_B c^2 \Rightarrow \textit{Somma delle masse dello stato fin.}
    \\
    \tilde T_b + \tilde T_B \Rightarrow \textit{Ener. cin. totale dello stato
    fin.}
    \\
    \tilde T_a + \tilde T_A \Rightarrow \textit{Ener. cin. totale dello stato
    ini.}
  \end{dcases}
\end{align*}
Il \textit{$Q$-value} della reazione:
\[
  Q = \msum{\textit{ini}}{} m_i c^2 - \msum{\textit{fin}}{} m_f c^2 = \Delta
  \tilde T = \mrb{\tilde T_\text{tot}}_\text{fin} -
  \mrb{\tilde T_\text{tot}}_\text{ini}
\]
quindi:
\begin{align*}
  Q < 0 \Rightarrow \mrb{\tilde T_\text{tot}}_\text{fin} > \mrb{\tilde
  T_\text{tot}}_\text{ini} \qquad \textit{Reazione esotermica}
\end{align*}
con $\mrb{\tilde T_\text{tot}}_\text{fin} = \mrb{\tilde
T_\text{tot}}_\text{ini} + Q$.
% TODO: scrivi reazione endotermica

\subsection{Fusione nucleare}
Il processo di \textit{\textbf{fusione nucleare}} è un processo in cui abbiamo
la fusione di nuclei:
\[
  \mrb{Z_1, A_1} + \mrb{Z_2, A_2} \rightarrow \mrb{Z_1 + Z_2, A_1 + A_2} +
  \gamma
\]
Il \textit{$Q$-value} di tale reazione:
\begin{align*}
  Q &= m \mrb{Z_1, A_1} c^2 + m \mrb{Z_2, A_2} c^2 - m \mrb{Z_1 + Z_2, A_1 +
  A_2} c^2
  \\
  &= Z_1 m_p c^2 + \mrb{A_1 - Z_1} m_n c^2 - B \mrb{Z_1, A_1}
  \\
  & + Z_2 m_p c^2 +
  \mrb{A_2 - Z_2} m_n c^2 - B \mrb{Z_2, A_2}
  \\
  & - \mrb{Z_1 + Z_2} m_p c^2 + \mrb{A_1 + A_2 - Z_1 - Z_2} m_n c^2 - B
  \mrb{Z_1 + Z_2, A_1 + A_2}
  \\
  &= B \mrb{Z_1 + Z_2, A_1 + A_2} - B \mrb{Z_1, A_1} - B \mrb{Z_2, A_2}
  \\
  &= \eps \mrb{A_1 + A_2} \mrb{A_1 + A_2} - \eps \mrb{A_1} A_1 - \eps \mrb{A_2}
  A_2
  \\
  &\geq \eps \mrb{A_1 + A_2} \mrb{A_1 + A_2} - \max \mrb{\eps \mrb{A_1}, \eps
  \mrb{A_2}} \mrb{A_1 + A_2}
\end{align*}
quindi:
\[
  Q \geq \msb{\eps \mrb{A_1 + A_2} - \max \mrb{\eps \mrb{A_1}, \eps \mrb{A_2}}}
  \mrb{A_1 + A_2} \geq 0
\]
dove il termine in parentesi quadra è un termine positivo, poiché $\eps
\mrb{A}$ è una funzione crescente di $A$ per $A \leq 56$.
Quindi: \textit{la \textbf{fusione nucleare di nuclei leggeri} è un
\textbf{processo esotermico}}.
% TODO: Riascolta tante cose imporanti sulla fusione nucleare!
\begin{note}[]
  Generalmente, in condizioni \quot{\textit{normali}} non avvengono fusioni
  spontanee fra nuclei (se non ad esempio nelle stelle), o meglio sono
  altamente improbabili, causa effetto tunnel nella barriera coulombiana. Per
  questo motivo, per farle avvenire abbiamo bisogno di collisioni ad alta
  energia.
\end{note}

\subsection{Fissione nucleare}
% TODO: 

\subsection{Considerazioni sulle reazioni nucleari}
Affinché una reazione nucleare:
\[
  \mrb{Z_1, A_1} + \mrb{Z_2, A_2} \rightarrow \mrb{Z', A'} + \dots
\]
possa avvenire è necessario portare i due nuclei a distanze \textit{piccole}
($\sim \si{\femto\m}$). Se i nuclei sono carichi, ciò è ostacolato dalla
repulsione coulombiana.
% TODO: figura
La \textit{sezione d'urto} di un processo di fusione fra due nuclei sarà:
\[
  \sigma = \tilde \sigma P_T
\]
con $\tilde \sigma \sim \si{\femto\m^2}$ (\textit{stima geometrica}), dove
$P_T$ è la \textit{probabilità di tunneling attraverso la barriera
coulombiana}:
\[
  P_T = e^{-G}
\]
dove $G$ è il \textit{fattore di penetrazione di Gammon}, che in questo caso:
\[
  G = \frac{2 \pi \mrb{Z_1 Z_2 \alpha}}{\frac{v}{c}} - 4 \sqrt{\frac{2
  \mrb{\mu c^2} Z_1 Z_2 \alpha R}{\hbar c}}
\]
quindi $P_T$ è il fattore che ci dice che un processo spontaneo di fusione
nucleare è altamente improbabile.
\begin{note}[]
  La barriera coulombiana, chiaramente, è assente per reazioni indotte da
  neutroni\footnote{
    Neutroni liberi pronti a reagire sono poco probabili, perché i
    \textit{neutroni liberi sono instabili}
  }. 
\end{note}
\begin{note}[]
  Se la collisione avviene con $l \neq 0$, dobbiamo tenere conto anche della
  \textit{barriera centrifuga}:
  \[
    V_C \mrb{r} = \frac{l \mrb{l + 1} \hbar^2}{2 \mu r^2}
  \]
\end{note}

\subsection{Fusione dell'idrogeno nel Sole}
Il processo di fusione dell'idrogeno nel sole è un processo del tipo:
\[
  2 e^- + 4p \rightarrow \ch{^4He} + 2 \nu_e
\]
dove la presenza $\nu_e$ soddisfa la legge di conservazione fondamentale di
\textit{conservazione di numero leptonico}. Questo è il motivo per cui il sole
è una \textit{sorgente di neutrini di tipo $e$}.
\[
  Q = \msb{2m_e + 4 m_p - m \mrb{\ch{^{4}He}}} \simeq \SI{26}{\MeV}
\]

Il flusso di neutrini prodotto a terra (sulla terra):
\[
  \phi _{\nu} \sim \SI{6e10}{\cm^-2 s^-1}
\]

\paragraph{Catena $pp$}
Dati i primi due processi:
\begin{align*}
  p + p \rightarrow d + e^+ + \nu_e
  \\
  p + d \rightarrow \ch{^{3}He} + \gamma
\end{align*}
possiamo ottenere \textit{Terminazione $ppI$} ($90\%$ della reazione):
\begin{align*}
  \ch{^{3}He} + \ch{^{3}He} \rightarrow \ch{^{4}He} + 2p
\end{align*}
o le \textit{terminazioni $ppII$} ($10\%$):
% \begin{align*}
%   \ch{^{3}He + \ch{^{4}He}} \rightarrow \ch{^{7}Be} + \gamma
%   \\
%   \ch{^{7}Be} + e^- \rightarrow \ch{^{7}Li} + \nu_e
%   % \\
%   % ??? % TODO: 
% \end{align*}
o $ppIII$:
% \begin{align*}
%   ??? % TODO: 
% \end{align*}
