%LTeX: language=it
\chapter{Esercitazione 2022-03-31}
\begin{example}[]
	Fascio di pioni $\pi$ con energia totale $E _{\pi}$ su un bersaglio di
	$\ch{^{2}H}$. Viene prodotta una risonanza:
	\begin{equation}
		M \rightarrow m_1 + m_2
	\end{equation}
	Sapendo che $M = 2.58 m_1$ e che $m_2$ è trascurabile rispetto a $m_1$ ($m_2
		\gg m_1$):
	\begin{enumerate}
		\item Qual è $E _{\pi}$ minima per avere un angolo massimo (\textbf{Lab})
		      per la particella $1$ ($m_1$).
		\item La particella $M$ sarà una particella $\Delta (2420)$ di massa
		      $\qty{2420}{\MeV \per c^2}$.
		      \begin{equation}
			      \Delta (2420) \rightarrow \Sigma + k
		      \end{equation}
		      Le masse sono: $m _{\Sigma} = \qty{1.189}{\GeV \per c^2}$ e $m _{k} =
			      \qty{0.494}{\GeV \per c^2}$.
		      Se la $\Sigma$ viene emessa ad un angolo di $\qty{120}{\degree}$ nel
		      sistema del CM (data l'energia del fascio trovata nel punto $1$) $E
				      _{T} = \qty{2.65}{\GeV}$:
		      \begin{enumerate}
			      \item qual è l'angolo $\theta$ (sistema del laboratorio) che
			            corrisponde all'angolo $\theta^\ast = \qty{120}{\degree}$ nel
			            sistema del centro di massa?
			      \item a quale impulso $p$ (sistema del laboratorio) corrisponde?
		      \end{enumerate}
		\item se un rilevatore lungo \qty{26}{\cm} vede il \qty{99}{\%} dei punti
		      di decadimento della particella $\Sigma$, allora qual è il tempo di
		      vita medio della $\Sigma$?
	\end{enumerate}

	\paragraph{Soluzione}
	\begin{enumerate}
		\item La condizione da soddisfare sarà:
		      \begin{equation}
			      \beta _\text{CM} \geq \beta _{1}^{\ast}
		      \end{equation}
		      L'impulso delle due particelle:
		      \begin{equation}
			      \abs{\vec{p}_{1}^{\,\ast}}
			      = \abs{\vec{p}_{2}^{\,\ast}}
			      = \abs{\vec{p}^{\,\ast}}
			      = \frac{\sqrt{\msb{M^2 - \mrb{m_1 + m_2}^{2}}
					      \msb{M^2 - \mrb{m_1 + m_2}^{2}}}}{2M}
		      \end{equation}
		      L'energia della particella $1$:
		      \begin{equation}
			      \eps _{1}^{\ast} = \frac{M^2 + m_1^2 - m_2^2}{2M}
		      \end{equation}
		      quindi:
		      \begin{equation}
			      p_1^\ast
			      = \frac{p ^{\ast}}{\eps _{1}^{\ast}}
			      = \frac{\sqrt{\mrb{M^2 - m_1^2} \mrb{M^2 - m_1^2}}}{M^2 + m_1^2}
			      = \frac{M^2 - m_1^2}{M^2 + m_1^2}
			      = \frac{2.58^2 - 1}{2.58^2 + 1}
			      = 0.7388
		      \end{equation}
		      Il beta del centro di massa:
		      \begin{equation}
			      \beta _\text{CM}
			      = \frac{\abs{\vec{p} _{\pi}}}{E _{\pi} + m_p}
			      = \frac{\sqrt{E^2 _{\pi} - m _{\pi}^{2}}}{E _{\pi} + m_p}
			      = \beta_1^\ast
		      \end{equation}
		      quindi $E _{\pi}$:
		      \begin{equation}
			      E _{\pi}
			      = \frac{
				      \beta _{1}^{\ast} m_p
				      + \sqrt{\mrb{\beta_1^\ast m_p}^{2}
					      + \mrb{1 - \beta_1^\ast} \mrb{
						      m _{\pi}^{2} + \beta_1 ^{\ast} m_p}
				      }}{1 - \beta_1 ^{\ast\,2}}
			      = \qty{2.65}{\GeV}
		      \end{equation}

		\item Dato che $\theta^\ast = \qty{120}{\degree}$:
		      \begin{equation}
			      \beta _\text{CM} = \beta^\ast_1 = 0.7388
		      \end{equation}
		      \begin{equation}
			      \eps _{\Sigma}^{\ast}
			      = \sqrt{p ^{\ast\, 2} + m _{\Sigma}^{2}}
			      = \qty{1.452}{\GeV}
		      \end{equation}
		      L'impulso $p^\ast$:
		      \begin{equation}
			      p^\ast = \qty{0.833}{\GeV \per c}
		      \end{equation}
		      \begin{equation}
			      \begin{dcases}
				      \mrb{p _{\Sigma}}_{L}
				      = \gamma _\text{CM} \mrb{
					      p ^{\ast} \cos \theta^\ast + \beta _\text{CM} \eps _{\Sigma}^{\ast}
				      }
				      = \qty{0.974}{\GeV \per c}
				      \\
				      \mrb{p _{\Sigma}}_{T}
				      = p^\ast \sin \theta^\ast = \qty{0.721}{\GeV \per c}
			      \end{dcases}
		      \end{equation}
		      Quindi $p _{\Sigma}$:
		      \begin{equation}
			      p _{\Sigma}
			      = \sqrt{\mrb{p _{\Sigma}}_{L}^{2} + \mrb{p _{\Sigma}}_{T}^{2}}
			      = \qty{1.21}{\GeV \per c}
		      \end{equation}
		      L'angolo $\theta _{\Sigma}$ relativo al sistema del laboratorio che
		      stavamo cercando:
		      \begin{equation}
			      \theta _{\Sigma}
			      = \arccos \msb{\frac{\mrb{p _{\Sigma}}_{T}}{p _{\Sigma}}}
			      = \qty{36.5}{\degree}
		      \end{equation}

		\item La distanza media percorsa dalle particelle $\Sigma$ prima del
		      decadimento può essere espresso come:
		      \begin{equation}
			      \braket{L}
			      = c \tau _{\Sigma} \beta _{\Sigma} \gamma _{\Sigma}
			      = c \tau _{\Sigma} \frac{p _{\Sigma}}{m _{\Sigma}}
		      \end{equation}
		      A questo punto, consideriamo:
		      \begin{equation}
			      p _{\Sigma}^\text{max}
			      = \gamma _\text{CM} \mrb{p ^{\ast}
				      + \beta _\text{CM} \eps _{\Sigma}^{\ast}}
			      = \qty{2.83}{\GeV}
		      \end{equation}
		      Il \qty{99}{\%} dgli eventi di decadimento corrisponde a:
		      \begin{equation}
			      \frac{N}{N _\text{tot}} = 0.99
		      \end{equation}
		      e ricordando che il decadimento di particelle è associato a legge
		      esponenziale:
		      \begin{equation}
			      \mint{0}{\infty}{t}{N \mrb{t}}
			      = N _{0} \mint{0}{\infty}{t}{e^{- \frac{t}{\tau _{\Sigma}}}}
			      = N _{0} \tau _{\Sigma}
		      \end{equation}
		      per cui:
		      \begin{equation}
			      0.99
			      = \frac{1}{N _{0} \tau _{\Sigma}} \mint{0}{T}{t}{N \mrb{t}}
			      = \frac{1}{\cancel{N_0} \tau _{\Sigma}}
			      \cancel{N_0} \mint{0}{T}{t}{e^{- \frac{t}{\tau _{\Sigma}}}}
			      = 1 - \mint{T}{\infty}{t}{\frac{1}{\tau _{\Sigma}}
				      e^{- \frac{t}{\tau _{\Sigma}}}}
		      \end{equation}
		      Quindi:
		      \begin{equation}
			      0.99 = 1 - e^{\frac{T}{\tau _{\Sigma}}}
			      \mthen
			      T = - \ln \mrb{0.01} \tau _{\Sigma} = 4.6 \tau _{\Sigma}
		      \end{equation}
		      Ora conosciamo la vita media della particella $\Sigma$, quindi
		      sostituendo nella lunghezza percorsa prima del decadimento:
		      \begin{equation}
			      L
			      = c T \frac{p _{\Sigma}^\text{max}}{m _{\Sigma}}
			      = 4.6 \tau _{\Sigma} \frac{p _{\Sigma}^\text{max}}{m _{\Sigma}}
		      \end{equation}
		      e dunque:
		      \begin{equation}
			      \tau _{\Sigma}
			      = \frac{L m_{\Sigma}}{4.6 c\, p _{\Sigma}^\text{max}}
			      = \frac{0.26 \cdot 1.189}{4.6 \cdot \qty{3e8} \cdot 2.83}\, \si{\s}
			      = \qty{0.79e-10}{\s}
		      \end{equation}
	\end{enumerate}
\end{example}
