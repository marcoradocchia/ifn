%LTeX: language=it
\chapter{2022-03-31}
\begin{example}[]
  Fascio di pioni $\pi$ con energia totale $E _{\pi}$ su un bersaglio di
  $\ch{^{2}H}$. Viene prodotta una risonanza:
  \[
    M \rightarrow m_1 + m_2
  \]
  Sapendo che $M = 2.58 m_1$ e che $m_2$ è trascurabile rispetto a $m_1$ ($m_2
  \gg m_1$):
  \begin{enumerate}
    \item Qual è $E _{\pi}$ minima per avere un angolo massimo (\textbf{Lab})
      per la particella $1$ ($m_1$).
    \item La particella $M$ sarà una particella $\Delta (2420)$ di massa
      $\SI{2420}{\MeV \per c^2}$.
      \[
        \Delta (2420) \rightarrow \Sigma + k
      \]
      Le masse sono: $m _{\Sigma} = \SI{1.189}{\GeV \per c^2}$ e $m _{k} =
      \SI{0.494}{\GeV \per c^2}$.
      Se la $\Sigma$ viene emessa ad un angolo di $\SI{120}{\degree}$ nel
      sistema del CM (data l'energia del fascio trovata nel punto $1$) $E _{T}
      = \SI{2.65}{\GeV}$:
      \begin{enumerate}
        \item qual è l'angolo $\theta$ (sistema del laboratorio) che
          corrisponde all'angolo $\theta^\ast = \SI{120}{\degree}$ nel
          sistema del centro di massa?
        \item a quale impulso $p$ (sistema del laboratorio) corrisponde?
      \end{enumerate}
    \item se un rilevatore lungo \SI{26}{\cm} vede il \SI{99}{\%} dei punti di
      decadimento della particella $\Sigma$, allora qual è il tempo di vita
      medio della $\Sigma$?
  \end{enumerate}

  \paragraph{Soluzione}
  \begin{enumerate}
    \item La condizione da soddisfare sarà:
      \[
        \beta _\text{CM} \geq \beta _{1}^{\ast}
      \]
      L'impulso delle due particelle:
      \[
        \abs{\vec{p}_{1}^{\,\ast}} = \abs{\vec{p}_{2}^{\,\ast}} =
        \abs{\vec{p}^{\,\ast}} = \frac{\sqrt{\msb{M^2 - \mrb{m_1 + m_2}^{2}}
        \msb{M^2 - \mrb{m_1 + m_2}^{2}}}}{2M}
      \]
      L'energia della particella $1$:
      \[
        \eps _{1}^{\ast} = \frac{M^2 + m_1^2 - m_2^2}{2M}
      \]
      quindi:
      \[
        p_1^\ast = \frac{p ^{\ast}}{\eps _{1}^{\ast}} = \frac{\sqrt{\mrb{M^2 -
        m_1^2} \mrb{M^2 - m_1^2}}}{M^2 + m_1^2} = \frac{M^2 - m_1^2}{M^2 +
        m_1^2} = \frac{2.58^2 - 1}{2.58^2 + 1} = 0.7388
      \]
      Il beta del centro di massa:
      \[
        \beta _\text{CM} = \frac{\abs{\vec{p} _{\pi}}}{E _{\pi} + m_p} =
        \frac{\sqrt{E^2 _{\pi} - m _{\pi}^{2}}}{E _{\pi} + m_p} = \beta_1^\ast
      \]
      quindi $E _{\pi}$:
      \[
        E _{\pi} = \frac{\beta _{1}^{\ast} m_p + \sqrt{\mrb{\beta_1^\ast
        m_p}^{2} + \mrb{1 - \beta_1^\ast} \mrb{m _{\pi}^{2} + \beta_1 ^{\ast}
        m_p}}}{1 - \beta_1 ^{\ast\,2}} = \SI{2.65}{\GeV}
      \]

    \item Dato che $\theta^\ast = \SI{120}{\degree}$:
      \[
        \beta _\text{CM} = \beta^\ast_1 = 0.7388
      \]
      \[
        \eps _{\Sigma}^{\ast} = \sqrt{p ^{\ast\, 2} + m _{\Sigma}^{2}} =
        \SI{1.452}{\GeV}
      \]
      L'impulso $p^\ast$:
      \[
        p^\ast = \SI{0.833}{\GeV \per c}
      \]
      \[
        \begin{dcases}
          \mrb{p _{\Sigma}}_{L} = \gamma _\text{CM} \mrb{p ^{\ast} \cos
          \theta^\ast + \beta _\text{CM} \eps _{\Sigma}^{\ast}} =
          \SI{0.974}{\GeV \per c}
          \\
          \mrb{p _{\Sigma}}_{T} = p^\ast \sin \theta^\ast = \SI{0.721}{\GeV
          \per c}
        \end{dcases}
      \]
      Quindi $p _{\Sigma}$:
      \[
        p _{\Sigma} = \sqrt{\mrb{p _{\Sigma}}_{L}^{2} + \mrb{p
        _{\Sigma}}_{T}^{2}} = \SI{1.21}{\GeV \per c}
      \]
      L'angolo $\theta _{\Sigma}$ relativo al sistema del laboratorio che
      stavamo cercando:
      \[
        \theta _{\Sigma} = \arccos \msb{\frac{\mrb{p _{\Sigma}}_{T}}{p
        _{\Sigma}}} = \SI{36.5}{\degree}
      \]

    \item La distanza media percorsa dalle particelle $\Sigma$ prima del
      decadimento può essere espresso come:
      \[
        \braket{L} = c \tau _{\Sigma} \beta _{\Sigma} \gamma _{\Sigma} = c \tau
        _{\Sigma} \frac{p _{\Sigma}}{m _{\Sigma}}
      \]
      A questo punto, consideriamo:
      \[
        p _{\Sigma}^\text{max} = \gamma _\text{CM} \mrb{p ^{\ast} + \beta
        _\text{CM} \eps _{\Sigma}^{\ast}} = \SI{2.83}{\GeV}
      \]
      Il \SI{99}{\%} dgli eventi di decadimento corrisponde a:
      \[
        \frac{N}{N _\text{tot}} = 0.99
      \]
      e ricordando che il decadimento di particelle è associato a legge
      esponenziale:
      \[
        \mint{0}{\infty}{t}{N \mrb{t}} = N _{0} \mint{0}{\infty}{t}{e^{-
        \frac{t}{\tau _{\Sigma}}}} = N _{0} \tau _{\Sigma}
      \]
      per cui:
      \[
        0.99 = \frac{1}{N _{0} \tau _{\Sigma}} \mint{0}{T}{t}{N \mrb{t}} =
        \frac{1}{\cancel{N_0} \tau _{\Sigma}} \cancel{N_0} \mint{0}{T}{t}{e^{-
        \frac{t}{\tau _{\Sigma}}}} = 1 - \mint{T}{\infty}{t}{\frac{1}{\tau
        _{\Sigma}} e^{- \frac{t}{\tau _{\Sigma}}}}
      \]
      Quindi:
      \[
        0.99 = 1 - e^{\frac{T}{\tau _{\Sigma}}}
        \mthen
        T = - \ln \mrb{0.01} \tau _{\Sigma} = 4.6 \tau _{\Sigma}
      \]
      Ora conosciamo la vita media della particella $\Sigma$, quindi
      sostituendo nella lunghezza percorsa prima del decadimento:
      \[
        L = c T \frac{p _{\Sigma}^\text{max}}{m _{\Sigma}} = 4.6 \tau
        _{\Sigma} \frac{p _{\Sigma}^\text{max}}{m _{\Sigma}}
      \]
      e dunque:
      \[
        \tau _{\Sigma} = \frac{L m_{\Sigma}}{4.6 c\, p _{\Sigma}^\text{max}}
        = \frac{0.26 \cdot 1.189}{4.6 \cdot \SI{3e8} \cdot 2.83}\,
        \si{\s} = \SI{0.79e-10}{\s}
      \]
  \end{enumerate}
\end{example}
