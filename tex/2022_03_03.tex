%LTeX: language=it
\chapter{2022-03-03} % 2022-03-03
\section{Relatività}
\subsection{Relatività Galileiana}
La \textit{relatività galileiana} si basa sui principi di \textit{spazio e
	tempo \textbf{assoluti}}. Possiamo definire quindi infiniti sistemi di
riferimento, tutti equivalenti. Non possiamo definire, tuttavia, in
\textit{modo assoluto}, lo stato di moto di un corpo, ma solo il suo stato di
moto in relazione al sistema di riferimento scelto.

Una caratteristica importante della \textit{relatività galileiana} è proprio il
principio di \textbf{tempo assoluto}: il tempo scorre allo stesso modo in
\textit{tutti} i sistemi di riferimento.

\subsection{Relatività ristretta di Einstein}
La \textit{relatività ristretta} di Einstein si basa su due \textit{postulati}.
\begin{enumerate}
	\item \textbf{Postulato di relatività}: le leggi della natura e i risultati
	      di tutti gli esperimenti eseguiti in un dato sistema di riferimento sono
	      indipendenti dal moto di traslazione dell'intero sistema;
	\item \textbf{Postulato di costanza della velocità della luce nel vuoto}: la
	      velocità della luce nel vuoto è la stessa in tutti i sistemi di riferimento
	      inerziali, indipendentemente dal modo della sorgente e del suo osservatore;
\end{enumerate}

\begin{note}[Simultaneità]
	Con la relatività ristretta e questi postulati nasce il problema della
	simultaneità degli eventi.
\end{note}

\subsubsection{Trasformazioni di Lorentz}
Le \textbf{Trasformazioni di Lorentz} sono una revisione compatibile con la
relatività speciale delle trasformazioni di Galileo.

Le \textit{trasformazioni di Lorentz} sono state ricavate sulla base dei 3
requisiti (\textit{vincoli di Einstein}):
\begin{enumerate}
	\item \textbf{invertibilità};
	\item \textbf{linearità} (al fine di soddisfare la condizione di
	      relatività).
	\item \textbf{velocità relativa diretta lungo un asse} (si può sempre
	      scegliere un orientamento del SR tale che sia soddisfatta questa
	      condizione).
\end{enumerate}

\begin{equation}
	\begin{dcases}
		x \mprime = a _{11} x + a _{12} y + a _{13} z + a _{14} t
		\\
		y \mprime = a _{21} x + a _{22} y + a _{23} z + a _{24} t
		\\
		z \mprime = a _{31} x + a _{32} y + a _{33} z + a _{34} t
		\\
		t \mprime = a _{41} x + a _{42} y + a _{43} z + a _{44} t
	\end{dcases}
\end{equation}

\paragraph{Trasformazione lungo l'asse $x$}
Per una trasformazione lungo l'asse $x$ avremo che:
\begin{equation}
	\begin{dcases}
		a _{22} = a _{33} = 1
		\\
		a _{21} = a _{23} = a _{24} = 0
		\\
		a _{31} = a _{32} = a _{34} = 0
		\\
		a _{12} = a _{13} = a _{42} = a _{43} = 0
	\end{dcases}
\end{equation}
Quindi le trasformazioni di coordinate diventano:
\begin{equation}
	\label{eq:trasf_lorentz}
	\begin{dcases}
		x \mprime = a _{11} x + a _{14} t
		\\
		y \mprime = y
		\\
		z \mprime = z
		\\
		t \mprime = a _{41} x + a _{44} t
	\end{dcases}
\end{equation}

\begin{equation}
	x \mprime = 0
	\mthen
	x = vt
	\mthen
	a _{14} = -v a _{11}
\end{equation}

Utilizzando il secondo postulato ($c = \text{const}$):
\begin{equation}
	\begin{dcases}
		x^2 + y^2 + z^2 = c^2 t^2
		\\
		x^{\prime\, 2} + y^{\prime\, 2} + z^{\prime\, 2} = c^2 t^{\prime\, 2}
	\end{dcases}
\end{equation}
Rappresentano la distanza percorsa da un fotone in $t$ o $t \mprime$.

Consideriamo il sistema di riferimento primato e sostituiamo
le~\ref{eq:trasf_lorentz}:
\begin{equation}
	\mrb{a _{11} x + a _{14} t}^{2} + y^2 + z^2 = c^2 \mrb{a _{41} x + a _{44}
		t}^{2}
\end{equation}
dunque sostituendo $a _{14} = - v a _{11}$:
\begin{equation}
	\mrb{a _{11} x - v a _{11} t}^{2} + y ^{2} + z ^{2} = c ^{2} \mrb{a _{41} x +
		a _{44} t} ^{2}
\end{equation}
Svolgendo i quadrati:
\begin{equation}
	a _{11}^{2} x^2 + v^2 a _{11}^{2} t^2 - 2 a _{11}^{2} vxt + y^2 + z^2 =
	c^2 a _{41}^{2} x^2 + c^2 a _{44}^{2} t^2 + 2 a _{41} a _{44} c^2 xt
\end{equation}
quindi:
\begin{equation}
	t^2 \mrb{-v^2 a _{11}^{2} + c^2 a _{44}^{2}}
	= x^2 \mrb{a _{11}^{2} - c^2 a _{41}^{2}} + y^2 + z^2 + 2 xt \mrb{a _{41} a
			_{44} c^2 + a _{11}^{2} v}
\end{equation}
Ora, poiché dovrà valere la relazione $x^2 + y^2 + z^2 = c^2 t^2$, imponiamo:
\begin{equation}
	\begin{dcases}
		- v^2 a _{11}^{2} + c^2 a _{44}^{2} = c^2
		\\
		a _{11}^{2} - c^2 a _{41}^{2} = 1
		\\
		a _{11}^{2} v + a _{41} a _{44} c^2 = 0
	\end{dcases}
\end{equation}
Dal sistema otteniamo un'espressione esplicita per coefficienti $a _{11}, a
		_{44}, a _{41}$:
\begin{equation}
	\begin{dcases}
		a _{11} = a _{44} = \pm \frac{1}{\sqrt{1 - \mrb{\frac{v}{c}}^{2}}}
		\\
		a _{41} = \pm \frac{\frac{v}{c^2}}{\sqrt{1 - \mrb{\frac{v}{c}}^{2}}}
	\end{dcases}
\end{equation}
Definiamo i coefficienti $\beta$ e $\gamma$ di \textit{Lorentz}:
\begin{equation}
	\beta = \frac{v}{c} \quad \text{(velocità del SR in unità di $c$)};
	\qquad
	\gamma = \frac{1}{\sqrt{1 - \beta ^{2}}}
\end{equation}

\paragraph{Quadrivettori}
Un quadrivettore è (semplicisticamente) un oggetto matematico della forma:
\begin{equation}
	\qvec{X} \equiv
	\mcb{ct, x, y, z}
	= \mcb{ct, \vec{x}}
	\equiv
	\begin{pmatrix}
		x \\
		y \\
		z \\
		ct
	\end{pmatrix}
	=
	\begin{pmatrix}
		\vec{x} \\
		ct
	\end{pmatrix}
\end{equation}

Le trasformazioni di Lorentz possono essere rappresentate in forma matriciale
tramite la matrice $\operatorname{L}\mrb{\beta}$ che è una matrice a
\textbf{determinante unitario}\footnote{
	$\det L \mrb{\beta} = \gamma ^{2} - \beta ^{2} \gamma ^{2} = 1$
} e rappresenta dunque una \textit{rotazione} nello spazio-tempo.
\begin{equation}
	\qvec{X}\mprime = \operatorname{L}\mrb{\beta} \qvec{X}
\end{equation}
In rappresentazione matriciale:
\begin{equation}
	\begin{pmatrix}
		x \mprime \\
		y \mprime \\
		z \mprime \\
		ct \mprime
	\end{pmatrix} =
	\begin{bmatrix}
		\gamma        & 0 & 0 & -\beta \gamma \\
		0             & 1 & 0 & 0             \\
		0             & 0 & 1 & 0             \\
		-\beta \gamma & 0 & 0 & \gamma
	\end{bmatrix}
	\begin{pmatrix}
		x \\
		y \\
		z \\
		ct
	\end{pmatrix}
	=
	\begin{pmatrix}
		\gamma x - \beta \gamma ct \\
		y                          \\
		z                          \\
		-\beta \gamma x + \gamma ct
	\end{pmatrix}
\end{equation}

La trasformazione inversa:
\begin{equation}
	\qvec{X} = \operatorname{L}^{-1}\mrb{\beta} \qvec{X}\mprime
\end{equation}
dove $\operatorname{L}^{-1}\mrb{\beta} = \operatorname{L}\mrb{-\beta}$.
Quindi in forma esplicita:
\begin{equation}
	\begin{pmatrix}
		x
		\\
		y
		\\
		z
		\\
		ct
	\end{pmatrix} =
	\begin{bmatrix}
		\gamma       & 0 & 0 & \beta \gamma
		\\
		0            & 1 & 0 & 0
		\\
		0            & 0 & 1 & 0
		\\
		\beta \gamma & 0 & 0 & \gamma
	\end{bmatrix}
	\begin{pmatrix}
		x \mprime
		\\
		y \mprime
		\\
		z \mprime
		\\
		ct \mprime
	\end{pmatrix} =
	\begin{pmatrix}
		\gamma x \mprime + \beta \gamma c t \mprime
		\\
		y \mprime
		\\
		z \mprime
		\\
		\beta \gamma x \mprime + \gamma ct \mprime
	\end{pmatrix}
\end{equation}

\paragraph{Approssimazione non relativistica, limite classico}
L'approssimazione non relativistica si ottiene nel caso in cui la velocità
relativa $v$ sia molto minore della velocità della luce nel vuoto $c$, che è la
velocità di riferimento. In particolare quando $v \ll c$, quindi in termini di
$\beta$, quando $\beta \ll 1$. Sviluppando in serie al primo ordine in $\beta$:
\begin{equation}
	\gamma = 1 + \frac{\beta^2}{2} + \dots
\end{equation}
quindi sostituendo $\gamma$ otteniamo:
\begin{equation}
	x \mprime = \mrb{1 + \frac{\beta^2}{2} + \dots} x - \beta \mrb{1 +
		\frac{\beta^2}{2} + \dots} ct
	= x + \frac{x \beta^2}{2} - \beta ct - \frac{\beta^3 ct}{2} + \dots
	\simeq x - \beta ct
\end{equation}
e dalla definizione di $\beta$ possiamo scrivere:
\begin{equation}
	x \mprime \simeq x - vt
\end{equation}
per $v \ll c$. Quindi abbiamo riottenuto la trasformazione di Galileo, come
volevamo.

Analogamente:
\begin{equation}
	c t \mprime = \mrb{1 + \frac{\beta^2}{2} + \dots} ct - \mrb{1 +
		\frac{\beta^2}{2} + \dots} \beta x
	= ct + \frac{\beta^2 ct}{2} - \beta x - \frac{\beta^3 x}{2} + \dots
	\simeq ct - \beta x
\end{equation}
Ora, dato che $\frac{\beta}{c} = \frac{v}{c^2}$, considerata l'approssimazione
$v \ll c$, possiamo trascurare il termine nella somma e otteniamo:
\begin{equation}
	t \mprime \simeq t
\end{equation}
come da trasformata di Galileo.

\paragraph{Forma vettoriale generale delle trasformazioni di Lorentz}
Consideriamo ora il caso con $\vec{\beta}$ del sistema in moto relativo non
parallelo a uno degli assi (in precedenza abbiamo supposto parallela a $x$).
Scomponiamo $\vec{x}$ in direzione parallela e perpendicolare a $\vec{\beta}$.
\begin{equation}
	\begin{dcases}
		\vec{x} \mprime _{\parallel} = \vec{\beta}
		\frac{\dpr{\vec{\beta}}{\vec{x}}}{\beta^2}
		\\
		\vec{x} \mprime _{\perp} = \vec{x} \mprime - \vec{\beta}
		\frac{\dpr{\vec{\beta}}{\vec{x}}}{\beta^2}
	\end{dcases}
\end{equation}
con $\beta^2 = \abs{\vec{\beta}}^{2}$.
Quindi:
\begin{equation}
	\begin{dcases}
		\vec{x} _{\parallel} \mprime = \gamma \mrb{\vec{x} _{\parallel} + c
			\vec{\beta} t \mprime}
		\\
		\vec{x}_{\perp} \mprime = \vec{x}_{\perp}
	\end{dcases}
\end{equation}
con $ct = \gamma \mrb{c t \mprime + \dpr{\vec{\beta}}{\vec{x}}}$.
Allora abbiamo:
\begin{equation}
	\vec{x}
	= \vec{x} _{\parallel} + \vec{x} _{\perp}
	= \gamma \mrb{\vec{\beta} \frac{\dpr{\vec{\beta}}{\vec{x}}}{\beta^2} - c
		\vec{\beta} t \mprime} + \vec{x} \mprime - \vec{\beta}
	\frac{\dpr{\vec{\beta}}{\vec{x}\mprime}}{\beta^2}
	= \vec{x} \mprime + \vec{\beta} \mrb{\dpr{\vec{\beta}}{\vec{x} \mprime}
		\frac{\gamma - 1}{\beta^2} + \gamma c t \mprime}
\end{equation}
Utilizzando $\beta \mprime = \frac{\mrb{\gamma ^{2} - 1}}{\gamma^2}$ otteniamo
la \textit{forma vettoriale \textbf{generale} delle trasformate di Lorentz}:
\begin{equation}
	\begin{dcases}
		\vec{x} = \vec{x} \mprime + \vec{\beta} \gamma \mrb{\frac{\gamma}{\gamma +
				1} \dpr{\vec{\beta}}{\vec{x} \mprime} + c t \mprime}
		\\
		ct = \gamma \mrb{ct \mprime + \dpr{\vec{\beta}}{\vec{x} \mprime}}
	\end{dcases}
	\label{eq:lorentz_vettoriale}
\end{equation}

