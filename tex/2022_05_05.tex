%LTeX: language=it
\chapter{2022-05-05}
\section{Radioattività - Decadimenti Nucleari (reprise)}
\subsection{Decadimenti sequenziali}
Si parla di \textit{decadimenti sequenziali} quando un ``\textit{nulceo
	padre}'' può produrre un ``\textit{nucleo figlio}'' che è a sua volta
instabile. Indicando con $1$ il \textit{nucleo padre} e con $2$ il
\textit{nucleo figlio} e con $\omega_1$ e $\omega_2$ rispettivamente le
\textit{probabilità totali di decadimento per unità di tempo}:
\begin{equation}
	\begin{dcases}
		\mdv{N_1}{t} & = - \omega_1 N_1
		\\
		\mdv{N_2}{t} & = - \mdv{N_1}{t} - \omega_2 N_2 = \omega_1 N_1 - \omega_2 N_2
	\end{dcases}
\end{equation}
Il nucleo figlio che decade, a sua volta produrrà un nulceo figlio che
indichiamo con $3$:
\begin{align*}
	\mdv{N_3}{t} & = - \msb{\mdv{N_2}{t}}_{2\to3} - \omega_3 N_3 = \omega_2 N_2 -
	\omega_3 N_3
	\\
	             & \vdots
\end{align*}
e così via\footnote{
	Potremmo avere anche altri decadimenti sequenziali!
}.
Quindi:
\begin{equation}
	N_2 \mrb{t}
	= \frac{N_1 \mrb{t = 0} \omega_1}{\omega_2 - \omega_1}
	\mrb{e^{-\omega_1 t} - e^{-\omega_2 t }}
\end{equation}
e
\begin{equation}
	N_3 \mrb{t}
	= N_1 \mrb{t = 0} \omega_1 \omega_2
	\msb{
		\frac{e^{-\omega_1 t}}{\mrb{\omega_2 - \omega_1} \mrb{\omega_3 - \omega_1}}
		+
		\frac{e^{-\omega_2 t}}{\mrb{\omega_3 - \omega_2} \mrb{\omega_1 - \omega_2}}
		+
		\frac{e^{-\omega_1 t}}{\mrb{\omega_1 - \omega_3} \mrb{\omega_2 - \omega_3}}
	}
\end{equation}


La condizione di equilibrio del numero di particelle figlio $N_2$ è la
condizione per cui la derivata temporale del numero di particelle diventa
stazionaria:
\begin{equation}
	\mdv{N_2}{t} = -\omega_2 N_2 + \omega_1 N_1 = 0
	\mthen
	\omega_2 N_2 \mrb{t} = \omega_1 N_1 \mrb{t}
\end{equation}
da questo otteniamo la condizione:
\begin{equation}
	\textbf{Equilibrio secolare}
	\qquad
	\boxed{
		N_2 \mrb{t} = \frac{\omega_1}{\omega_2} N_1 \mrb{t}
	}
\end{equation}
questa condizione di equilibrio è valida nel caso in cui:
\begin{equation}
	t \gg t_2 = \frac{1}{\omega_2}
\end{equation}

A sua volta vale anche per le particelle prodotte (eventualmente) dai
decadimenti delle particelle figlio:
\begin{equation}
	\mdv{N_3}{t} = -\omega_3 N_3 + \omega_2 N_2 = 0
	\mthen
	N_3 \mrb{t} = \frac{\omega_2}{\omega_3} N_2 \mrb{t}
\end{equation}

\subsection{Condizioni energetiche per i vari decadimenti}
\subsubsection{Decadimenti $\alpha$}
\begin{equation}
	\mrb{Z,A} \rightarrow \mrb{Z-2, A-4} + \mrb{2,4}
\end{equation}

\begin{equation}
	\begin{dcases}
		\mrb{E_\text{tot}}_\text{ini} = \mrb{E_\text{tot}}_\text{fin}
		\\
		\mrb{\vec{p}_\text{tot}}_\text{ini} = \mrb{\vec{p}_\text{tot}}_\text{fin}
	\end{dcases}
\end{equation}
\textit{invarianza per traslazioni spazio-temporali}.
Oltre a questo:
\begin{equation}
	\mrb{\vec{J}_\text{tot}}_\text{ini} = \mrb{\vec{J}_\text{tot}}_\text{fin}
\end{equation}
\textit{invarianza per rotazione}.

\paragraph{Sistema di quiete della particella che decade}
Consideriamo il \textit{sistema di quiete delle particella che decade}. In tale
sistema di riferimento abbiamo:
\begin{equation}
	\begin{dcases}
		\mrb{E_\text{tot}}_\text{ini}       & = m \mrb{Z,A} c^2
		\\
		\mrb{\vec{p}_\text{tot}}_\text{ini} & = 0
	\end{dcases}
\end{equation}
L'energia totale e l'impulso finali delle particelle dovranno essere uguali
alle quantità iniziali:
\begin{align*}
	\mrb{E_\text{tot}}_\text{fin}       & = \mrb{E_\text{tot}}_\text{fin} = m \mrb{Z-2,
		A-4}c^2 + m \mrb{2,4} c^2 + T
	\\
	\mrb{\vec{p}_\text{tot}}_\text{fin} & = 0
\end{align*}
Quindi:
\begin{equation}
	m \mrb{Z,A} c^2 = m \mrb{Z-2, A-4} c^2 + m \mrb{2,4}c^2 + T
\end{equation}
dove $T \geq 0$ ha un valore minimo possibile che è $0$.
Affinché il \textit{decadimento $\alpha$} possa avvenire, si deve verificare la
condizione:
\begin{equation}
	m \mrb{Z,A} c^2 - \msb{m \mrb{Z-2, A-4} c^2 + m \mrb{2,4} c^2} \geq 0
\end{equation}
dove $m \mrb{Z,A} c^2$ è la \textit{massa iniziale} ed il termine in parentesi
quadre rappresenta la \textit{somma delle masse nello stato finale}.
Possiamo scrivere, dato il decadimento:
\begin{equation}
	\begin{dcases}
		m \mrb{Z,A} c^2 = z m_p c^2 + \mrb{A-Z} m_n c^2 - B \mrb{Z,A}
		\\
		\msb{\dots} = - \mrb{Z-2} m_p c^2 - \mrb{A - Z -
			2} m_n c^2 + B \mrb{Z - 2, A - 4} - 2 m_p c^2 - 2 m_n c^2 + B \mrb{2, 4}
	\end{dcases}
\end{equation}
quindi sostituendo nell'equazione precedente:
\begin{align*}
	\cancel{z m_p c^2} + \cancel{\mrb{A-Z} m_n c^2} & - B \mrb{Z,A} - \mrb{Z-2}
	m_p c^2 - \mrb{A - Z - 2} m_n c^2 +
	\\
	                                                & + B \mrb{Z - 2, A - 4} - 2 m_p c^2 - 2 m_n c^2
	+ B \mrb{2, 4} \geq 0 % TODO: finisci di cancellare i termini
\end{align*}
quindi:
\begin{equation}
	- B \mrb{Z,A} + B \mrb{Z-2, A-4} + B \mrb{2,4} \geq 0
\end{equation}
allora:
\begin{equation}
	\msb{B \mrb{Z-2, A-4} + B \mrb{2,4}} - B \mrb{Z,A} \geq 0
\end{equation}
dove:
\begin{align*}
	\msb{B \mrb{Z-2, A-4} + B \mrb{2,4}} & \textit{Somma delle energie di legame
		dello stato finale}
	B \mrb{Z,A}                          & \textit{Stato iniziale}
\end{align*}
Quindi otteniamo dei \textit{decadimenti $\alpha$} solo se è soddisfatta questa
condizione!

\begin{note}[]
	Questo è il motivo per cui non esistono in natura dei nuclei arbitrariamente
	grandi: ad un certo punto le dimensioni dei nuclei e il numero dei nucleoni
	diventa tale che è probabile il decadimento $\alpha$ e il nucleo è quindi
	instabile.
\end{note}

\subsubsection{Decadimento $\beta$}
Il decadimento $\beta$ è un processo tale che $A = \text{const}$ e $Z
\rightarrow Z \pm 1$:
\begin{equation}
	\begin{dcases}
		\mrb{Z,A} \rightarrow \mrb{Z+1,A} + e^- + \cc{\nu_e} & \textit{decadimento $\beta^-$}
		\\
		\mrb{Z,A} \rightarrow \mrb{Z-1,A} + e^+ + \nu_e      & \textit{decadimento $\beta^+$}
		\\
		e^- + \mrb{Z,A} \rightarrow \mrb{Z-1,A} + \nu_e      & \textit{electronic capture}
	\end{dcases}
\end{equation}

\begin{note}[Conservazione del numero leptonico]
  Le particelle portano numero leptonico $+1$ e le antiparticelle portano
  numero leptonico $-1$. Nei decadimenti deve valere la \textit{conservazione
  del numero leptonico}.
\end{note}

\paragraph{Q-valore}
\begin{equation}
  \begin{dcases}
    Q_{\beta^-} = \msb{M \mrb{Z, A} - M \mrb{Z + 1, A} - m_e} c^2
    \\
    Q_{\beta^+} = \msb{M \mrb{Z, A} - M \mrb{Z - 1, A} - m_e} c^2
    \\
    Q_{\textit{e.c.}} = \msb{M \mrb{Z, A} + m_e - M \mrb{Z - 1, A}} c^2
  \end{dcases}
\end{equation}
Nel sistema di riferimento del centro di massa, il \textit{Q-valore} fornisce
l'informazione dell'energia cinetica dello stato finale. Ad esempio, per il
decadimento $\beta^-$:
\begin{equation}
  M \mrb{Z, A} c^2
  = M \mrb{Z + 1, A} c^2 + m_e c^2 + T_N + T_e
\end{equation}
Nel sistema del centro di massa, l'impulso:
\begin{equation}
  0 = p_n + p_e
\end{equation}
Quindi avremo che, data la differenza di massa fra il nucleo $N$ e l'elettrone
$e$:
\begin{equation}
  T_e = \frac{p_e}{2 m_e} \gg T_n = \frac{p_N}{2 m_N}
\end{equation}
Quindi potremmo trascurare $T_N$:
\begin{equation}
  M \mrb{Z, A} c^2
  = M \mrb{Z + 1, A} c^2 + m_e c^2 + \cancel{T_N} + T_e
  \mthen
  Q _{\beta^-} = T_e
\end{equation}
Questo portò Pauli a ipotizzare la presenza dei \textit{neutrini}, particelle
neutre dalla massa estremamente piccola, che non conosciamo, ma che alla quale
è stato posto un limite superiore: $m_{\nu} < \qty{1}{\eV}$.

In termini di \textit{masse atomiche} $\mathcal{M}$:
\begin{equation}
  M \mrb{Z, A} = \mathcal{M} \mrb{Z, A} - m_e Z
\end{equation}
il \textit{Q-valore}:
\begin{equation}
  \begin{dcases}
    Q _{\beta^-} = \msb{\mathcal{M} \mrb{Z, A} - \mathcal{M} \mrb{Z - 1}}
    Q _{\beta^+} = \msb{}
    Q _{\textit{e.c.}} = \msb{}
  \end{dcases}
\end{equation}

\paragraph{Energia}
Per i decadimenti $\beta$ otteniamo le seguenti \textit{condizioni energetiche}:
\begin{equation}
  \begin{dcases}
    m \mrb{Z,A} c^2 - \msb{m \mrb{Z+1,A}c^2 + m_e c^2 + \cancel{m_{\nu}c^2}} \geq
    0            & \qquad\textit{decadimento $\beta^+$}
    \\
    m \mrb{Z,A} c^2 - \msb{m \mrb{Z-1,A}c^2 + m_e c^2 + \cancel{m_{\nu}c^2}} \geq
    0            & \qquad\textit{decadimento $\beta^-$}
    \\
    \msb{m_e c^2 + m \mrb{Z,A} c^2} - \msb{m \mrb{Z-1,A} c^2 + \cancel{m_{\nu}
    c^2}} \geq 0 & \qquad\textit{e.c.}
  \end{dcases}
\end{equation}

??
\begin{align}
	z m_p c^2 + \mrb{A-Z} m_n c^2 - B \mrb{Z,A} - \mrb{Z+1} m_p c^2
	          - \mrb{A - Z - 1} m_n c^2 + B \mrb{Z + 1, A} - m_e c^2
	\\\notag
	= \mrb{m_n c^2 - m_p c^2 - m_e c^2} - B \mrb{Z,A} + B \mrb{Z+1, A} \geq 0
\end{align}
quindi per i tre processi:

\begin{itemize}
	\item \textit{decadimento $\beta^-$}, $n \rightarrow p + e^- +
		      \cc{\nu}_{e}$:
	      \begin{equation}
		      B \mrb{Z+1, A} - B \mrb{Z,A} \geq m_p c^2 + m_e c^2 - m_n c^2 \simeq
		      -1.3 + \SI{0.5}{\MeV} \simeq \SI{-0.8}{\MeV}
	      \end{equation}

	\item \textit{decadimento $\beta^-$}, $p \rightarrow n + e^+ + \nu_e$:
	      \begin{equation}
		      B \mrb{Z-1, A} - B \mrb{Z,A} \geq m_p c^2 + m_e c^2 - m_p c^2 \simeq
		      1.3 + \SI{0.8}{\MeV} \simeq \SI{-0.8}{\MeV}
	      \end{equation}

	\item \textit{electronic capture}, $e^- + p \rightarrow n + \nu_e$:
	      \begin{equation}
		      B \mrb{Z-1, A} - B \mrb{Z,A} \geq m_n c^2 - m_p c^2 - m_e c^2 = 1.3 -
		      \SI{}{\MeV} % TODO: copia valore
	      \end{equation}
\end{itemize}

\section{Modello standard delle interazioni fondamentali}
Tutti i fenomeni fisici, secondo questo modello, sono riconducibili ad un certo
numero di \textit{\textbf{particelle elementari}}, che interagiscono fra di
loro grazie ad un certo numero di \textit{\textbf{interazioni fondamentali}}.
Secondo il \textit{\textbf{modello standard delle interazioni fondamentali}} le
interazioni fondamentali sono:
\begin{itemize}
	\item \textit{interazione gravitazionale};
	\item \textit{interazione elettromagnetica};
	\item \textit{interazione forte};
	\item \textit{interazione debole}.
\end{itemize}
Nei processi microscopici possiamo considerare trascurabile l'interazione
gravitazionale.

\paragraph{Interazioni deboli}
Le \textit{interazioni deboli} sono mediate da \dots % TODO: 

\paragraph{Interazioni forti}
Le \textit{interazioni forti} sono mediate da \textbf{\textit{gluoni}}\dots
% TODO: 

I \textbf{mediatori delle interazioni} sono particelle con spin $1$, mentre le
\textbf{particelle fondamentali} sono particelle con spin $\frac{1}{2}$.

\paragraph{Particelle fondamentali}
Le particelle fondamentali sono particelle a spin $\frac{1}{2}$ si distinguono
in:
\begin{itemize}
	\item \textbf{Leptoni} (\textit{interazioni elettromagnetiche} +
	      \textit{interazioni deboli}), divisi in $3$ famiglie:
	      \begin{equation}
		      \begin{pmatrix}
			      \nu_e \\ e^-
		      \end{pmatrix}
		      \qquad
		      \begin{pmatrix}
			      \nu_{\mu} \\
		      \end{pmatrix}
		      \qquad
		      \begin{pmatrix}
			      \nu_{\tau} \\ \theta^-
		      \end{pmatrix}
		      \qquad
		      + \text{antiparticelle}
	      \end{equation}
	      I \textit{leptoni} hanno carica intera: $Q = 0$, $Q = -1$;

	\item \textbf{Quarks} (\textit{interazioni elettromagnetiche} +
	      \textit{interazioni deboli} + \textit{interazioni forti})
	      \begin{equation}
		      \begin{pmatrix}
			      \mu \\ d
		      \end{pmatrix}
		      \qquad
		      \begin{pmatrix}
			      c \\ s
		      \end{pmatrix}
		      \qquad
		      \begin{pmatrix}
			      t \\ b
		      \end{pmatrix}
	      \end{equation}
	      I \textit{quarks} hanno carica frazionaria: $Q = \frac{2}{3}$, $Q = -
		      \frac{1}{3}$: per questo motivo \textit{in natura non si osservano qark
		      liberi}.
	      \begin{equation}
		      \begin{dcases}
			      \textbf{Mesoni} - q \cc{q}
			      \\
			      \textbf{Barioni} \mrb{\textbf{Anti-Barioni}} - qqq
			      \mrb{\cc{q}\cc{q}\cc{q}}
		      \end{dcases}
	      \end{equation}
	      dove abbiamo indicato con $q$ i \textit{quarks} e con $\cc{q}$ gli
	      \textit{anti-quarks}. Ad esempio, i \textit{barioni più leggeri} sono
	      quelli che osserviamo in natura:
	      \begin{align*}
		      p = \mrb{\mu \mu d} \qquad n = \mrb{\mu dd}
		      \\
		      \pi^+ = \mrb{\mu \cc{d}} \qquad \pi^- = \mrb{d \cc{\mu}}
	      \end{align*}
\end{itemize}

\subsection{Leggi di conservazione}
\begin{align*}
	Q                                                                 & = \textbf{carica}
	\\
	B = \frac{1}{3} \mcb{N \mrb{q} - N \mrb{\cc{q}}}                  & = \textbf{numero barionico}
	\\
	L = \mcb{N \mrb{l} - N \mrb{\cc{l}}}                              & = \textbf{numero leptonico}
	\\
	L _{\alpha} = \mcb{N \mrb{l _{\alpha}} - N \mrb{\cc{l}_{\alpha}}} & =
	\textbf{numero leptonico di famiglia}
	% TODO: 
\end{align*}

