%LTeX: language=it
\chapter{2020-03-17}
\section{Quadrivettori e Invarianti Relativistici}
Oggetto quadrimensionale che trasforma secondo le trasformazioni di
Lorentz\footnote{
  Vedi definizione corretta di quadrivettore - Corso di Introduzione alla
  fisica moderna
}. Abbiamo già definito il quadrivettore posizione dello spazio-tempo:
\begin{equation}
  \qvec{X} = \mcb{ct,\vec{x}} = \mcb{ct,x,y,z}
\end{equation}

\paragraph{Proprietà dei quadrivettori}
Elenchiamo alcune delle proprietà dei quadrivettori:
\begin{itemize}
  \item \textbf{Prodotto per uno scalare}: sia $\qvec{P}$ un quadrivettore e
    $a$ uno scalare, allora $a \qvec{P}$ è ancora quadrivettore;
  \item \textbf{Somma di quadrivettori}: siano $\qvec{P}, \qvec{Q}$ due
    quadrivettori, allora $\qvec{P} + \qvec{Q}$ è ancora quadrivettore;
  \item le trasformazioni di Lorentz lasciano invariata la norma dei
    quadrivettori.
\end{itemize}

\paragraph{Norma di un quadrivettore} Definiamo:
\begin{itemize}
  \item \textbf{quadrivettore controvariante}
    \begin{equation}
      \qvec{V}^{\mu} = \mcb{V_0, V_x, V_y, V_z}
    \end{equation}
  \item \textbf{quadrivettore covariante}
    \begin{equation}
      \qvec{V}_{\mu} = \mcb{V_0, -V_x, -V_y, -V_z}
    \end{equation}
  \item \textbf{tensore metrico} (\textit{metrica di Minkowski})
    \begin{equation}
      g_{\mu\nu} =
      \begin{pmatrix}
        1 & 0  & 0  & 0 \\
        0 & -1 & 0  & 0 \\
        0 & 0  & -1 & 0 \\
        0 & 0  & 0  & -1
      \end{pmatrix}
    \end{equation}
\end{itemize}
Quindi definiamo la norma di un quadrivettore come segue\footnote{
  Stiamo facendo un prodotto scalare nella metrica di Minkowski, quindi nel
  penultimo passaggio c'è un abbassamento dell'indice tramite la metrica
}:
\begin{equation}
  \qvec{V}^2 
  = g \mrb{\qvec{V}, \qvec{V}}
  = \sum_{\mu=0}^3\sum_{\nu=0}^3 g_{\mu\nu} \qvec{V}^\mu \qvec{V}^\nu
  = \sum_{\mu=0}^3\sum_{\nu=0}^3 \qvec{V}^\mu g_{\mu\nu} \qvec{V}^\nu
  = \sum_{\mu=0}^3 \qvec{V}_\mu \qvec{V}^\nu
  = V_0^2 - V_x^2 - V_y^2 - V_z^2
\end{equation}
Utilizzando la convenzione di Einstein della somma sugli indici ripetuti:
\begin{align}
  \textbf{Norma di un quadrivettore}
  \qquad
  \boxed{
    \qvec{V}^2
    = g \mrb{\qvec{V}, \qvec{V}}
    = g_{\mu\nu}\qvec{V}^\mu \qvec{V}^\nu
    = \qvec{V}^\mu g_{\mu\nu} \qvec{V}^\nu
    = \qvec{V}_\mu \qvec{V}^\mu 
  }
\end{align}
Una trasformazione di Lorentz lascia invariata la norma di un quadrivettore.

\paragraph{Invarianza della norma di un quadrivettore sotto trasformazioni di
Lorentz}
Vogliamo dimostrare ora che le trasformazioni di Lorentz lasciano invariata
la norma di un quadrivettore e che quindi che valga l'uguaglianza:
\begin{equation}
  \qvec{X}^{\prime\, 2} = \qvec{X}^2
\end{equation}
Dunque definendo $\qvec{X^\prime} = \mcb{ct^\prime, x^\prime, y^\prime,
z^\prime}$ e applicando\footnote{
  Consideriamo un caso specifico per semplicità, ma il ragionamento ha valenza
  generale, potendo sempre scegliere l'asse $x$ del SR parallelo alla direzione
  del boost di Lorentz
} un \textit{boost di Lorentz lungo la direzione $x$} all'espressione
esplicita della sua norma, poiché $\mrb{\gamma^2}^{-1} = \mrb{1-\beta^2}$,
abbiamo:
\begin{align}
    \qvec{X}^{\prime\, 2} 
    &= ct^{\prime 2} - x^{\prime 2} - y^2 - z^2
    \\\notag
    &= \mrb{-\beta\gamma x + \gamma ct}^2 - \mrb{\gamma x - \beta\gamma ct}^2 -
    y^2 - z^2
    \\\notag
    &= \beta^2\gamma^2x^2 + \gamma^2c^2t^2 - \cancel{2\beta\gamma^2xct} 
    - \gamma^2x^2 - \beta^2\gamma^2c^2t^2 + \cancel{2\beta\gamma^2xct} - y^2
    - z^2
    \\\notag
    &= c^2t^2\gamma^2\mrb{1-\beta^2} - x^2\gamma^2\mrb{1-\beta^2} - y^2 - z^2 
    \\\notag
    &= c^2t^2 - x^2 - y^2 - z^2 
    \\\notag
    &= \qvec{X}^2
\end{align}
e questo dimostra l'invarianza della norma di un quadrivettore sotto
trasformazioni di Lorentz.

\begin{note}[]
  Abbiamo sfruttato:
  \begin{equation}
    \gamma^2 \mrb{1 - \beta^2} 
    = \mrb{\frac{1}{\sqrt{1 - \frac{v^2}{c^2}}}}^{2} \msb{{1 -
    \mrb{\frac{v}{c}}^2}}
    = \frac{1 - \frac{v^2}{c^2}}{1 - \frac{v^2}{c^2}}
    = 1
  \end{equation}
\end{note}

\begin{exercise}[Da svolgere]
  Dimostrare che il prodotto scalare tra 2 quadrivettori non dipende dal sistema
  di riferimento.
\end{exercise}
