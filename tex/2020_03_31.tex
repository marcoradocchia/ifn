%LTeX: language=it
\chapter{Lezione 31/03/2020}

\section{Metodo delle ellissi - Cambiamento dello spettro di impulsi dal CM al
  Lab}
Trattiamo il problema di una reazione in cui una \textit{particella incide su
	un bersaglio fisso} nel sistema del laboratorio (\textbf{Lab}) note le masse
delle particelle e l'energia della particella incidente.

\begin{note}[]
	D'ora in avanti indicheremo con $p_x, p_y, p_z$ e $p_x^\ast, p_y^\ast,
		p_z^\ast$ rispettivamente le componenti dei vettori $\vec{p}$ e
	$\vec{p}^{\,\ast}$.
\end{note}

\paragraph{Sistema del Centro di Massa}
Nel sistema del centro di massa (\textbf{CM}) sappiamo tutto quello che accade,
perché le particelle finali saranno emesse con impulsi uguali e contrari e le
quantità associate dipenderanno solo dalle masse finali e dalla massa
invariante (ovvero l'energia del centro di massa).
\begin{equation}
	\abs{\vec{p}^{\,\ast}}^{2} = \frac{\msb{M^2 - \mrb{m_1 + m_2}^2} \msb{M^2 -
			\mrb{m_1 - m_2}^2}}{4 M^2}
\end{equation}
\begin{equation}
	\varepsilon_1^\ast = \frac{M^2 + m_1^2 - m_2^2}{2 M};
	\qquad
	\varepsilon_2^\ast = \frac{M^2 + m_2^2 - m_1^2}{2 M}
\end{equation}
Nello spazio degli impulsi possiamo rappresentare i possibili stati della
particella come luogo dei vettori che si muovono su una superficie sferica,
poiché, \textit{in assenza di effetti dinamici}, non c'è motivo di pensare che
ci sia una direzione privilegiata, per cui abbiamo quella che si dice
\textbf{distribuzione angolare isotropa}\footnote{
	Questo per via delle considerazioni puramente cinematiche che stiamo
	effettuando
}.
Questo vuol dire che l'equazione di tale \textit{superficie sferica} sarà data
da:
\begin{align}
	\boxed{\frac{p_x^{\ast 2}}{p^{\ast2}} + \frac{p_y^{\ast2}}{p^{\ast2}} +
	\frac{p_z^{\ast2}}{p^{\ast2}} = 1}
\end{align}

\paragraph{Sistema del Laboratorio}
Nel sistema del centro di massa, come detto, tutto è determinato, ma delle
eventuali misure sperimentali saranno effettuate nel \textit{sistema del
	laboratorio}!

Nel passaggio dal sistema del centro di massa a quello del laboratorio,
supponiamo che la trasformazione di Lorentz avvenga lungo la direzione
$\hat{x}$\footnote{
	È sempre possibile orientare il sistema di riferimento in modo tale che il
	boost di Lorentz avvenga lungo tale direzione
}. Definiamo $\vec{\beta}$ e $\gamma$ rispettivamente la velocità e il
fattore di Lorentz di tale trasformazione. L'\textit{impulso} trasforma nel
seguente modo:
\begin{equation}
	\begin{dcases}
		p_x = \gamma\mrb{p_x^\ast + \beta E^\ast}
		\\
		p_y = p_y^\ast
		\\
		p_z = p_z^\ast
	\end{dcases}
\end{equation}
dove $E^\ast$ è l'energia della particella in oggetto, nel sistema del centro
di massa.

\begin{figure}[ht!]
	\centering
	\begin{tikzpicture}
		\tikzset{
			hatch distance/.store in=\hatchdistance,
			hatch distance=10pt,
			hatch thickness/.store in=\hatchthickness,
			hatch thickness=2pt
		}

		\makeatletter
		\pgfdeclarepatternformonly[\hatchdistance,\hatchthickness]{flexible hatch}
		{\pgfqpoint{0pt}{0pt}}
		{\pgfqpoint{\hatchdistance}{\hatchdistance}}
		{\pgfpoint{\hatchdistance-1pt}{\hatchdistance-1pt}}%
		{
			\pgfsetcolor{\tikz@pattern@color}
			\pgfsetlinewidth{\hatchthickness}
			\pgfpathmoveto{\pgfqpoint{0pt}{0pt}}
			\pgfpathlineto{\pgfqpoint{\hatchdistance}{\hatchdistance}}
			\pgfusepath{stroke}
		}
		\begin{axis}[
				title=\textbf{CM},
				xlabel=$p_x^\ast$,
				ylabel=$p_y^\ast$,
				axis equal,
				xmin=-1.2,
				xmax=1.2,
				ymin=-1.4,
				ymax=1.4,
			]

			\draw[
				thick
			] (axis cs:0,0) circle [blue, radius=1];

			\draw[
			] (axis cs:0,0) -- (axis cs:0.707,-0.707)
			node[below right] {$p^\ast$};

			\draw[
			] (axis cs:0,0) -- (axis cs:0.707,0.707);

			\draw[
			] (axis cs:0,0) -- (axis cs:-0.707,0.707);

			\draw[
				Stealth-Stealth,
				thick
			] (axis cs:-0.707,0.707) -- (axis cs:0.707,0.707)
			node[midway, below] {$\Delta p _{x} ^{\ast} = 2a$};

			\draw[
				dashed
			] (axis cs:0.707,0) node[below] {$a$} -- (axis cs:0.707,0.707);

			\draw[
				dashed
			] (axis cs:-0.707,0) node[below] {$-a$} -- (axis cs:-0.707,0.707);
		\end{axis}
	\end{tikzpicture}
	\caption{}
\end{figure}

Consideriamo ora un valore generico $p^\ast_x = a$ tale che $0 < a < p^\ast$;
data la condizione di isotropia, allora possiamo dire che esiste anche un
valore uguale e opposto $p^\ast_x = -a$.
La differenza fra questi due valori sarà $\Delta p_x^\ast = 2a$. Quindi nella
trasformazione per passare al sistema del laboratorio avremo un \textit{boost
	di Lorentz} nella direzione $\hat{x}$, per cui: \begin{equation}
	\begin{dcases}
		\Delta p_x = \gamma \Delta p_x^\ast = \gamma 2a
		\\
		\Delta p_y = \Delta p_y^\ast
		\\
		\Delta p_z = \Delta p_z^\ast
	\end{dcases}
\end{equation}
quindi le altre direzioni, ovviamente, non subiranno boost di Lorentz.

A seguito di tale boost lungo una singola direzione avremo che quella che prima
era una superficie sferica si trasformi in un \textbf{ellissoide} nello spazio
tridimensionale, dove l'asse $x$ è dilatato dal fattore di Lorentz. Tale
ellissoide avrà semiassi:
\begin{align}
	\begin{dcases}
		a_x = \gamma p^\ast
		\\
		a_y = p^\ast
		\\
		a_z = p^\ast
	\end{dcases}
\end{align}

\begin{figure}[ht!]
	\centering
	\begin{tikzpicture}
		\begin{axis}[
				title=\textbf{Lab},
				xlabel=$p_x$,
				ylabel=$p_y$,
				axis equal,
				xmin=-0.2,
				xmax=2.5,
				ymin=-1.2,
				ymax=1.2,
			]

			\draw[
				thick
			] (axis cs:1.2,0) ellipse [x radius=1, y radius=0.5];

			\fill[
			] (axis cs:1.2,0) node[below] {$p_{x, \text{cen}}$} circle [radius=0.02];

			\node[below right] at (axis cs:0.2,0) {$p_{x, \text{min}}$};

			\node[below left] at (axis cs:2.2,0) {$p_{x, \text{max}}$};

			\draw[
				Stealth-Stealth,
				thick
			] (axis cs:0.33,0.25) -- (axis cs:2.07,0.25)
			node[midway, below] {$\Delta p _{x} = \gamma 2a$};
		\end{axis}
	\end{tikzpicture}
	\caption{}
\end{figure}

Nello spazio degli impulsi del \textbf{Lab}, il centro dell'ellissoide
coincide al caso $p_x ^{\ast} = 0$, quindi abbiamo:
\begin{equation}
	p_{x}^{\text{cen}} = \beta\gamma E^\ast;
	\qquad
	p_{x}^{\text{min}} = \gamma\mrb{-p^\ast + \beta E^\ast};
	\qquad
	p_{x}^{\text{max}} = \gamma\mrb{p^\ast + \beta E^\ast}
\end{equation}

L'equazione di tale ellissoide\footnote{
	La dipendenza da $E^\ast$ fa sì che l'ellissoide sia diverso per le
	particelle ottenute come prodotto di una certa reazione
}:
\begin{align}
	\textbf{Lab}
	\qquad
	\boxed{
		\frac{\mrb{p_x - \beta \gamma E^\ast}^2}{\gamma^2 p^{\ast\, 2}}
		+ \frac{p_y^2}{p^{\ast2}}
		+ \frac{p_z^2}{p^{\ast2}}
		= 1
	}
\end{align}
La dipendenza dell'ellissoide è dai parametri della trasformazione di Lorentz e
dai parametri della particella particolare presa in considerazione.

Osserviamo che l'ellissoide tocca il piano $p_x = 0$ se:
\begin{equation}
	p_{x}^{\text{min}} = \gamma\mrb{-p^\ast + \beta E^\ast} = 0
\end{equation}
\begin{equation}
	\gamma p^\ast\mrb{- 1 + \frac{\beta E^\ast}{p^\ast}} = 0
	\mthen
	\gamma p^\ast\mrb{\frac{\beta}{\beta^\ast} - 1} = 0
	\msse
	\beta = \beta^\ast
\end{equation}
ovvero se la velocità della trasformazione $\beta$ è uguale alla velocità della
particella nel centro di massa $\beta^\ast = \frac{p^\ast}{E^\ast}$.
Della relazione fra $\beta$ e $\beta^\ast$ possiamo distinguere i tre seguenti
casi:
\begin{itemize}
	\item $\beta < \beta^\ast$: l'ellissoide taglia il piano $p_x$, per cui sono
	      ammessi impulsi tali che $p_x < 0$;
	\item $\beta = \beta^\ast$: l'ellissoide è tangente al piano $p_x$, per cui
	      sono ammessi solo impulsi tali che $p_x \geq 0$;
	\item $\beta > \beta^\ast$: l'ellissoide è tutto a destra del piano $p_x$ per
	      cui sono ammessi solo impulsi tali che $p_x > 0$.
\end{itemize}
