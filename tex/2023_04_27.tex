%LTeX: language=it
\chapter{2023-04-27}
\section{Serie (catene) radiattive}
Furono individuate $3$ \textit{serie} (catene) \textit{radiattive}:

Le catene radiattive sono originate da tre tipi di decadimento:
\begin{itemize}
	\item $\alpha$ \rightarrow \textit{particelle cariche} $\ch{^{4}He^{++}}$;
	\item $\beta$ \rightarrow \textit{particelle negative};
	\item $\gamma$ \rightarrow \textit{particelle neutre} (fotoni energetici).
\end{itemize}
I primi e maggiori sviluppi in fisica nucleare sono stati fatti utilizzando
particelle $\alpha$.

\section{Rutherford scattering formula}
Utilizzeremo particelle con carica $ze$.

\subsection{Il modello}
\begin{itemize}
	\item L'atomo è costituito da un nucleo con carica positiva $Ze$ che porta
	      praticamente tutta la massa dell'atomo;
	\item l'atomo è elettricamente neutro e contiene $Z$ elettroni che orbitano
	      intorno al nucleo. Questi elettroni non possono dar luogo a deflessioni
	      grandi. Sono trascurabili rispetto alla particella $\alpha$, considerata
	      come proiettile;
	\item il nucleo bersaglio è molto più massivo della particella incidente;
	      questo permetterà in prima approssimazione di considerare il rinculo del
	      target nullo, $m \ll M$;
	\item si utilizza solo meccanica classica per descrivere l'urto, quindi
	      assumeremo la velocità del proiettile non relativistica, $v \ll c$;
	\item il nucleo bersaglio e la particella incidente hanno distribuzioni di
	      carica puntiforme e interagiscono attraverso un potenziale coulombiano
	      (interazione puramente elettromagnetica):
	      \begin{equation}
		      V \mrb{z} = \frac{Z z e^2}{4 \pi \eps_0 r}
	      \end{equation}
	\item si considera l'interazione elettromagnetica come unico tipo
	      d'interazione;
	\item il bersaglio e il proiettile non subiscono alcuna eccitazione durante
	      l'urto (scattering elastico).
\end{itemize}


La forza d'interazione del modello è la forza coulombiana:
\begin{equation}
  \vec{F} = \frac{z Z e^2}{4 \pi \eps_0} \frac{\hat{r}}{r^2}
\end{equation}
L'impulso trasferito, se gli impulsi iniziale e finale della particella prima e
dopo l'urto sono $\vec{p}_{i}$ e $\vec{p}_{f}$ rispettivamente:
\begin{equation}
  \Delta \vec{p} = \abs{\vec{p}_{f} - \vec{p}_{i}} = 2 p \sin \frac{\theta}{2}
\end{equation}
allora:
\begin{equation}
  \Delta p = \int_{-\infty}^{\infty} F_T \md[]{t}
  = \int_{-\infty}^{\infty} \frac{z Z e^2}{4 \pi \eps_0} \frac{cos \beta}{r^2} \md[]{t}
\end{equation}
dove, con $\beta$ l'angolo in figura \ref{fig:rutherford_scattering}:
\begin{equation}
  \beta \mrb{t = - \infty} = - \mrb{\frac{\pi}{2} - \frac{\theta}{2}};
  \qquad
  \beta \mrb{t = \infty} = \frac{\pi}{2} - \frac{\theta}{2}
\end{equation}

\subsection{Raggio di collisione}
\begin{equation}
  \tan \frac{\theta}{2}
  = \frac{z Z e^2}{4 \pi \eps_0 b} \frac{m}{p^2}
  = \frac{R}{b}
\end{equation}
con $R = \frac{z Z e^2 m}{4 \pi \eps_0 p^2}$.
In particolare, per $R = b$ abbiamo:
\begin{equation}
  \tan \frac{\theta}{2} = 1
  \mthen
  \frac{\theta}{2} = \qty{45}{\degree}
  \mthen \theta = \qty{90}{\degree}
\end{equation}

Per il caso di impatto frontale, utilizziamo il \textit{teorema delle forze
vive}:
\begin{equation}
  T + V = \text{const}
\end{equation}
quindi:
\begin{equation}
  \frac{1}{2} m v_0^2 + 0
  = \text{const}
  = 0 + \frac{z Z e^2}{4 \pi \eps_0 r_{\text{min}}}
\end{equation}
allora abbiamo che:
\begin{equation}
  r_{\text{min}} = 2 R
\end{equation}
Questo vuol dire che L'ordine di grandezza di $R$ rappresenta l'ordine di
grandezza della minima distanza della particella $\alpha$ dal nucleo
scatteratore.
