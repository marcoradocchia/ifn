%LTeX: language=it
\chapter{2022-05-03}
\section{Sezioni d'urto per nuclei non puntiformi}
\subsection{Fattori di forma}
Abbiamo visto che la \textit{sezione d'urto differenziale} può essere scritta
come:
\[
  \mdv{\sigma}{\Omega} = \mrb{\mdv{\sigma}{\Omega}}_\text{puntiforme} \abs{F
  \mrb{q^2}}^{2}
\]
dove $q$ è il modulo dell'\textit{impulso trasferito} \footnote{
  Ricordiamo che in approssimazione di nucleo fisso, quindi \textit{nucleo non
  rinculante}, $\abs{\vec{q}\,} = 2p \sin \frac{\theta}{2}$ e
  $\abs{\vec{p}_{i}} = \abs{\vec{p}_{f}} = p$
} $\vec{q} = \vec{p}_{i} - \vec{p}_{f}$ e
$F$ il \textit{\textbf{fattore di forma}}:
\[
  F \mrb{\vec{q}\,} = \int \mathrm{d}^{3} r \, \rho_N \mrb{\vec{r}\,} \exp
  \mcb{i \frac{\dpr{\vec{q}}{\vec{r}}}{\hbar}}
\]

Ora, abbiamo che il \textit{raggio del nucleo}:
\[
  R _{N} \simeq \mrb{\SI{1.2}{\femto\m}} A ^{\frac{1}{3}}
\]
quindi:
\[
  \frac{q R_N}{\hbar} \gtrsim 1
\]
allora:
\[
  qc \gtrsim \frac{\hbar c}{R_N} = \frac{\SI{200}{\MeV \femto\m}}{R_N}
\]
e di conseguenza:
\[
  q \gtrsim \frac{\SI{200}{\MeV \per c}}{R_N \msb{\si{\femto\m}}}
\]

Quindi se vogliamo sondare oggetti della dimensione di un \textit{fermi},
dobbiamo utilizzare sonde di impulso superiore a:
\[
  p \simeq q \gtrsim \frac{\SI{200}{\MeV \per c}}{R_N \msb{\si{\femto\m}}}
\]

Se consideriamo uno scattering di elettroni sui nuclei (se consideriamo invece
uno scattering di \textit{particelle alfa} su nuclei dovremmo tener conto anche
delle interazioni forti), possiamo cominciare a vedere la dimensione finita del
nucelo, per valori di impulso che fanno sì che gli \textit{elettroni}
siano \textit{\textbf{ultra-realtivistici}}:
\[
  pc \simeq \SI{100}{\MeV} \gg m_e c^2 \simeq \SI{0.5}{\MeV}
\]
L'ipotesi che il nucleo non rinculi è ancora adeguata, perché l'energia
cinetica della collisione, che è dell'ordine dei $\si{\MeV}$, è inferiore
rispetto all'energia a riposo della massa del centro scatteratore, che è
dell'ordine dei $\si{\GeV}$.

Questo vuol dire che la formula di Rutherford deve essere modificata per tener
conto di effetti relativistici: \textit{Scattering Rutherford} $\rightarrow$
\textit{Scattering Mott}.

\subsection{Scattering Mott}
Ricordiamo la sezione d'urto differenziale dello \textit{scattering
Rutherford}:
\[
  \mrb{\mdv{\sigma}{\Omega}}_\text{Rutherford} = \frac{Z^2 \alpha^2 \hbar^2
  c^2}{4 p^2 v^2} \frac{1}{\sin^{4} \frac{\theta}{2}}
\]
La sezione d'urto differenziale dello \textit{\textbf{scattering Mott}} è
ricavata a partire da quella dello \textit{scattering Rutherford}, ma è adatta
per il caso di \textit{\textbf{effetti ultra-relativistici}}:
\begin{align*}
  \textbf{Sezione d'urto diff. - Scattering Mott}\qquad
  \boxed{
    \mrb{\mdv{\sigma}{\Omega}}_\text{Mott} = \frac{Z^2 \alpha^2 \hbar^2 c^2}{4
    p^2 v^2} \frac{1}{\sin^{4} \frac{\theta}{2}} \mcb{1 - \frac{V^2}{c^2}
    \sin^2 \frac{\theta}{2}}
  }
\end{align*}

\section{Proprietà dei Nuclei}
\paragraph{Numero Atomico}
In un atomo neutro abbiamo che, indicando con $Z$ il \textbf{numero atomico},
$Ze$ corrisponde alla \textit{carica del nucleo}. In particolare $Z$ indica il
numero di protoni contenuti nel nucleo atomico e nel caso di atomi neutri
questo numero corrisponde esattamente al numero di elettroni. Il numero atomico
determina il comportamento chimico della sostanza.

\paragraph{Isotopi}
Studiando le masse atomiche si è scoperto che un determinato elemento chimico,
quindi $Z$ fissato, può presentarsi con masse atomiche differenti,
quindi si parla di \textbf{isotopi}. Tali valori riscontrati sono sempre vicini
($\sim 0.1\%$) ad un multiplo intero di $m_H$.
Tale massa è stata associata al diverso numero di neutroni nei nuclei.

Nella tavola periodica la massa atomica è indicata come la media delle masse
atomiche dei vari isotopi.

\paragraph{Numero di Massa Atomica}
Il \textbf{numero di massa atomica}, indicato con $A$, è definito come l'intero
più vicino al rapporto $\nicefrac{m_A}{m_H}$, dove $m_A$ rappresenta la
\textit{massa dell'atomo} considerato e $m_H$ la \textit{massa dell'idrogeno}
(che quindi funge da riferimento).

In genere si utilizza la seguente notazione:
\begin{equation}
  \ch{_Z^AX}
\end{equation}
dove $X$ rappresenta il \textit{simbolo chimico} e come detto $Z$ il
\textit{numero atomico} e $A$ il \textit{numero di massa atomica}. Il numero
atomico e il numero di massa atomica determinano la struttura del nucleo:
\begin{itemize}
  \item $Z$: \textit{numero di protoni};
  \item $A$: \textit{numero di nucleoni}\footnote{
      Termine utilizzato per indicare indifferentemente protoni e neutroni,
      costituenti il nucleo
    }
  \item $N = A - Z$: \textit{numero di neutroni}
\end{itemize}
\begin{note}[]
  Notiamo che:
  \begin{align*}
    m_p c^2 \simeq \SI{1}{\GeV}
    \\
    m_n c^2 \simeq \SI{1}{\GeV}
    \\
    m_n c^2 - m_p c^2 \simeq \SI{1.29}{\MeV}
  \end{align*}
  in particolare:
  \[
    \frac{m_n - m_p}{m_N} \simeq 10^{-3}
  \]
  dove $m_N$ è la \textit{massa del nucleo}.
  In natura, il fatto che il neutrone sia più massivo del protone, permette il
  processo di decadimento, nel vuoto, del neutrone:
  \[
    n \longrightarrow p + e^- + \cc{\nu}_{e^-}
  \]
  dove $\cc{\nu}_{e}$ è l'\textit{antineutrino elettronico}. Il tempo
  caratteristico del decadimento del neutrone è $\tau \simeq \SI{880}{s}$.

  Il processo inverso di decadimento del protone:
  \[
    \bcancel{p \longrightarrow n + e^+ + \nu_{e^+}}
  \]
  non può avvenire spontaneamente per via della differenza di massa fra proton
  e neutrone.
\end{note}

\subsection{Difetto di massa ed energia di legame}
Consideriamo un nucleo $\mrb{Z, A}$:
\[
  m \mrb{Z, A} c^2 \neq Z m_p c^2 + \mrb{A + Z} m_n c^2
\]

Una particella libera (puntiforme o composta) ha una \textit{relazione
impulso-energia}:
\[
  E \mrb{\vec{p}\,} = \sqrt{c^2 \abs{\vec{p}\,}^{2} + m^2 c^4}
\]
La massa di una particella equivale all'energia a riposo della particella nel
sistema di riferimento in cui la particella ha impulso nullo:
\[
  mc^2 = E \mrb{\vec{p} = 0}
\]
Questo è ovvio per sistemi puntiformi, ma non è ovvio per sistemi composti.
\begin{example}[Nucleo di \ch{^{3}He}]
  La massa dell'\textit{elio} \ch{^{3}He} sarà:
  \[
    m \mrb{\ch{^{3}He}} c^2 = E _\text{tot} \mrb{p _\text{tot} = 0}
    = m_p c^2 + m_p c^2 + m_n c^2 + T + V \neq m_p c^2 + m_p c^2 + m_n c^2
  \]
  dove $T$ \textit{energia cinetica} e $V$ rappresenta l'\textit{energia di
  interazione}.
  Tipicamente $T + V < 0 \sim \si{\MeV}$ (la somma dei termini è negativa
  perché stiamo parlando di uno stato legato) e $m_p c^2 + m_p c^2 + m_n c^2
  \sim \si{\GeV}$.

  L'energia cinetica del sistema si può scrivere come:
  \[
    T = \frac{p_p^2}{2 m_p} + \frac{p_p^2}{2 m_p} + \frac{p_n^2}{2 m_n}
  \]
  Allora:
  \[
    \Delta p_x\, \Delta x \gtrsim \hbar
  \]
  dove:
  \[
    \Delta p_x = \mcb{\braket{p_x^2} - \braket{p_x}^{2}}^{\nicefrac{1}{2}} =
    \braket{p_x^2}^{\nicefrac{1}{2}}
  \]
  dato che $\braket{p_x}^{2} = 0$.
  Considerando il fatto che le dimensioni fisiche del sistema ($\Delta x$) sono
  quelle delle dimensioni del nucleo, quindi dell'ordine del \textit{fermi}:
  \[
    \braket{p_x^2} \gtrsim \frac{\hbar^2}{R_N}
    \mthen
    \frac{\braket{p_x^2}}{2m} \gtrsim \frac{\hbar^2 c^2}{2m c^2 R_N^2} \gtrsim
    \frac{\SI{20}{\MeV}}{(R_N \msb{\si{\femto m}})^{2}}
  \]
\end{example}

\paragraph{Energia di legame del nucleo}
Tipicamente ci aspettiamo \textbf{\textit{difetti di massa} dell'ordine del}
$\si{\MeV}$ perché sono legati ai contributi apportati da $T + V < 0 \simeq
\si{\MeV}$. Potremo scrivere che:
\[
  m \mrb{Z, A} c^2 = \msb{Z m_p c^2 + \mrb{A - Z} m_n c^2} - B \mrb{Z, A}
\]
dove $B \mrb{Z, A}$ rappresenta l'\textbf{energia di legame del nucleo}.

\section{Radioattività - Decadimenti Nucleari}
Fenomeno che riguarda esclusivamente i nuclei atomici.
Storicamente i \textbf{decadimenti nucleari} sono raggruppati in 3 classi
principali, che riguardano tutti fenomeni di scala energetica del $\si{\MeV}$
(poiché stiamo considerando particelle di $1\si{\MeV}$ confinate in dimensioni
spaziali della scala di $1\si{\femto\m}$).
\begin{table}[h!]
  \centering
  \caption{Decadimenti di particelle}
  \begin{tabular}{llll}
    \toprule
    \textsc{Dec.} & \textsc{Tipo di Interazione} & \textsc{Particella
    prodotta} & \textsc{Reazione}
    \\
    \midrule
    $\alpha$ &
    \makecell[l]{Interazioni forti\\+ interazioni EM} &
    $\alpha$: fotoni &
    $\mrb{Z, A} \rightarrow \mrb{Z-2, A-4} + \mrb{2,4}$
    \\
    $\beta^+$ &
    Interazioni deboli &
    $\beta$: nuclei di $\ch{He^4}$ &
    $\mrb{Z, A} \rightarrow \mrb{Z-1, A} + e^+ + \nu_e$
    \\
    $\beta^-$ &
    Interazioni deboli &
    $\beta$: nuclei di $\ch{He^4}$ &
    $\mrb{Z, A} \rightarrow \mrb{Z+1, A} + e^- + \cc{\nu}_{e}$
    \\
    $\gamma$ &
    Interazioni deboli &
    $\gamma$: elettrone/positrone &
    $\mrb{Z,A}^{\ast} \rightarrow \mrb{Z, A} + \gamma$
    \\
    \bottomrule
  \end{tabular}
\end{table}
dove con $\mrb{Z,A}^{\ast}$ abbiamo indicato uno stato eccitato di $\mrb{Z,A}$.

\begin{note}[Conservazione del numero di nucleoni]
  In queste reazioni di decadimento il \textit{numero di nucleoni} deve
  rimanere costante.
\end{note}

\subsection{Legge di decadimento esponenziale}
Indichiamo con $\omega$ la \textit{probabilità di decadimento per unità di
tempo}, ipotizzando che la \textit{probabilità di decadimento sia costante nel
tempo}. Quindi:
\[
  \mdv{P}{t} = \omega = \frac{1}{\tau}
\]
dove $\tau$ rappresenta il \textit{tempo di vita medio}.
Indicando con $N$ il \textit{numero di nuclei} del tipo considerato:
\[
  \md{N} = -\omega N \md{t}
\]
per cui:
\[
  \mdv{N}{t} = - \omega N
\]
allora abbiamo ottenuto la:
\[
  \textbf{Legge di decadimento esponenziale}\qquad
  \boxed{
    N \mrb{t} = N_0 \exp \mcb{- \omega t} = N_0 \exp \mcb{-\frac{t}{\tau}}
  }
\]

Questa legge viene spesso espressa in termini del \textit{\textbf{tempo di
dimezzamento}} $t_{\frac{1}{2}}$, definito da:
\[
  \frac{N \mrb{t_{\frac{1}{2}}}}{N_0} = \frac{1}{2}
\]
quindi:
\[
  \exp \mcb{- \omega t _{\frac{1}{2}}} = \frac{1}{2}
  \mthen
  \exp \mcb{\frac{t _{\frac{1}{2}}}{\tau}} = 2
  \mthen
  t _{\frac{1}{2}} = \tau \ln \mrb{2}
  \mthen
  t _{\frac{1}{2}} \simeq 0.693\tau 
\]

\begin{note}[]
  La quantità misurata in un campione radioattivo non è $N$ (\textit{numero di
  nuclei}) ma l'attività del campione $I$ (\textit{numero di decadimenti al
  secondo nel campione}).
  \[
    I = - \mdv{N}{t} = \omega N
  \]
  quindi la grandezza che viene effettivamente misurata:
  \[
    I \mrb{t} = \omega N_0 \exp \mcb{- \omega t} = I _{0} \exp \mcb{- \omega t}
  \]

\end{note}

Fino ad ora abbiamo considerato il caso particolarmente semplice in cui il
nucleo può avere \textit{un unico modo di decadimento}. Realisticamente questo
non accade, infatti un unico nucleo può avere diversi modi di decadimento.

\subsection{Decadimenti multimodali}
Consideriamo ora il caso in cui un unico atomo può avere \textit{diversi modi
di decadimento}:
\[
  \mdv{N}{t} = - \omega_1 N - \omega_2 N \equiv - \omega N
\]
dove $\omega = \omega_1 + \omega_2$. La \textit{probabilità totale di
decadimento} $\omega$ sarà chiaramente la somma delle probabilità di
decadimento di ogni modo.

Questo implica che si sommino gli inversi delle vite medie per ogni modo di
decadimento:
\[
  \frac{1}{\tau} = \frac{1}{\tau}_{1} + \frac{1}{\tau}_{2}
\]
dove $\tau = \frac{1}{\omega}$, $\tau_1 = \frac{1}{\omega_1}$ e $\tau_2 =
\frac{1}{\omega_2}$.

In questo caso abbiamo che:
\[
  \begin{dcases}
    f_1 = \frac{\omega_1}{\omega}
    \\
    f_2 = \frac{\omega_2}{\omega}
  \end{dcases}
  \mthen
  \msum{i}{} f_i = 1
\]

L'\textit{evoluzione temporale dei nuclei presenti nel campione}:
\[
  N \mrb{t} = N \mrb{0} \exp \mcb{- \omega t}
\]
Allora le attività di decadimento $1$ e $2$ seguono le seguenti leggi
esponenziali:
\begin{align*}
  I_1 \mrb{1} = \omega_1 N \mrb{t} = \omega _{1} N \mrb{0} \exp \mcb{-\omega t}
  = \mrb{I_1}_{0} \exp \mcb{- \omega t}
  \\
  I_2 \mrb{2} = \omega_2 N \mrb{t} = \omega _{2} N \mrb{0} \exp \mcb{-\omega t}
  = \mrb{I_2}_{0} \exp \mcb{- \omega t}
\end{align*}
