%LTeX: language=it
\chapter{Lezione 02/04/2020}
\section{Metodo delle Ellissi (reprise)}
\subsection{Angolo limite nel sistema del Laboratorio}
Per semplicità ma senza perdita di generalità ci poniamo nella condizione $p_z
	= 0$ (con una rotazione attorno all'asse $x$), ovvero riduciamo il problema a
un problema bidimensionale nel piano $\mrb{p_x,p_y}$ nel sistema del
laboratorio e nel corrispondente $\mrb{p_x^\ast,p_y^\ast}$ nel sistema del
centro di massa.

Per il sistema del Laboratorio (\textbf{Lab}) possiamo scrivere:
\begin{equation}
	p_y = p_x \tan \theta
\end{equation}
quindi l'equazione dell'ellisse diventa (nel piano):
\begin{equation}
	\frac{\mrb{p_x - \beta \gamma E^\ast}^2}{\gamma^2 p^{\ast2}}
	+ \frac{p_x^2 \tan^2 \theta}{p^{\ast2}}
	= 1
\end{equation}
che, risolvendo rispetto a $p_x$ ha soluzione:
\begin{equation}
	\frac{p_x^2 + \beta^2 \gamma^2 E^{\ast2} - 2 p_x \beta \gamma E^\ast}{\gamma^2 p^\ast}
	+ p_x^2 \frac{\tan^2 \theta}{p^{\ast2}}
	- 1
	= 0
\end{equation}
quindi:
\begin{equation}
	p_x^2 \mrb{1 + \gamma^2 \tan^2 \theta}
	+ p_x \mrb{-2 \beta \gamma E^\ast}
	+ \mrb{\beta^2 \gamma^2 E^{\ast2}
		- \gamma^2 p^{\ast2}}
	= 0
\end{equation}
Le soluzioni:
\begin{equation}
	p_x^\pm = \frac{\beta \gamma E^\ast \pm \sqrt{\mrb{\beta \gamma
				E^\ast}^2 - \mrb{1 + \gamma^2 \tan^2 \theta} \mrb{\beta^2 \gamma^2 E^{\ast2} -
				\gamma^2 p^{\ast2}}}}{1 + \gamma^2 \tan^2 \theta}
\end{equation}
Il discriminante dell'equazione:
\begin{equation}
	\Delta = \mrb{\beta \gamma E^\ast}^2 - \mrb{1 + \gamma^2 \tan^2 \theta}
	\mrb{\beta^2 \gamma^2 E^{\ast2} - 			\gamma^2 p^{\ast2}} = \gamma^2 E^{\ast2}
	\msb{\beta^\ast - \gamma^2 \tan^2 \theta \mrb{\beta^2 - \beta^{\ast2}}}
\end{equation}
e, poiché $\beta^{\ast2} > 0$:
\begin{itemize}
	\item se $\beta < \beta^\ast$, ovvero $\beta^2 - \beta^{\ast2} < 0$, allora il
	      discriminante $\Delta$ sarà positivo e l'equazione ammetterà \textit{due
		      soluzioni} $p_x^\pm$ \textit{reali e distinte}, per qualunque angolo
	      $\theta$;
	\item se $\beta^\ast < \beta$, allora si avrà discriminante positivo, quindi due
	      soluzioni, se e solo se:
	      \begin{equation}
		      \gamma^2 \tan^2 \theta \mrb{\beta^2 - \beta^{\ast2}} \leq \beta^{\ast2}
	      \end{equation}
	      ovvero:
	      \begin{equation}
		      \Rightarrow \tan^2 \theta \leq \frac{\beta^{\ast2}}{\gamma^2 \mrb{\beta^2
				      - \beta^{\ast2}}}
	      \end{equation}
	      Avremo dunque una condizione di \textit{angolo limite} $\theta_{max}$ nel
	      sistema del \textbf{Lab}:
	      \begin{equation}
		      \boxed{
			      \tan^2 \theta_{max}
			      = \frac{\beta^{\ast2}}{\gamma^2 \mrb{\beta^2 - \beta^{\ast 2}}}
		      }
	      \end{equation}
	      e avremo due soluzioni $p_x^\pm$ solo se $\theta < \theta_{max}$. In
	      corrispondenza dell'angolo limite le soluzioni saranno coincidenti e
	      l'impulso considerato è \textit{tangente} all'ellisse.
\end{itemize}

% TODO: figure

\subsection{Angolo limite nel sistema del Centro di Massa}
Possiamo valutare l'angolo $\theta^\ast\mrb{\theta_{max}}$ corrispondente
all'\textit{angolo limite nel sistema del centro di massa} \textbf{CM}.
Effettuando una trasformazione di Lorentz dell'impulso dal sistema del
laboratorio a quello del centro di massa ($\textbf{Lab} \to \textbf{CM}$):
\begin{equation}
	\begin{dcases}
		p_x = \gamma \mrb{p_x^\ast + \beta E^\ast}
		\\
		p_y = p_y^\ast
	\end{dcases}
\end{equation}
quindi:
\begin{equation}
	\tan \theta = \frac{p_y}{p_x} = \frac{p_y^\ast}{\gamma \mrb{p_x^\ast + \beta E^\ast}}
	= \frac{\beta^\ast \sin \theta^\ast}{\gamma \mrb{\beta^\ast \cos \theta^\ast + \beta}}
\end{equation}
Imponendo l'angolo limite:
\begin{align}
	 & \frac{\beta^\ast \sin \theta^\ast\mrb{\theta_{max}}}{\gamma \msb{\beta^\ast \cos
			\theta^\ast\mrb{\theta_{max}} + E^\ast}} = \sqrt{\frac{\beta^{\ast2}}{\gamma^2
			\mrb{\beta^2 - \beta^{\ast2}}}}
	\\\notag
	 & \Rightarrow \frac{\cancel{\beta ^{\ast\,2}} \sin ^{2} \theta^\ast
		\mrb{\theta _\text{max}}}{\cancel{\gamma^2} \msb{\beta^\ast \cos
			\theta^\ast \mrb{\theta_\text{max}} + \beta}^{2}} = \frac{\cancel{\beta
			^{\ast\,2}}}{\cancel{\gamma^2} \mrb{\beta^2 - \beta ^{\ast\,2}}}
	\\\notag
	 & \Rightarrow \beta^2 \sin ^{2} \theta^\ast \mrb{\theta _\text{max}} - \beta
	^{\ast\,2} \sin ^{2} \theta^\ast \mrb{\theta _\text{max}} = \beta
	^{\ast\,2} \cos ^{2} \theta^\ast \mrb{\theta _\text{max}} + \beta^2 + 2
	\beta \beta^\ast \cos \theta^\ast \mrb{\theta _\text{max}}
	\\\notag
	 & \Rightarrow \beta^2 \sin \theta^\ast \mrb{\theta _\text{max}} - \beta
	^{\ast\,2} - \beta^2 - 2 \beta \beta^\ast \cos \theta^\ast \mrb{\theta
		_\text{max}} = 0
	\\\notag
	 & \Rightarrow \beta^2 \msb{1 -\cos \theta^\ast \mrb{\theta _\text{max}}} -
	\beta ^{\ast\,2} - \beta^2 - 2 \beta \beta^\ast \cos \theta^\ast
	\mrb{\theta _\text{max}} = 0
	\\\notag
	 & \Rightarrow \cancel{\beta^2} - \beta^2 \cos \theta^\ast \mrb{\theta
		_\text{max}} - \beta ^{\ast\,2} - \cancel{\beta^2} - 2 \beta \beta^\ast
	\cos \theta^\ast \mrb{\theta _\text{max}} = 0
	\\\notag
	 & \Rightarrow \msb{\beta \cos \theta^\ast \mrb{\theta_\text{max}} +
		\beta^\ast}^{2} = 0
\end{align}
abbiamo così ottenuto una relazione per l'\textit{angolo $\theta^\ast$ visto dal
	sistema del centro di massa, che corrisponde all'angolo $\theta_{max}$ limite
	visto dal sistema del laboratorio}:
\begin{equation}
	\boxed{\cos \theta^\ast\mrb{\theta_{max}} = -\frac{\beta^\ast}{\beta}}
\end{equation}

\begin{note}[]
	È importare che l'angolo $\theta$ è sempre nell'emisfero sinistro, infatti
	$\cos \theta \leq 0$ ed esiste solo per $\beta > \beta ^{\ast}$.
\end{note}

\begin{example}[Angolo limite]
	Vogliamo studiare il decadimento della particella $\Sigma^+$ con un
	\textit{tracciatore}:
	\begin{equation}
		\pi^+ + p \rightarrow \Sigma^+ + k^+
	\end{equation}
	Le masse delle particelle sono:
	\begin{equation}
		m _{\pi} = \qty{0.1396}{\GeV \per c^2};
		\qquad
		m _{p} = \qty{0.9383}{\GeV \per c^2};
		\qquad
		m _{\Sigma} = \qty{1.189}{\GeV \per c^2};
		\qquad
		m _{k} = \qty{0.4937}{\GeV \per c^2}
	\end{equation}
	Mentre l'impulso iniziale dei pioni è $p _{\pi^+} = \qty{20}{\GeV \per c}$.
	Vediamo in figura~\ref{fig:rilevatore} uno schema del rivelatore.

	% TODO: figura tracciatore (cilindro) HANDWRITTEN 2023-03-21
	Rispondere ai seguenti quesiti:
	\begin{enumerate}
		\item il rilevatore è in grado di vedere tutte le $\Sigma^+$ prodotte? (in
		      altri termini: esiste un angolo massimo? e se sì, qual è?);
		\item le particelle $\Sigma^+$ sono instabili e decadono con un tempo di
		      vita media (a riposo) $\tau_\Sigma = \qty{0.799e-10}{s}$.
		      Assumiamo che tutte le particelle $\Sigma^+$ siano decadute dopo un
		      tempo $3 \tau$; quale deve esser la dimensione $L$ affinché il punto
		      di decadimento sia contenuto nel rivelatore?;
		\item calcolare il raggio minimo del rilevatore affinché esso contenga
		      tutti i vertici di decadimento delle $\Sigma^+$;
		\item possiamo vedere tutte le particelle $k^+$?;
		\item se così non dovesse essere, quale frazione del numero di particelle
		      $k^+$ che vediamo?
	\end{enumerate}

	\paragraph{Soluzione}
	\begin{enumerate}
		\item \textit{Possiamo vedere tutte le particelle prodotte dalla reazione
			      se l'angolo massimo è $\leq \qty{90}{\degree}$}, considerato come è
		      realizzato il rilevatore. L'energia dei pioni incidenti:
		      \begin{equation}
			      \eps _{\pi} = \sqrt{\abs{\vec{p}_{\pi^+}}^{2} + m _{\pi^+}^{2}}
			      \simeq \qty{20}{\GeV}
		      \end{equation}
		      Quindi:
		      \begin{equation}
			      \qvec{P} = \Set{\eps _{\pi^+} + m _{p}, \vec{p}_{\pi^+}}
		      \end{equation}
		      Calcoliamo l'\textit{energia disponibile nel centro di massa}, che
		      \textit{corrisponde alla massa invariante del sistema} e che ci sarà
		      utile per passare le quantità cinematiche da un sistema di
		      riferimento all'altro in quanto invariante relativistico:
		      \begin{equation}
			      E ^{\ast\,2}
			      = \mrb{\eps _{\pi^+} + m _{p}}^{2} - \abs{\vec{p} _{\pi^+}}^{2}
			      = \eps _{\pi^+}^{2} + m _{p}^{2} + 2 m_p \eps _{\pi^+} -
			      \abs{\vec{p} _{\pi^+}}^{2}
			      = m _{p}^{2} + m _{\pi^+}^{2} + 2 m_p \eps _{\pi^+}
		      \end{equation}
		      quindi:
		      \begin{equation}
			      E^\ast
			      = \sqrt{m _{p}^{2} + m _{\pi^+}^{2} + 2 m_p \eps _{\pi^+}}
			      = \qty{6.199}{\GeV}
		      \end{equation}
		      Il beta nel centro di massa:
		      \begin{equation}
			      \beta _{\text{CM}}
			      = \frac{\abs{\vec{p}_{\pi^+}}}{\eps _{\pi^+} + m _{p}}
			      \simeq 0.955187
		      \end{equation}
		      Il gamma del centro di massa:
		      \begin{equation}
			      \gamma _{\text{CM}}
			      = \frac{\eps _{\pi^+} + m _{p}}{E^\ast}
			      \simeq 3.3777
		      \end{equation}
		      L'impulso della particella virtuale che decade a riposo nel sistema
		      del centro di massa:
		      \begin{equation}
			      p^\ast
			      = \frac{\sqrt{
					      \msb{E ^{\ast\,2} - \mrb{m _{\Sigma^+} + m _{k^+}}^{2}}
					      \msb{E ^{\ast\,2} - \mrb{m _{\Sigma^+} - m _{k^+}}^{2}}
				      }}{2 E^\ast} = \qty{2.965}{\GeV \per c}
		      \end{equation}
		      Quindi l'energia nel centro di massa della particella $\Sigma^+$:
		      \begin{equation}
			      \eps ^{\ast}_{\Sigma^+}
			      = \sqrt{p ^{\ast\,2} + m^2 _{\Sigma^+}}
			      \simeq \qty{3.194}{\GeV}
		      \end{equation}
		      allora $\beta ^{\ast}_{\Sigma^+}$:
		      \begin{equation}
			      \beta ^{\ast}_{\Sigma^+}
			      = \frac{p^\ast}{\eps ^{\ast}_{\Sigma^+}}
			      = 0.9283
			      < \beta _{\text{CM}}
		      \end{equation}
		      Questo vuol dire che, per le $\Sigma^+$ \textit{esiste} un angolo
		      limite nel sistema del laboratorio e saremo in grado, dunque, di
		      rilevare \textbf{tutte} le particelle $\Sigma^+$ prodotte.
		      Appurato che esiste un angolo massimo, calcoliamo qual è:
		      \begin{equation}
			      \tan \theta _\text{max}
			      = \frac{\beta _{\Sigma^+}^{\ast}}{
				      \gamma _\text{CM} \sqrt{\beta _\text{CM}^{2}
					      - \beta ^{\ast\,2} _{\Sigma^+}}
			      }
			      = 1.217
		      \end{equation}
		      quindi:
		      \begin{equation}
			      \theta _\text{max}
			      = \arctan \msb{\frac{\beta _{\Sigma^+}^{\ast}}
				      {\gamma _\text{CM} \sqrt{\beta _\text{CM}^{2}
						      - \beta ^{\ast\,2} _{\Sigma^+}}}}
			      = \qty{50.6}{\degree}
		      \end{equation}
		      questo vuol dire che:
		      \begin{equation}
			      \cos \theta^\ast \mrb{\theta _\text{max}}
			      = - \frac{\beta _{\Sigma^+}^{\ast}}{\beta _\text{CM}}
			      = -0.9717
			      \mthen
			      \theta^\ast \mrb{\theta _\text{max}} = \qty{166}{\degree}
		      \end{equation}
		      All'\textit{angolo massimo}, l'\textit{impulso longitudinale} o
		      \textit{componente longitudinale} dell'impulso della particella:
		      \begin{equation}
			      \mrb{p _{\Sigma^+}}_{L}
			      = \gamma _\text{CM} \mrb{
				      p ^{\ast} \cos \theta^\ast \mrb{\theta _\text{max}}
				      + \beta _\text{CM} \eps _{\Sigma^+}^{\ast}
			      }
			      = \qty{0.573}{\GeV \per c}
		      \end{equation}
		      mentre l'\textit{impulso trasverso} o \textit{componente trasversa}
		      dell'impulso della particella:
		      \begin{equation}
			      \mrb{p _{\Sigma^+}}_{T}
			      = \mrb{p _{\Sigma^+}^{\ast}}_{T}
			      = p ^{\ast} \sin \theta^\ast \mrb{\theta _\text{max}}
			      = \qty{0.7}{\GeV \per c}
		      \end{equation}

		\item Rispondiamo alla seconda domanda.
		      \begin{note}[Tempi di vita]
			      Il tempo di vita indicato nel quesito è il tempo di vita a riposo!
		      \end{note}
		      \begin{equation}
			      D _{\Sigma^+}
			      = 3 \tau _{\Sigma^+} \gamma _{\Sigma^+} \beta _{\Sigma^+} c
			      = 3 c \tau _{\Sigma^+} \frac{p _{\Sigma^+}}{m _{\Sigma^+}}
			      = 6.05 \frac{p _{\Sigma^+}}{\si{\GeV \per c}} \si{\cm}
		      \end{equation}
		      Il caso peggiore si ha quando otteniamo l'impulso longitudinale
		      massimo, ovvero l'emissione della $\Sigma^+$ avviene collineare con
		      il beta del centro di massa.
		      L'\textit{impulso massimo longitudinale} nel sistema del laboratorio,
		      quindi:
		      \begin{equation}
			      \mrb{p _{\Sigma^+}} _{L, \text{max}}
			      = \gamma _\text{CM} \mrb{p ^{\ast} + \beta _\text{CM}
				      + \eps _{\Sigma^+}^{\ast}}
			      = \qty{20.3}{\GeV \per c}
		      \end{equation}
		      La \textit{lunghezza minima per il rilevatore}, affinché tutte
		      le particelle $\Sigma^+$ decadano all'interno del rilevatore:
		      \begin{equation}
			      L _\text{min}
			      = D _{\Sigma^+} \mrb{p _{\Sigma^+}}_{L, \text{max}}
			      = \qty{122.8}{\cm}
		      \end{equation}

		\item Rispondiamo al terzo quesito. L'\textit{impulso trasverso massimo}:
		      \begin{equation}
			      \mrb{p _{\Sigma^+}}_{T, \text{max}}
			      = \mrb{p _{\Sigma^+}^{\ast}}_{T}
			      = p^\ast
		      \end{equation}
		      quindi il \textit{raggio minimo per la sezione del rilevatore}
		      affinché tutti i decadimenti siano contenuti nel rilevatore:
		      \begin{equation}
			      R _\text{min}
			      = 6.05 \frac{
				      \mrb{p _{\Sigma^+}}_{T, \text{max}}
			      }{
				      \si{\GeV \per c}} \si{\cm
			      }
			      = \qty{17.9}{\cm}
		      \end{equation}

		\item Rispondiamo alla quarta domanda.
		      Dobbiamo vedere se esiste un angolo massimo e di conseguenza se
		      possiamo vedere tutte le $k^+$ emesse.
		      Sia $m _{k^+}$ la massa della particella $k^+$, con
		      $p ^{\ast} _{k^+}$:
		      \begin{equation}
			      \eps _{k^+}
			      = \sqrt{p ^{\ast\, 2} + m _{k^+}^{2}}
			      = \qty{3.006}{\GeV}
		      \end{equation}
		      allora:
		      \begin{equation}
			      \beta _{k^+}^{\ast}
			      = \frac{p ^{\ast}}{\eps _{k^+}^{\ast}}
			      = 0.9864
		      \end{equation}
		      quindi $\beta _{k^+}^{\ast} > \beta _\text{CM}$. Questo vuol dire che
		      non esiste un angolo massimo per le particelle $k^+$ (ragionevolmente
		      perché sono particelle più leggere e quindi possono essere
		      back-scatterate) e di conseguenza alcune particelle verrano emesse
		      all'indietro (\textit{back-scattering}), quindi non vedremo
		      tutte le particelle emesse.

		\item Rispondiamo alla quinta domanda, quindi calcoliamo qual è la frazione
		      di particelle $k^+$ vediamo nel rilevatore.
		      Le particelle rilevate saranno quelle che nel sistema del laboratorio
		      sono emesse con un angolo di al più $\qty{90}{\degree}$.
		      % TODO: aggiungi figura HANDWRITTEN 2023-03-21
		      Calcoliamo la componente longitudinale di $p ^{\ast}_{k^+}$ in questa
		      configurazione.
		      Il boost di Lorentz per l'impulso longitudinale della particella
		      $k_+$:
		      \begin{equation}
			      \mrb{p _{k^+}}_{L}
			      = \gamma _{\text{CM}} \msb{
				      \mrb{p _{k^+} ^{\ast}} + \beta _{\text{CM}} \eps _{k^+}^{\ast}
			      }
		      \end{equation}
		      Quindi imponendo $\mrb{p _{k^+}}_{L} = 0$, che equivale a dire che la
		      particella nel sistema del laboratorio è emessa con impulso
		      longitudinale nullo, che vuol dire che la particella è emessa con
		      angolo di $\qty{90}{\degree}$:
		      \begin{equation}
			      \gamma _{\text{CM}} \msb{
				      \mrb{p _{k^+} ^{\ast}} + \beta _{\text{CM}} \eps _{k^+}^{\ast}
			      } = 0
			      \msse
			      \mrb{p _{k^+}^{\ast}}_{L}
			      = p ^{\ast} \cos \theta^\ast \mrb{\qty{90}{\degree}}
			      = - \beta _\text{CM} \eps _{k^+}^{\ast}
		      \end{equation}
		      quindi:
		      \begin{equation}
			      \cos \theta^\ast \mrb{\qty{90}{\degree}}
			      = - \beta _\text{CM} \frac{\eps _{k^+}^{\ast}}{p ^{\ast}}
		      \end{equation}
		      ricordando che $p^\ast$ è diretto lungo l'asse $y$.
		      Quindi:
		      \begin{equation}
			      \theta ^{\ast} \mrb{\qty{90}{\degree}}
			      = \arccos \msb{- \beta _\text{CM} \frac{\eps _{k^+}^{\ast}}{p ^{\ast}}}
			      = \qty{165.5}{\degree}
		      \end{equation}
		      La porzione visibile di particelle prodotte:
		      \begin{equation}
			      \tau
			      = \frac{1}{4 \pi} \dint{0}{2 \pi}{\phi}{f}{1}{\cos \theta^\ast}{}
			      = ??? % TODO: controlla qui
			      = \frac{1}{2} \mrb{1 - f} = \frac{1.98687}{2}
			      \equiv 98.4 \%
		      \end{equation}
	\end{enumerate}
\end{example}
