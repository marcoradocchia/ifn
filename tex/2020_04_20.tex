%LTeX: language=it
\chapter{Lezione 02/04/2020}
\section{Metodo delle Ellissi (reprise)}
\subsection{Angolo limite nel sistema del Laboratorio}
Per semplicità ma senza perdità di generalità ci poniamo nella condizione $p_z
= 0$ (con una rotazione attorno all'asse $x$), ovvero riduciamo il problema ad
un problema bidimensionale nel piano $\mrb{p_x,p_y}$ nel sistema del
laboratorio e nel corrispondente $\mrb{p_x^*,p_y^*}$ nel sistema del centro di
massa.

Per il sistema del Laboratorio (\textbf{Lab}) possiamo scrivere:
\begin{equation}
  p_y = p_x \tan \theta
\end{equation}
quindi l'equazione dell'ellisse diventa (nel piano):
\begin{equation}
  \frac{\mrb{p_x - \beta \gamma E^*}^2}{\gamma^2 p^{*2}} + \frac{p_x^2 \tan^2
  \theta}{p^{*2}} = 1
\end{equation}
che, risolvendo rispetto a $p_x$ ha soluzione:
\begin{subequations}
  \begin{equation}
    \frac{p_x^2 + \beta^2 \gamma^2 E^{*2} - 2 p_x \beta \gamma E^*}{\gamma^2
    p^*} + p_x^2 \frac{\tan^2 \theta}{p^{*2}} - 1 = 0
  \end{equation}
  \begin{equation}
    \Rightarrow p_x^2 \mrb{1 + \gamma^2 \tan^2 \theta} + p_x \mrb{-2 \beta
    \gamma E^*} + \mrb{\beta^2 \gamma^2 E^{*2} - \gamma^2 p^{*2}} = 0
  \end{equation}
  \begin{equation}
    \Rightarrow p_x^\pm = \frac{\beta \gamma E^* \pm \sqrt{\mrb{\beta \gamma
    E^*}^2 - \mrb{1 + \gamma^2 \tan^2 \theta} \mrb{\beta^2 \gamma^2 E^{*2} -
    \gamma^2 p^{*2}}}}{1 + \gamma^2 \tan^2 \theta}
  \end{equation}
\end{subequations}
Il discriminante dell'equazione:
\begin{equation}
  \Delta = \mrb{\beta \gamma E^*}^2 - \mrb{1 + \gamma^2 \tan^2 \theta}
  \mrb{\beta^2 \gamma^2 E^{*2} - 			\gamma^2 p^{*2}} = \gamma^2 E^{*2}
  \msb{\beta^* - \gamma^2 \tan^2 \theta \mrb{\beta^2 - \beta^{*2}}}
\end{equation}
e, poiché $\beta^{*2} > 0$:
\begin{itemize}
  \item se $\beta < \beta^*$, ovvero $\beta^2 - \beta^{*2} < 0$, allora il
    discriminante $\Delta$ sarà positivo e l'equazione ammetterà \textit{due
    soluzioni} $p_x^\pm$ \textit{reali e distinte}, per qualunque angolo
    $\theta$;
  \item se $\beta^* < \beta$, allora si avrà discriminante positivo, quindi due
    soluzioni, se e solo se:
    \begin{subequations}
      \begin{equation}
        \gamma^2 \tan^2 \theta \mrb{\beta^2 - \beta^{*2}} \leq \beta^{*2}
      \end{equation}
      \begin{equation}
        \Rightarrow \tan^2 \theta \leq \frac{\beta^{*2}}{\gamma^2 \mrb{\beta^2
        - \beta^{*2}}}
      \end{equation}
    \end{subequations}
    Avremo dunque una condizione di \textit{angolo limite} $\theta_{max}$ nel
    sistema del \textbf{Lab}:
    \begin{equation}
      \boxed{\tan^2 \theta_{max} = \frac{\beta^{*2}}{\gamma^2 \mrb{\beta^2 -
      \beta^{*2}}}}
    \end{equation}
    e avremo due soluzioni $p_x^\pm$ solo se $\theta < \theta_{max}$. In
    corrispondenza dell'angolo limite le soluzioni saranno coincidenti e
    l'impulso considerato è \textit{tangente} all'ellisse.
\end{itemize}

% TODO: figure

\subsection{Angolo limite nel sistema del Centro di Massa}
Possiamo valutare l'angolo $\theta^*\mrb{\theta_{max}}$ corrispondente
all'\textit{angolo limite nel sistema del centro di massa} \textbf{CM}.
Effettuando una trasformazione di Lorentz dell'impulso dal sistema del
laboratorio a quello del centro di massa ($\textbf{Lab} \to \textbf{CM}$):
\begin{equation}
  \begin{dcases}
    p_x = \gamma \mrb{p_x^* + \beta E^*}
    \\
    p_y = p_y^*
  \end{dcases}
\end{equation}
quindi:
\begin{equation}
  \tan \theta = \frac{p_y}{p_x} = \frac{p_y^*}{\gamma \mrb{p_x^* + \beta E^*}}
  = \frac{\beta^* \sin \theta^*}{\gamma \mrb{\beta^* \cos \theta^* + \beta}}
\end{equation}
Imponendo l'angolo limite:
\begin{subequations}
  \begin{equation}
    \frac{\beta^* \sin \theta^*\mrb{\theta_{max}}}{\gamma \msb{\beta^* \cos
    \theta^*\mrb{\theta_{max}} + E^*}} = \sqrt{\frac{\beta^{*2}}{\gamma^2
    \mrb{\beta^2 - \beta^{*2}}}}
  \end{equation}
  \begin{equation}
    \Rightarrow \frac{\cancel{\beta ^{\ast\,2}} \sin ^{2} \theta^\ast
    \mrb{\theta _\text{max}}}{\cancel{\gamma^2} \msb{\beta^\ast \cos
    \theta^\ast \mrb{\theta_\text{max}} + \beta}^{2}} = \frac{\cancel{\beta
    ^{\ast\,2}}}{\cancel{\gamma^2} \mrb{\beta^2 - \beta ^{\ast\,2}}}
  \end{equation} 
  \begin{equation}
    \Rightarrow \beta^2 \sin ^{2} \theta^\ast \mrb{\theta _\text{max}} - \beta
    ^{\ast\,2} \sin ^{2} \theta^\ast \mrb{\theta _\text{max}} = \beta
    ^{\ast\,2} \cos ^{2} \theta^\ast \mrb{\theta _\text{max}} + \beta^2 + 2
    \beta \beta^\ast \cos \theta^\ast \mrb{\theta _\text{max}}
  \end{equation}
  \begin{equation}
    \Rightarrow \beta^2 \sin \theta^\ast \mrb{\theta _\text{max}} - \beta
    ^{\ast\,2} - \beta^2 - 2 \beta \beta^\ast \cos \theta^\ast \mrb{\theta
    _\text{max}} = 0
  \end{equation}
  \begin{equation}
    \Rightarrow \beta^2 \msb{1 -\cos \theta^\ast \mrb{\theta _\text{max}}} -
    \beta ^{\ast\,2} - \beta^2 - 2 \beta \beta^\ast \cos \theta^\ast
    \mrb{\theta _\text{max}} = 0
  \end{equation}
  \begin{equation}
    \Rightarrow \cancel{\beta^2} - \beta^2 \cos \theta^\ast \mrb{\theta
    _\text{max}} - \beta ^{\ast\,2} - \cancel{\beta^2} - 2 \beta \beta^\ast
    \cos \theta^\ast \mrb{\theta _\text{max}} = 0
  \end{equation}
  \begin{equation}
    \Rightarrow \msb{\beta \cos \theta^\ast \mrb{\theta_\text{max}} +
    \beta^\ast}^{2} = 0
  \end{equation}
\end{subequations}
abbiamo così ottenuto una relazione per l'\textit{angolo $\theta^*$ visto dal
sistema del centro di massa, che corrisponde all'angolo $\theta_{max}$ limite
visto dal sistema del laboratorio}:
\begin{equation}
  \boxed{\cos \theta^*\mrb{\theta_{max}} = -\frac{\beta^*}{\beta}}
\end{equation}

\begin{note}[]
  È importare che l'angolo $\theta$ è sempre nell'emisfero sinistro, infatti
  $\cos \theta \leq 0$ ed esiste solo per $\beta > \beta ^{\ast}$.
\end{note}

\begin{example}[Angolo limite]
  Vogliamo studiare il decadimento della particella $\Sigma^+$:
  \begin{equation}
    \pi^+ + p \rightarrow \Sigma^+ + k^+
  \end{equation}
  Le masse delle particelle sono:
  \begin{equation}
    \begin{dcases}
      m _{\pi} = \SI{0.1396}{\GeV \per c^2}
      \\
      m _{p} = \SI{0.9383}{\GeV \per c^2}
      \\
      m _{\Sigma} = \SI{1.189}{\GeV \per c^2}
      \\
      m _{k} = \SI{0.4937}{\GeV \per c^2}
    \end{dcases}
  \end{equation}
  l'impulso iniziale dei pioni è $p _{\pi} = \SI{20}{\GeV \per c}$. Vediamo in
  figura~\ref{fig:rilevatore} uno schema del rivelatore.

  % TODO: figura rilevatore
  Rispondere ai seguenti quesiti:
  \begin{enumerate}
    \item il rilevatore è in grado di vedere tutte le $\Sigma^+$ prodotte (in
      altri termini: esiste un angolo massimo? e se sì, qual è?)?
    \item le particelle $\Sigma^+$ sono instabili e decadono con un tempo di
      vita media $\tau_\Sigma = \SI{0.799e-10}{s}$. Assumiamo che tutte le
      particelle $\Sigma^+$ siano decadute dopo un tempo $3 \tau$; quale deve
      esser la dimensione $L$ affinché il punto di decadimento sia contenuto
      nel rivelatore?
    \item calcolare il raggio minimo del rilevatore affinché eso contenga tutti
      i vertici di decadimento delle $\Sigma^+$;
    \item possiamo vedere tutte le particelle $k^+$?
    \item se così non dovesse essere, quale frazione del numero di particelle
      $k^+$ che vediamo?
  \end{enumerate}

  \paragraph{Soluzione}
  \begin{enumerate}
    \item \textit{Possiamo vedere tutte le particelle prodotte dalla reazione
      se l'angolo massimo è $\leq \SI{90}{\degree}$}, considerato come è
      realizzato il rilevatore. L'energia dei pioni incidenti:
      \begin{equation}
        \eps _{\pi} = \sqrt{\abs{\vec{p}_{\pi}}^{2} + m _{\pi}^{2}} \simeq
        \SI{20}{\GeV}
      \end{equation}
      Quindi:
      \begin{equation}
        \qvec{P} = \mcb{\eps _{\pi} + m _{p}, \vec{p}_{\pi}}
      \end{equation}
      L'\textit{energia del centro di massa}, che \textit{corrisponde alla
      massa invariante del sistema}:
      \begin{subequations}
        \begin{equation}
          E ^{\ast\,2} = \mrb{\eps _{\pi} + m _{p}}^{2} - \abs{\vec{p}
          _{\pi}}^{2} = \eps _{\pi}^{2} + m _{p}^{2} + 2 m_p \eps _{\pi} -
          \abs{\vec{p} _{\pi}}^{2} = m _{p}^{2} + m _{\pi}^{2} + 2 m_p \eps
          _{\pi}
        \end{equation}
        \begin{equation}
          E^\ast = \sqrt{m _{p}^{2} + m _{\pi}^{2} + 2 m_p \eps _{\pi}} =
          \SI{6.199}{\GeV}
        \end{equation}
      \end{subequations}
      Il $\beta$ nel centro di massa:
      \begin{equation}
        \beta _{\text{CM}} = \frac{\abs{\vec{p}_{\pi}}}{\eps _{\pi} + m _{p}} =
        0.955187
      \end{equation}
      Il $\gamma$ del centro di massa:
      \begin{equation}
        \gamma _{\text{CM}} = \frac{\eps _{\pi} + \eps _{p}}{E^\ast} \simeq
        3.3777
      \end{equation}
      L'impulso della particella virtuale che decade a riposo nel sistema del
      centro di massa:
      \begin{equation}
        p^\ast = \frac{\sqrt{\msb{E ^{\ast\,2} - \mrb{m _{\Sigma} + m
        _{k}}^{2}} \msb{E ^{\ast\,2} - \mrb{m _{\Sigma} - m _{k}}^{2}}}}{2
        E^\ast} = \SI{2.965}{\GeV \per c}
      \end{equation}
      Quindi l'energia nel centro di massa della particella $\Sigma$:
      \begin{equation}
        \eps ^{\ast}_{\Sigma} = \sqrt{p ^{\ast\,2} + m^2 _{\Sigma}} \simeq
        \SI{3.194}{\GeV}
      \end{equation}
      allora $\beta ^{\ast}_{\Sigma}$:
      \begin{equation}
        \beta ^{\ast}_{\Sigma} = \frac{p^\ast}{\eps ^{\ast}_{\Sigma}} = 0.9283
        < \beta _{\text{CM}}
      \end{equation}
      Questo vuol dire che, per le $\Sigma^+$ \textit{esiste} un angolo limite
      nel sistema del laboratorio e saremo in grando, dunque, di rilevare
      \textbf{tutte} le particelle $\Sigma^+$ prodotte.
      Appurato che esiste un angolo massimo, calcoliamo qual è:
      \begin{subequations}
        \begin{equation}
          \tan \theta _\text{max} = \frac{\beta _{\Sigma}^{\ast}}{\gamma
          _\text{CM} \sqrt{\beta _\text{CM}^{2} - \beta ^{\ast\,2} _{\Sigma}}}
          = 1.217
        \end{equation}
        \begin{equation}
          \Rightarrow \theta _\text{max} = \arctan \msb{\frac{\beta
          _{\Sigma}^{\ast}}{\gamma _\text{CM} \sqrt{\beta _\text{CM}^{2} -
          \beta ^{\ast\,2} _{\Sigma}}}} = \SI{50.6}{\degree}
        \end{equation}
      \end{subequations}
      questo vuol dire che:
      \begin{subequations}
        \begin{equation}
          \cos \theta^\ast \mrb{\theta _\text{max}} = - \frac{\beta
          _{\Sigma}^{\ast}}{\beta _\text{CM}} = -0.9717
        \end{equation}
        \begin{equation}
          \Rightarrow \theta^\ast \mrb{\theta _\text{max}} = \SI{166}{\degree}
        \end{equation}
      \end{subequations}
      All'\textit{angolo massimo}, l'\textit{impulso longitudinale}:
      \begin{equation}
        \mrb{p _{\Sigma}}_{L} = \gamma _\text{CM} \mrb{p ^{\ast} \cos
        \theta^\ast \mrb{\theta _\text{max}} + \beta _\text{CM} \eps
        _{\Sigma}^{\ast}} = \SI{0.573}{\GeV \per c}
      \end{equation}
      mentre l'\textit{impulso trasverso}:
      \begin{equation}
        \mrb{p _{\Sigma}}_{T} = \mrb{p _{\Sigma}^{\ast}}_{T} = p ^{\ast} \sin
        \theta^\ast \mrb{\theta _\text{max}} = \SI{0.7}{\GeV \per c}
      \end{equation}

    \item Rispondiamo alla seconda domanda.
      \[
        D _{\Sigma} = 3 \tau _{\Sigma} \gamma _{\Sigma} \beta _{\Sigma} c = 3 c
        \tau _{\Sigma} \frac{p _{\Sigma}}{m _{\Sigma}} =
        6.05 \frac{p _{\Sigma}}{\si{\GeV \per c}} \si{\cm}
      \]
      L'impulso massimo longitudinale nel sistema del laboratorio:
      \[
        \mrb{p _{\Sigma}} _{L, \text{max}} = \gamma _\text{CM} \mrb{p ^{\ast} +
        \beta _\text{CM} + \eps _{\Sigma}^{\ast}} = \SI{20.3}{\GeV \per c}
      \]
      La lunghezza minima per il rilevatore, affinché tutte le particelle
      $\Sigma$ decadano all'interno del rilevatore:
      \[
        L _\text{min} = D _{\Sigma} \mrb{p _{\Sigma}}_{L, \text{max}} =
        \SI{122.8}{\cm}
      \]
    
    \item Rispondiamo al terzo quesito. L'impulso trasverso massimo:
      \[
        \mrb{p _{\Sigma}}_{T, \text{max}} = \mrb{p _{\Sigma}^{\ast}}_{T} =
        p^\ast
      \]
      quindi il raggio minimo del rilevatore:
      \[
        R _\text{min} = 6.05 \frac{\mrb{p _{\Sigma}}_{T, \text{max}}}{\si{\GeV
        \per c}} \si{\cm} = \SI{17.9}{\cm}
      \]

    \item Rispondiamo alla quarta domanda. Sia $m _{k^+}$ la massa della
      particella
      $k^+$, con $p ^{\ast} _{k^+}$:
      \[
        \eps _{k} = \sqrt{p ^{\ast\, 2} + m _{k}^{2}} = \SI{3.006}{\GeV}
      \]
      allora:
      \[
        \beta _{k}^{\ast} = \frac{p ^{\ast}}{\eps _{k}^{\ast}} = 0.9864
      \]
      quindi $\beta _{k}^{\ast} > \beta _\text{CM}$. Questo vuol dire che
      alcune particelle verrano emesse all'indietro, quindi non vedremo tutte
      le particelle emesse.

    \item Rispondiamo alla quinta domanda.
      \[
        \mrb{p _{k}^{\ast}}_{L} = p ^{\ast} \cos \theta^\ast \mrb{\SI{90}{\degree}}
        = - \beta _\text{CM} \eps _{k}^{\ast}
      \]
      quindi:
      \[
        \cos \theta^\ast \mrb{\SI{90}{\degree}} = - \beta _\text{CM} \frac{\eps
        _{k}^{\ast}}{p ^{\ast}}
      \]
      ricordando che $p^\ast$ è diretto lungo l'asse $y$.
      Quindi:
      \[
        \theta ^{\ast} \mrb{\SI{90}{\degree}} = \SI{165.5}{\degree}
      \]
      La porzione visibile di particelle prodotte:
      \[
        \tau = \frac{1}{4 \pi} \dint{0}{2 \pi}{\phi}{f}{1}{\cos \theta^\ast}{} = % TODO: controlla qui
        \frac{1}{2} \mrb{1 - f} = \frac{1.98687}{2} \equiv 98.4 \%
      \]
  \end{enumerate}
\end{example}
