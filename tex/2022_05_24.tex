%LTeX: language=it
\chapter{2022-05-24}
\section{Instabilità dei nuclei}
\subsection{Decadimenti $\beta$}
Decadimenti $\beta^+$:
\[
  \mrb{A, Z} \rightarrow \mrb{A, Z - 1} + e^+ + \nu_e
\]
Il decadimento è possibile se:
\[
  m \mrb{A, Z} c^2 \geq m \mrb{A, Z - 1} c^2 + m_e c^2
\]
e moltiplicando entrambi i membri per $Z m_e c^2$: % TODO: riascolta
\[
  \mu \mrb{A, Z} c^2 \geq \mu \mrb{A, Z - 1} c^2 + 2 m_e c^2
\]
Oppure, sempre per decadimenti $\beta^+$:
\[
  e^+ + \mrb{A, Z} \rightarrow \mrb{A, Z - 1} + \nu_e
\]
che è possibile se:
\[
  m_e c^2 + m \mrb{A, Z} c^2 \geq m \mrb{A, Z - 1} c^2
\]
e moltiplicando entrambi i membri per $\mrb{Z - 1} m_e c^2$: % TODO: riascolta
\[
  \mu \mrb{A, Z} c^2 \geq \mu \mrb{A, Z - 1} c^2
\]

Decadimenti $\beta^-$:
\[
  \mrb{A, Z} \rightarrow \mrb{A, Z + 1} + e^- + \cc{\nu}_e
\]
che è possibile se:
\[
  m \mrb{A, Z} c^2 \geq m \mrb{A, Z + 1} c^2
\]
e moltiplicando entrambi i membri per $Z m_e c^2$: % TODO: riascolta
\[
  \mu \mrb{A, Z} c^2 \geq \mu \mrb{A, Z + 1} c^2
\]

\begin{note}[]
  Ci si aspetta un andamento \textit{smooth} con un \textit{unico isobaro
  stabile}.
\end{note}

\subsection{Decadimenti $\alpha$}
Un decadimento $\alpha$ è un decadimento del tipo:
\[
  \mrb{Z, A} \rightarrow \mrb{Z - 2, A - 4} + \mrb{2, 4}
\]
dove $\mrb{2, 4}$ rappresenta la \textit{particella $\alpha$}.
Il processo è possibile se:
\[
  m \mrb{Z, A} c^2 - \msb{m \mrb{Z - 2, A - 4} c^2 + m \mrb{2, 4} c^2} \geq 0
\]
Il processo si può riscrivere in termini di \textit{energie di legame}. Il
processo di \textit{decadimento $\alpha$} è possibile se vale la seguente
relazione fra le energie di legame:
\[
  B \mrb{Z - 2, A - 4} + B \mrb{2, 4} - B \mrb{Z, A} \geq 0
\]
Quindi, muovendoci lungo la \textit{\textbf{valle di stabilità}}:
\[
  B \mrb{2, 4} \geq B \mrb{Z, A} - B \mrb{Z - 2, A - 4} \simeq \mdv{B}{A} 4
\]
$\eps$ è l'\textit{energia di legame per nucleone}:
\[
  \eps = \frac{B}{A}
\]
allora:
\[
  \mdv{\eps}{A} = \mdv{\mrb{\nicefrac{B}{A}}}{A} = \frac{1}{A} \mdv{B}{A} -
  \frac{1}{A^2} B
\]
quindi:
\[
  \mdv{B}{A} = A \mdv{\eps}{A} + \eps
\]
Quindi sostituendo vediamo che il decadimento $\alpha$ diventa
\textit{energeticamente possibile quando}:
\[
  B \mrb{2, 4} \geq 4 \mcb{A \mdv{\eps}{A} + \eps}
\]
dove $B \mrb{2, 4} \simeq \SI{28.3}{\MeV}$ è l'\textit{energia di legame
di} $\ch{^4He}$, dove $A \gtrsim 100$, $\mdv{\eps}{A} \simeq -
\SI{7.7e-3}{\MeV}$ e $\nicefrac{B \mrb{2, 4}}{4} = \SI{7.075}{\MeV}$. Quindi
possiamo dire che:
\[
  \SI{7.075}{\MeV} \geq - \SI{7.7e-3}{\MeV} \cdot A + \eps
\]
allora:
\[
  \boxed{
    \eps \leq \SI{7.075}{\MeV} + \SI{7.7e-3}{\MeV} \cdot A
  }
\]

% TODO: qui ha fatto vedere quello che si trova a pagina 21 (ultima pagina) del
% file di appunti 06_decadiemnti_dei_nuclei.pdf
% TODO: vedi anche registrazione

\begin{note}[]
  Al crescere del numero atomico l'interazione coulombiana rendono i nuclei
  instabili (vedi plot energia media di legame in funzione del numero atomico).
\end{note}

\subsubsection{Vita media rispetto al decadimento $\alpha$}
La \textit{vita media rispetto al decadimento $\alpha$ è estremamente
variabile}, sebbene le interazioni alla base dei decadimenti sono le stesse
(interazioni forti e interazioni elettromagnetiche).
Quindi: \textit{stessa fisica, ma risultati molto diversi fra loro}.
\begin{example}[]
  Due esempi di decadimenti $\alpha$ con tempi di vita media di molti ordini di
  grandezza differenti fra loro:
  \begin{align*}
    \ch{^{232}_{90}Th} \longrightarrow \ch{^{228}_{88} Ra} + \alpha
    \qquad
    \tau \simeq \SI{2.0e10}{y}
    \\
    \ch{^{212}_{84} Po} \longrightarrow \ch{^{208}_{82} Pb} + \alpha
    \qquad
    \tau \simeq \SI{4.3e-7}{s}
  \end{align*}

\end{example}

\paragraph{Legge di Geiger-Nuttal}
\[
  \textbf{Legge di Geiger-Nuttal}\qquad
  \boxed{
    \log_{10} \mrb{\omega} = B \log _{10} \mrb{Ra} + C
  }
\]
dove:
\begin{itemize}
  \item $\omega = \nicefrac{1}{\tau} = \msb{\si{s^{-1}}}$ è il \textit{rate di
    transizione}
  \item $R _{\alpha} = \msb{\si{\cm}}$ è il \textit{\quot{range}}\footnote{
      La distanza percorsa nel materiale in cui viene prodotta, prima che
      questa si fermi
    } della particella $\alpha$
\end{itemize}
Quindi abbiamo che $\omega \propto R _{\alpha}^{B}$. % TODO: ??? è proporzionale??? (riascolta la lezione)
Il range della particella $\alpha$ dipende dalla sua energia cinetica:
\[
  R _{\alpha} \propto T _{\alpha}^{\nicefrac{3}{2}} = \mcb{\frac{M _{d}}{M _{d}
  + M \alpha} Q _{\alpha}}^{\nicefrac{3}{2}}
\]

\subsubsection{Descrizione fenomenologica del decadimento $\alpha$}
Dato il decadimento $\alpha$:
\[
  \mrb{Z, A} \rightarrow \mrb{Z - 2, A - 4} + \alpha
\]
abbiamo che $T _{\alpha} \simeq Q _{\alpha}$ (dove $Q _{\alpha}$ è la
differenza di energia di legame del nucleo padre rispetto al nucleo figlio).
Assumiamo:
\begin{itemize}
  \item particella $\alpha$ è \textit{\quot{pre-formata}} all'interno del
    nucleo padre;
  \item transizione da uno stato legato ad uno stato libero ad energia $Q
    _{\alpha}$ (trascuriamo il \textit{rinculo del nucleo figlio}, i.e. $T _{d}
    \ll T _{\alpha} \simeq Q _{\alpha}$).
\end{itemize}

% TODO: figura a pagina 3 del file Decadimento alpha.pdf

\[
  V \mrb{r} = \frac{Z_0 Z _{\alpha} e^2}{4 \pi \eps_0} \frac{1}{r} \simeq
  \frac{Z_0 Z _{\alpha} e^2 \hbar c}{4 \pi \eps_0 \hbar c} \frac{1}{r} \simeq
  \frac{Z_0 Z _{\alpha} \alpha \mrb{\hbar c}}{r}
\]
dove $\alpha \simeq \nicefrac{1}{137}$ è la \textit{costante di struttura
fine}.

% TODO: riascolta tutto il discorso
Quindi la spiegazione dell'estrema variabilità della vita media rispetto al
decadimento $\alpha$ è associabile all'\textit{effetto tunnel quantistico}.

\paragraph{Effetto tunnel}
% TODO: figure
Abbiamo una particella ad energia $E$ fissata con $E < U$. L'equazione di
Schrodinger per questa particella:
\[
  i \hbar \mpdv{}{t} \ket{p} = \ham \ket{\psi}
\]
dove l'Hamiltoniana:
\[
  \ham = \frac{p^2}{2m} + V
\]
Gli autostati dell'Hamiltoniana sono gli stati che soddisfano l'equazione:
\[
  \ham \ket{\psi} = E \ket{p}
\]
Nel caso monodimensionale diventa:
\[
  - \frac{\hbar^2}{2m} \mdv[2]{}{x} \psi \mrb{x} + V \mrb{x} \psi \mrb{x} = E
  \psi \mrb{x}
\]
Risolviamo l'equazione di Schrodinger separatamente nelle tre regioni spaziali
I, II, III e successivamente imponiamo delle condizioni di raccordo (ossia
imponendo che la funzione d'onda e la sua derivata siano continue):
\begin{itemize}
  \item zona I ($x \leq 0$):
    \[
      \psi \mrb{x} = A_R e^{ikx} + A_L e^{-ikx}
    \]
    con:
    \[
      \hbar k = \sqrt{2m E}
    \]
  \item zona II ($0 < x < L$):
    \[
      \psi \mrb{x} = B_R e^{Kx} + B_L e^{-Kx}
    \]
    con:
    \[
      \hbar K = \sqrt{2m \mrb{U - E}}
    \]
  \item zona III:
    \[
      \psi \mrb{x} = C_R e^{ik \mrb{x - L}} + C_L e^{-ik \mrb{x - L}}
    \]
    con:
    \[
      \hbar k = \sqrt{2m E}
    \]
\end{itemize}

Imponiamo \textit{continuità in $x = 0$}:
\[
  A_R + A_L = B_R + B_L
\]
Imponiamo la \textit{derivata prima continua in $x = 0$}:
\[
  \mrb{A_R - A_L} ik = \mrb{B_R - B_L} K
\]
Imponiamo la \textit{continuità in $x = L$}:
\[
  B_R e^{KL} + B_L e^{-KL} = C_R + C_L
\]
Imponiamo la \textit{derivata prima continua in $x = L$}:
\[
  \mrb{B_R e^{KL} - B_L e^{-KL}}K = \mrb{C_R - C_L} ik
\]

Cerchiamo soluzioni che abbiamo le seguenti proprietà:
\[
  \begin{dcases}
    A_R = 1
    \\
    A_L = ? = r \quad (\textit{riflessione})
    \\
    B_R = ?
    \\
    B_L = ?
    \\
    C_R = ? = t \quad (\textit{trasmissione})
    \\
    C_L = 0
  \end{dcases}
\]
quindi abbiamo $4$ coefficienti ignoti.
Otteniamo:
\begin{align*}
  \begin{rcases}
    1 + r = B_R + B_L
    \\
    \mrb{1 - r} \frac{ik}{K} = B_R - B_L
  \end{rcases}
  \mthen
  \begin{dcases}
    B_R = \frac{1}{2} \msb{\mrb{1 + r} + \frac{ik}{K} \mrb{1 - r}}
    \\
    B_L = \frac{1}{2} \msb{\mrb{1 + r} + \frac{ik}{K} \mrb{1 - r}}
  \end{dcases}
\end{align*}
e:
\begin{align*}
  \begin{rcases}
    t = B_R' + B_L'
    \\
    t \frac{ik}{K} = B_R' + B_L'
  \end{rcases}
  \mthen
  \begin{dcases}
    B_R' = \frac{1}{2} \msb{t + \frac{ik}{K} t}
    \\
    B_L' = \frac{1}{2} \msb{t - \frac{ik}{K} t}
  \end{dcases}
\end{align*}
dove $B_R' \equiv B_R e^{KL}$ e $B_L' \equiv B_L e^{-KL}$.
Dalle relazioni precedenti:
\begin{align*}
  \begin{dcases}
    \msb{\mrb{1 + i\delta} + r \mrb{1 - i \delta}} = e^{-KL} t \mrb{1 + i \delta}
    \\
    \msb{\mrb{1 - i\delta} + r \mrb{1 + i \delta}} = e^{-KL} t \mrb{1 - i \delta}
  \end{dcases}
\end{align*}
dove $\delta \equiv \nicefrac{k}{K}$. Da cui otteniamo:
\[
  \begin{dcases}
    r = \frac{e^{-KL} t \mrb{1 + i \delta} - \mrb{1 + i \delta}}{\mrb{1 - i
    \delta}}
    \\
    r = \frac{e^{-KL} t \mrb{1 - i \delta} - \mrb{1 - i \delta}}{\mrb{1 + i
    \delta}}
  \end{dcases}
\]
uguagliando le due espressioni di $r$, otteniamo un'espressione per $t$:
\[
  t = \frac{\mrb{1 + i \delta}^{2} - \mrb{1 - i \delta}^{2}}{\bcancel{e^{-KL}
  \mrb{1 + i \delta}^{2}} - e^{KL} \mrb{1 - i \delta}^{2}} \simeq \frac{4 i
  \delta}{\mrb{1 - i \delta}^{2}} e^{-KL}
\]
quindi:
\[
  t \simeq - \frac{4 i \delta}{\mrb{1 - i \delta}^{2}} e^{-KL}
\]
% TODO: riascolta, perché semplifichiamo?
dove $k$ è il vettore d'onda della funzione d'onda nella zona dove \textit{non}
è presente il potenziale, mentre $K$ è relativo alla zona \textit{classicamente
proibita}. Il \textit{dump esponenziale} è tanto più influente quanto più ampia
è la zona proibita ed intenso il potenziale.
