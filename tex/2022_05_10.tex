%LTeX: language=it
\chapter{2022-05-10}
\section{Modello standard (reprise)}
\subsection{Cinematica del decadimento $\alpha$}
Indichiamo con $D$ la \textit{particella figlia} (\textit{daughter}).
Dato che $\vec{p}_\text{tot} = 0$, dalla \textit{conservazione dell'impulso}:
\[
  \vec{p}_D + \vec{p}_{\alpha} = 0
\]
dove quindi:
\[
  \abs{\vec{p}_{D}} = \abs{\vec{p}_{\alpha}} = p
\]
Uniamo la conservazione dell'energia (siamo in un regime di particelle non
relativistiche):
\[
  m \mrb{Z-2, A-4} c^2 + m \mrb{2,4} c^2 + \frac{p^2}{2m_D} + \frac{p^2}{m
  _{\alpha}} = m \mrb{A,Z} c^2
\]
dove il primo membro rappresenta l'\textit{energia dello stato finale} e il
secondo membro rappresenta l'\textit{energia dello stato iniziale}.

Quindi:
\[
  Q _{\alpha} = \frac{p^2}{2 m_D} + \frac{p^2}{2 m _{\alpha}} = \frac{p^2}{2}
  \mcb{\frac{1}{m_D} + \frac{1}{m _{\alpha}}} = \frac{p^2}{2} \mcb{\frac{m_D +
  m _{\alpha}}{m_D m _{\alpha}}} = \frac{p^2}{2 \mu}
\]
dove $\mu$ rappresenta la \textit{massa ridotta del sistema}. Quindi l'impulso
finale delle due particelle sarà in modulo:
\[
  p^2 = 2 \mu Q _{\alpha}
  \mthen
  p = \sqrt{2 \mu Q _{\alpha}}
\]
La ripartizione di energia fra le particelle $D$ e $\alpha$, invece:
\[
  \begin{dcases}
    T_D = \frac{\mu}{m_D} Q _{\alpha}
    \\
    T _{\alpha} = \frac{2 \mu}{2 m _{\alpha}} Q _{\alpha}
  \end{dcases}
\]

\subsection{Cinematica del decadimento $\beta$}
% TODO: non si legge

\begin{example}[Esercizio per parziale]
  Consideriamo un fascio incidente di $\SI{5}{\u\ampere}$ di elettroni
  ultrarelativistici (questo vuol dire che si è in regime di \textit{scattering
  Mott}) di impulso $q_e$ = \SI{700}{\MeV\per c}, su una \textit{targhetta di
  calcio 40} $\ch{^{40}Ca}$ di densità $\rho = \SI{0.12}{\g \per \cm^2}$.
  ??? $S = \SI{20}{\cm^2}$, $R = \SI{1}{\m}$.

  % TODO: figura

  Formula empirica che fornisce il raggio di un nucleo, data la sua massa
  atomica:
  \[
    R_N = \mrb{118 A ^{\nicefrac{1}{3}} - 0.48} \si{\femto\m}
  \]

  % TODO: copia
  \[
    \mdv{N_e}{t} = \frac{\md{N_e}}{\md{t} \md{\Omega}} \Delta \Omega = \phi_e
    N_T \mdv{\sigma}{\Omega} \frac{S}{R^2} = \mdv{N_i}{t} \frac{N_T}{S \mprime}
    \mdv{\sigma}{\Omega} \frac{S}{R^2}
  \]
  
  \[
    \Delta \Omega = \frac{S}{R^2}
  \]

  $\phi_e$ rappresenta il \textit{flusso di elettroni}, quindi indicando con
  $N_i$ il numero di elettroni incidenti:
  \[
    \phi _{e} = \mdv{N_i}{t} \frac{1}{S \mprime}
  \]
  con $m_a$ l'\textit{unità di massa atomica}:
  \[
    \rho = \frac{m _a N_T}{S \mprime} = ??? % TODO: 
  \]
  Quindi, con $I$ la corrente incidente e $e$ la carica dell'elettrone:
  \[
    \mdv{N_e}{t} = \mdv{N_e}{t} \frac{\rho}{A m_a} \mdv{\sigma}{\Omega}
    \frac{S}{R^2} = \frac{I}{e} \frac{\rho}{A m_a} \frac{S}{R^2}
    \mdv{\sigma}{\Omega}
  \]

  \[
    \mdv{\sigma}{\Omega} = \mrb{\mdv{\sigma}{\Omega}}_\text{Mott} \abs{F
    \mrb{q}}^{2}
  \]
  dove la \textit{sezione d'urto di Mott} vale:
  \[
    \mrb{\mdv{\sigma}{\Omega}}_\text{Mott} = \frac{Z^2 \alpha^2 \mrb{\hbar
    c}^{2}}{4 \mrb{pc}^{2} \sin^4 \mrb{\frac{\theta}{2}}} \cos ^{2}
    \mrb{\frac{\theta}{2}}
  \]
  A questo punto resta da calcolare il \textit{fattore di forma} per una
  distribuzione sferica uniforme:
  \[
    F \mrb{q^2} = \frac{3}{x^3} \mrb{\sin x - x \cos x} % TODO: sicuro???
  \]
  dove $x = \frac{q R_N}{\hbar}$, con $q = 2 p \sin \mrb{\frac{\theta}{2}}$.
  Questo perché ossiamo considerare l'energia assorbita dal rinculo del nucleo
  trascurabile, quindi assumendo $\abs{\vec{p}_\text{in}} =
  \abs{\vec{p}_\text{fin}} = p$, l'\textit{impulso trasferito}, che
  vettorialmente scriviamo $\vec{q} = \vec{p}_\text{fin} - \vec{p}_\text{in}$,
  vale in modulo:
  \[
    \abs{\vec{q}\,} = 2 p \sin \mrb{\frac{\theta}{2}}
  \]
\end{example}
