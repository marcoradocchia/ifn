%LTeX: language=it
% TODO: rivedere bene questa lezione sfruttando la registrazione del 28 aprile 2022
\chapter{Lezione 2020-05-07}
\section{Scattering Rutherford (reprise)}
Fino ad ora abbiamo lavorato in approssimazione di \textit{nucleo puntiforme}.
Questa approssimazione non è più adeguata se vogliamo andare a studiare gli
effetti della dimensione finita del nucleo o studiare la dimensione del nucleo
stesso. Quindi il quesito che ci poniamo è: \textit{come varia l'espressione
della sezione d'urto differenziale di Rutherford se rimuoviamo
l'approssimazione di nucleo puntiforme}?

\subsection{Sezione d'urto per nucleo \textit{non} puntiforme}
Consideriamo il nucleo atomico come una distribuzione estesa di carica per
cui\footnote{
  La notazione utilizzata prevede che $\vec{r}$ sia il vettore e $r$ il suo
  modulo
}:
\begin{itemize}
  \item $\rho(\vec{r}\,)$: \textit{densità di carica} $\longrightarrow \int
    \rho(\vec{r}\,) \, \mathrm{d}^3r = Ze$
  \item $\rho_N \mrb{\vec{r}\,} = \dfrac{\rho\mrb{\vec{r}\,}}{Ze}$:
    \textit{densità di carica normalizzata} $\longrightarrow \int
    \rho_N\mrb{\vec{r}\,} \, \mathrm{d}^3r = 1$
\end{itemize}
Questa modifica alle ipotesi di lavoro si ripercuote nell'espressione che
descrive l'Hamiltoniana di interazione $\ham_\text{int}$. 

\paragraph{Elemento di matrice per distribuzione estesa di carica}
In generale, l'elemento di matrice $T_{if}$ si calcola come:
\begin{equation}
	T_{if} = \braket{f \vert \ham_\text{int} \vert i}
\end{equation}
che abbiamo già visto nel caso puntiforme nelle equazioni
\ref{eq:ham_int_puntiforme} e \ref{eq:elemento_matrice_T_puntiforme}.

Nel caso \textit{non} puntiforme avremo quindi che l'Hamiltoniana di
interazione assume la seguente forma:
\begin{equation}
  \ham_\text{int} \mrb{\vec{r}\,} = \frac{ze}{4\pi \varepsilon_0} \int
  \frac{\rho\mrb{\vec{r}\mprime}}{\abs{\vec{r} - \vec{r}\,\mprime}} \,
  \mathrm{d}^3r\mprime
  = \frac{zZe^2}{4\pi \varepsilon_0} \int
  \frac{\rho_N\mrb{\vec{r}\mprime}}{\abs{\vec{r} - \vec{r}\,\mprime}} \,
  \mathrm{d}^3r\mprime
	\label{eq:ham_int_estesa}
\end{equation}
dove l'integrale è valutato in \textit{tutto lo spazio} e $\abs{\vec{r} -
\vec{r}\,\mprime}$ indica la distanza fra il punto $\vec{r}$ in cui voliamo
valutare l'interazione e il punto $\vec{r}\,\mprime$ dell'elemento infinitesimo
di carica della distribuzione estesa che la genera. Quindi abbiamo
l'espressione esplicita del \textit{potenziale elettrostatico generato da una
distribuzione arbitraria di carica}.

Quindi quando consideriamo la distribuzione estesa, otteniamo l'elemento di
matrice $T_{if}$ effettuando la trasformata di Fourier di
\ref{eq:ham_int_estesa}. Per cui:
\begin{subequations}
	\begin{equation}
    T_{if} = \frac{1}{V} \, \frac{zZe^2}{4\pi \varepsilon_0} \int \mathrm{d}^3r
    \! \int \mathrm{d}^3r\mprime \,
    \frac{\rho_N(\vec{r}\,\mprime)}{\abs{\vec{r} - \vec{r}\,\mprime}}
    \exp\mcb{i \, \frac{\mrb{\vec{p}_f - \vec{p}_i} \cdot \vec{r}}{\hbar}}
	\end{equation}
  Ma espressa in questo modo è di difficile risoluzione, quindi moltiplichiamo
  e dividiamo per \\ $\exp\mcb{-i \, \frac{\mrb{\vec{p}_f - \vec{p}_i} \cdot
  \vec{r}\,\mprime}{\hbar}}$ e otteniamo:
	\begin{equation}
    T_{if} = \frac{1}{V} \, \frac{zZe^2}{4\pi \varepsilon_0} \int \mathrm{d}^3r
    \! \int \mathrm{d}^3r\mprime \,
    \frac{\rho_N(\vec{r}\,\mprime)}{\abs{\vec{r} - \vec{r}\,\mprime}}
    \exp\mcb{i \, \frac{\mrb{\vec{p}_f - \vec{p}_i} \cdot \mrb{\vec{r} -
    \vec{r}\,\mprime}}{\hbar}} \exp\mcb{i \, \frac{\mrb{\vec{p}_f - \vec{p}_i}
    \cdot \vec{r}\,\mprime}{\hbar}}
	\end{equation}
  Osserviamo che ora la possiamo separare gli integrali nelle variabili
  $\vec{r}\,\mprime$ e $\vec{r}\,\mdprime = \vec{r} - \vec{r}\,\mprime$, come
  prodotto di due termini calcolabili in maniera indipendente, come segue:
	\begin{equation}
    T_{if} = \frac{1}{V} \, \frac{zZe^2}{4\pi \varepsilon_0} \int
    \mathrm{d}^3r\mdprime \, \frac{1}{r\mdprime} \exp\mcb{i \,
    \frac{\mrb{\vec{p}_f - \vec{p}_i} \cdot \vec{r}\,\mdprime}{\hbar}} \int
    d^3r\mprime \, \rho_N\mrb{\vec{r}\,\mprime} \exp\mcb{i \,
    \frac{\mrb{\vec{p}_f - \vec{p}_i} \cdot \vec{r}\,\mprime}{\hbar}}
	\end{equation}
\end{subequations}
In più il \textit{primo dei due termini moltiplicativi}, inclusi i prefattori,
costituiscono esattamente la trasformata di Fourier del potenziale puntiforme,
per cui rappresenta l'elemento di matrice del potenziale puntiforme già
calcolata in precedenza\footnote{
	Siamo passati ad utilizzare la quantità \textit{impulso trasferito} $\vec{q}$
}. 
\begin{equation}
  \mrb{T_{if}}_\text{punt} = \frac{1}{V} \, \frac{zZe^2}{4\pi \varepsilon_0}
  \int d^3r\mdprime \, \frac{1}{r\mdprime} \exp\mcb{i \, \frac{\vec{q} \cdot
  \vec{r}\,\mdprime}{\hbar}}
\end{equation}
Il \textit{secondo termine moltiplicativo} rappresenta, invece, la
trasformazione di Fourier della distribuzione normalizzata di carica,
$\rho_N\mrb{\vec{r}\,\mprime}$, con parametro di trasformazione (quindi con
vettore d'onda) $\frac{\vec{q}}{\hbar}$. Questo oggetto è una funzione
complessa di $\vec{q}$ ed e assume il nome di \textit{fattore di forma
associato alla distribuzione di carica} considerata:
\begin{align}
	\textbf{Fattore di Forma} &&
	\boxed{
    F\mrb{\vec{q}\,} = \int d^3r\mprime \, \rho_N\mrb{\vec{r}\,\mprime}
    \exp\mcb{i \, \frac{\vec{q} \cdot \vec{r}\,\mprime}{\hbar}}
	}
\end{align}
e rappresenta la modifica che si apporta all'elemento di matrice del caso
puntiforme per passare alla distribuzione estesa di carica.\\
\begin{note}[]
  Qualora il sistema che si sta considerando fosse dotato di \textit{simmetria
  sferica} la dipendenza di $F$ da $\vec{q}$ riguarderebbe solo il modulo,
  ovvero $F = F\mrb{\vec{q}\,}$.
\end{note}

Alla luce di queste considerazioni possiamo scrivere l'elemento di matrice come segue:
\begin{equation}
	\boxed{
		T_{if} = \mrb{T_{if}}_\text{punt} F\mrb{\vec{q}\,}
	}
\end{equation}

\paragraph{Sezione d'urto differenziale di Rutherford per distribuzione estesa
di carica}
Tenendo conto di questa nuova espressione dell'elemento di matrice di interazione, omettendo tutti i passaggi che formalmente prevedono che si sostituisca quest'ultima nell'espressione della regola d'oro di Fermi (equazione \ref{eq:regola_oro_fermi}) e che si proceda analogamente al caso puntiforme, scriviamo la \textit{sezione d'urto differenziale per scattering Rutherford in ipotesi di nucleo come distribuzione estesa di carica}:
\begin{align}
	\textbf{Sezione d'urto differenziale per distribuzione estesa} && 
	\boxed{
    \frac{d\sigma}{d\Omega} = \mrb{\frac{d\sigma}{d\Omega}}_\text{punt}
    \abs{F\mrb{\vec{q}\,}}^2
	}
\end{align}

\begin{note}[]
  Attenzione, perché questo approccio non determina univocamente
  la funzione $F\mrb{\vec{q}\,}$, ma ne determina solo il modulo; non abbiamo
  alcuna informazione sulla fase. Quindi, ad esempio, non possiamo utilizzare
  questo approccio per la ricostruzione del \textit{fattore di forma} da dati
  sperimentali antitrasformando, perché di nuovo, abbiamo che i dati
  eventualmente ottenuti sarebbero legati solo al modulo di $F$.
\end{note}

\subsection{Deviazione dal comportamento puntiforme}
Studiando le deviazioni del \textit{fattore di forma} dall'unità si possono
valutare le dimensioni dei nuclei.
Della definizione di \textit{fattore di forma},  in ipotesi di simmetria
sferica, distinguiamo i seguenti casi:
\begin{itemize}
  \item $\mrb{\frac{q}{\hbar}} R_N \ll 1$ nelle regioni di spazio in cui la
    distribuzione normalizzata di carica è sensibilmente diversa da 0 si ha
    che:
    \begin{equation}
      F\mrb{q \simeq 0} = 1
    \end{equation}
    per cui gli urti che avvengono con un impulso trasferito piccolo si possono
    considerare urti fra particelle puntiformi. Quindi
    \textit{non possiamo utilizzare urti con} $q \simeq 0$ \textit{per studiare
    le dimensioni del nucleo}.
  \item $\mrb{\frac{q}{\hbar}} R_N \gtrsim 1$ nelle regioni di spazio in cui la
    distribuzione normalizzata di carica è sensibilmente diversa da 0 si ha
    che:
    \begin{equation}
      F\mrb{q} \neq 1
    \end{equation}
\end{itemize}
Come già visto nell'equazione \ref{eq:modulo_q} che riportiamo nuovamente di
seguito (in riferimento alla figura \ref{fig:impulso_trasferito}), se $p =
\abs{\vec{p}_i} = \abs{\vec{p}_f}$:
\begin{equation}
	\abs{\vec{q}\,} = 2p \, \sin\frac{\theta}{2}
\end{equation}
avremo che ad impulso fissato l'impulso trasferito dipenderà dall'angolo
$\theta$. Quindi al variare di tale angolo si registreranno diversi valori di
$q$; per cui avremo che, tramite il fattore di forma, l'angolo $\theta$ modula
la sezione d'urto.

Ora, sapendo che i nuclei hanno dimensioni caratteristiche dell'ordine del
fermi, possiamo chiederci quale sia la condizione sull'impulso trasferito $q$
che causa le deviazioni che ci permettano di studiare le dimensioni del nucleo:
\begin{equation}
  q \gtrsim \frac{\hbar c}{R_N c} \implies qc \gtrsim \frac{\hbar c}{R_N}
  \simeq \frac{\SI{200}{\MeV}}{R_N \: \msb{\si{\femto\meter}}}
\end{equation}
quindi l'impulso trasferito per sondare strutture (nel nostro caso nulcei)
dalle dimensioni dell'ordine di grandezza del \textit{fermi} deve essere
dell'ordine di centinaia di \si{\MeV \per c}.

\section{Modelli per la distribuzione di carica nei nuclei}
% TODO: da qui ci sono state delle differenze nella notazione
\subsection{Sfera uniformemente carica}
Definiamo una distribuzione di carica la cui densità (normalizzata) dipenda
solo dalla distanza dal baricentro del nucleo come segue:
\begin{align}
	\rho \mrb{r} =
	\begin{dcases}
		\rho_0 \quad &r \leq a \\
		0 \quad &r > a
	\end{dcases}
	\label{eq:densita_carica_sfera_uniforme}
\end{align}

\begin{figure}[ht]
  \centering
  \begin{tikzpicture}
    \begin{axis}[
      title=,
      xlabel=$r$,
      ylabel=$\rho \mrb{r}$,
      xmin=0,
      ymin=0,
      xmax=1.5,
      ymax=1.5,
      ytick = {1},
      xtick = {1},
      xticklabels = {$a$},
      yticklabels = {$\rho_0$}
    ]

      \addplot[
        thick,
        domain=0:1
      ] {1};

      \draw[
        dashed
      ] (axis cs:1,0) -- (axis cs:1,1);
    \end{axis}
  \end{tikzpicture}
  \caption{Modello di sfera uniformemente carica}
	\label{fig:modello_sfera_uniformemente_carica}
\end{figure}

questo modello dipende da un \textbf{unico parametro}, poiché:
\begin{equation}
  \int \rho_N\mrb{r} \, \mathrm{d}^3r = 1 \implies \rho_0 =
  \frac{1}{\frac{4}{3}\pi a^3}
	\label{eq:rho_0}
\end{equation}
Quindi per questo semplice modello è possibile studiare analiticamente il
\textit{fattore di forma} e la dipendenza di quest'ultimo dal parametro che
caratterizza il modello, che è il raggio $a$ del nucleo.

\paragraph{Calcolo del fattore di forma}
Il calcolo del fattore di forma è possibile per via analitica e risulta
relativamente semplice. Procediamo con il calcolo:
\begin{equation}
	\begin{split}
    F\mrb{q} &= \int \rho_N\mrb{\vec{x}} \exp\mcb{i \,
    \frac{\vec{q}\cdot\vec{x}}{\hbar}} \, \mathrm{d}^3x
    \\ 
    &= \int_0^\infty \! \mathrm{d}r \, r^2 \int_{-1}^1 \! \mathrm{d}\cos\theta
    \int_0^{2\pi} \! \mathrm{d}\rho \, \rho\mrb{r} \exp\mcb{i \, \frac{qr
    \cos\theta}{\hbar}}
    \\
    &= 2\pi \int_0^\infty \! \mathrm{d}r \, r^2 \rho\mrb{r} \frac{\hbar}{iqr}
    \mcb{\exp\msb{i \, \frac{qr}{\hbar}} - \exp\msb{-i \, \frac{qr}{\hbar}}}
	\end{split}
\end{equation}
Fino ad ora il procedimento è generale perché non abbiamo fissato alcuna
ipotesi, quindi varrà analogamente anche per il modello illustrato
successivamente. Ora imponiamo dunque l'ipotesi di distribuzione sferica e
uniforme di carica (equazione \ref{eq:densita_carica_sfera_uniforme}),
ricordando il valore di $\rho_0$ (equazione \ref{eq:rho_0}):
\begin{subequations}
	\begin{equation}
		\begin{split}
			F\mrb{q}
      &= 2\pi\rho_0 \int_0^a \! \mathrm{d}r \, \frac{r^{\cancel2}
      \hbar}{iq\cancel r} \mcb{\exp\msb{i \, \frac{qr}{\hbar}} - \exp\msb{-i \,
      \frac{qr}{\hbar}}}
      \\
      &= \frac{4\pi\rho_0\hbar}{q} \int_0^a \! \mathrm{d}r \,
      r\sin\mrb{\frac{qr}{\hbar}}
      \\
      &= \frac{4\pi\rho_0\hbar}{q} \mrb{\frac{q}{\hbar}}^2 \int_0^a \!
      \mathrm{d}\mrb{\frac{qr}{\hbar}} \,
      \frac{qr}{\hbar}\sin\mrb{\frac{qr}{\hbar}}
		\end{split}
	\end{equation}
  Ed effettuando il cambio di variabile $y=\frac{qr}{\hbar}$, con il nuovo
  estremo di integrazione $\tilde{y} = \frac{aq}{\hbar}$, e sostituendo il
  valore esplicito di $\rho_0$:
	\begin{equation}
    F\mrb{q} = \frac{3\hbar^3}{a^3q^3} \mrb{\frac{q}{\hbar}}^2
    \int_0^{\tilde{y}} \! y\sin y \, \mathrm{d}y
	\end{equation}
	Svolgendo l'integrale per parti\footnote{
    $\int y\sin y \, \mathrm{d}y = - \int y\mdv{\cos y}{y} \, \mathrm{d}y =
    -y\cos y + \int \cos y \, \mathrm{d}y = -y\cos y + \sin y$
	} otteniamo quanto segue:
	\begin{equation}
    F\mrb{q} = \frac{3\hbar^3}{a^3q^3} \mrb{\frac{q}{\hbar}}^2 \mcb{\msb{-y
    \cos y}_0^{\tilde{y}} + \msb{\sin y}_0^{\tilde{y}}} =
    \frac{3\hbar^3}{a^3q^3} \mrb{\frac{q}{\hbar}}^2 \msb{-\tilde{y}\cos
    \tilde{y} + \sin \tilde{y}}
	\end{equation}
\end{subequations}
Quindi in conclusione possiamo riscrivere il \textbf{fattore di forma per il
modello di sfera uniformemente carica} come segue\footnote{
  In fisica delle particelle è più naturale esprimere le quantità in funzione
  di $q^2$ piuttosto che in funzione di $q$
}:
\begin{equation}
	\boxed{
    F\mrb{q^2} = \frac{3}{\tilde{y}} \msb{\sin\tilde{y} -
    \tilde{y}\cos\tilde{y}}
	}
	\label{eq:fattore_forma_sfera_uniforme}
\end{equation}
Studiamo il comportamento di questa funzione:
\begin{itemize}
	\item $\tilde{y} \ll 1 \rightarrow q \ll \frac{\hbar}{a}$:
    sviluppiamo i termini di $F\mrb{q^2}$ in serie di Taylor:
    \begin{equation}
      \begin{dcases}
        \sin \tilde{y} = \tilde{y} - \frac{\tilde{y}^3}{6} +
        \frac{\tilde{y}^5}{5!} + \cdots
        \\
        \cos \tilde{y} = 1 - \frac{\tilde{y}^2}{2} + \frac{\tilde{y}^4}{4!} +
        \cdots
      \end{dcases}
    \end{equation}
    per cui:
    \begin{subequations}
      \begin{equation}
        \begin{split}
          F\mrb{q^2}
          &= \frac{3}{\tilde{y}^3} \msb{\mrb{\tilde{y} - \frac{\tilde{y}^3}{6}
          + \frac{\tilde{y}^5}{5!} + \cdots} - \tilde{y}\mrb{1 -
          \frac{\tilde{y}^2}{2} + \frac{\tilde{y}^4}{4!} + \cdots}}
          \\
          &= \frac{3}{\tilde{y}^3} \msb{\tilde{y}^3\mrb{-\frac{1}{6} +
          \frac{1}{2}}+ \tilde{y}^5\mrb{\frac{1}{5!} - \frac{1}{4!}} + \cdots}
          \\
          &= \frac{3}{\cancel{\tilde{y}^3}} \, \cancel{\tilde{y}^3}
          \msb{\frac{1}{3} - \frac{\tilde{y}^2}{5!}\mrb{5-1} + \cdots}
          \\
          &= 1 - \frac{\tilde{y}^2}{10} + \cdots
        \end{split}
      \end{equation}
      Quindi, in forma esplicita:
      \begin{equation}
        F\mrb{q^2} \simeq 1 - \frac{1}{10} \mrb{\frac{q}{a\hbar}}^2
        \label{eq:deriv_fattore_forma}
      \end{equation}
    \end{subequations}
    Per cui nel momento in cui $q \ll \frac{\hbar}{a}$ il fattore di forma
      tende a 1 e possiamo assumere il nucleo come puntiforme.

    Inoltre osserviamo che la derivata del fattore di forma rispetto alla
      variabile $q^2$:
    \begin{equation}
      \frac{\mathrm{d}F\mrb{q^2}}{\mathrm{d}q^2} = -\frac{a^2}{10\hbar^2}
    \end{equation}

    In \textbf{termini generali} si può dimostrare che esiste una
    relazione di proporzionalità fra il raggio di estensione di una
    distribuzione di carica (qualunque, purché dotata di simmetria sferica) e
    la derivata del suo fattore di forma rispetto a $q^2$. Infatti, se
    definiamo il \textit{raggio medio quadro della distribuzione}\footnote{
      Per la distribuzione sferica uniforme si riottiene il risultato espresso
      nell'equazione \ref{eq:deriv_fattore_forma}, poiché $\braket{r^2} =
      \frac{3}{5}a^2$
    }:
    \begin{equation}
      \braket{r^2} = \int r^2 \rho_N\mrb{r} \, \mathrm{d}^3r
    \end{equation}
    per valori di $q \ll 1$ si ha:
    \begin{equation}
      F\mrb{q^2} \simeq 1 - \braket{r^2} \frac{q^2}{6\hbar^2}
    \end{equation}
    e la sua derivata:
    \begin{equation}
      \frac{\mathrm{d}F\mrb{q^2}}{\mathrm{d}q^2} =
      -\frac{\braket{r^2}}{6\hbar^2}
    \end{equation}

    % \begin{figure}[ht]
    %   \centering
    %   \begin{tikzpicture}
    %     \begin{axis}[
    %       title=,
    %       xlabel=$\frac{q}{\tilde y}$,
    %       ylabel=$\ln \mrb{\abs{F \mrb{q^2}}^{2}}$,
    %       xmin=0.1,
    %       ymin=0.1,
    %     ]
    %
    %       \addplot[
    %         domain=1:2
    %       ] (ln(abs(0.33 * (sin(x) - x * cos(x)))**2));
    %     \end{axis}
    %   \end{tikzpicture}
    %   \caption{}
    %   \label{}
    % \end{figure}

    % Nel grafico, i valori $\frac{q}{\tilde y}$ per cui $\ln \mrb{\abs{F
    % \mrb{q^2}}^{2}}$ è $-\infty$ si chiamano \textbf{minimi di diffrazione}.

    \begin{note}[Legge empirica per il raggio di un nucleo]
      Il raggio di un nucleo può essere stabilito tramite la legge empirica:
      \[
        r \simeq \SI{12}{\femto\m} \cdot A^{\frac{1}{3}}
      \]
      dove $A = \# \text{ nucleoni del nucleo}$.
    \end{note}

	\item $\tilde{y} \gtrsim 1$:
    dobbiamo studiare il comportamento della funzione completa. Rielaboriamo
    l'espressione del fattore di forma (equazione
    \ref{eq:fattore_forma_sfera_uniforme}) come di seguito:
    \begin{equation}
      F\mrb{q^2} = \frac{3}{\tilde{y}^3}\msb{\sin\tilde{y} -
      \tilde{y}\cos\tilde{y}} = \frac{3}{\tilde{y}} \cos\tilde{y}
      \msb{\tan\tilde{y} - \tilde{y}}
    \end{equation}
\end{itemize}

\subsection{Modello Saxon - Woods}
La densità normalizzata di carica è descritta in questo modello, che risulta
più elaborato del precedente, da una funzione del tipo:
\begin{equation}
	\rho_N\mrb{r} = \frac{\rho_0}{1 + \exp\mrb{\frac{r-a}{d}}}
\end{equation}
e anche in questo caso la costante $\rho_0$ non è un parametro libero, poiché è
determinato dalla condizione di normalizzazione, mentre i \textbf{due
parametri} $a$ e $d$ sono rispettivamente:
\begin{itemize}
	\item $a$: \textit{raggio nucleare}
  \item $d$: \textit{thickness del nucleo}, ovvero lo spessore della superficie
    del nucleo
\end{itemize}
