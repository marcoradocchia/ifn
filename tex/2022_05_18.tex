%LTeX: language=it
\chapter{2022-05-19}
\section{Formula semi-empirica di massa (reprise)}
\subsection{Nuclei stabili}
Consideriamo che l'\textit{impulso di fermi} di protoni e neutori è
rispettivamente:
\begin{align*}
  p _{F,p}^{3} = Z \mcb{\frac{8}{3} \pi \frac{V}{\mrb{2 \pi}^{3}
  \hbar^3}}^{-1}
  \\
  p _{F,n}^{3} = \mrb{A - Z} \mcb{\frac{8}{3} \pi \frac{V}{\mrb{2 \pi}^{3}
  \hbar^3}}^{-1}
\end{align*}
Otteniamo dunque che l'\textit{energia totale}:
\[
  E _\text{tot} = \frac{8 \pi}{5} \frac{V}{2 m_N} \frac{1}{\mrb{2 \pi}^{3}
  \hbar^3} \mcb{\frac{8}{3} \pi \frac{V}{\mrb{2 \pi}^{3}
  \hbar^3}}^{-\nicefrac{5}{3}} \msb{Z ^{\nicefrac{5}{3}} + \mrb{A - Z}
  ^{\nicefrac{5}{3}}}
\]
dove abbiamo confuso le masse di protone e neutrone con la \textit{massa del
generico nucleone} $m_N$. Quindi:
\[
  E _\text{tot} = K \frac{\mrb{2 \pi}^{3} \hbar^2}{2 m_N} \frac{1}{V
  ^{\nicefrac{2}{3}}} \mcb{Z ^{\nicefrac{5}{3}} + \mrb{A -
  Z}^{\nicefrac{5}{3}}}
\]
quindi, raccogliendo ulteriori costanti:
\[
  E_\text{tot} = \frac{C}{A ^{\nicefrac{2}{3}}} \mcb{Z ^{\nicefrac{5}{3}} +
  \mrb{A - Z}^{\nicefrac{5}{3}}}
\]
Calcoliamo la derivata di $E _\text{tot}$ rispetto a $Z$, ottenendo il minimo
(\textit{estremale}) dell'energia cinetica in funzione di $A$:
\[
  \mpdv{E _\text{tot}}{Z} = \frac{5}{3} \frac{C}{A ^{\nicefrac{2}{3}}} \mcb{Z
  ^{\nicefrac{2}{3}} + \mrb{A - Z}^{\nicefrac{5}{3}}}
\]
quindi:
\[
  \mpdv{E _\text{tot}}{Z} = 0
  \mthen
  \boxed{
    Z = \frac{A}{2}
  }
\]
Sostituendo nell'espressione dell'energia totale, otteniamo l'energia
\textit{totale minima}:
\[
  \mrb{E _\text{tot}}_\text{min} = E _\text{tot} \mrb{Z = \frac{A}{2}, A} =
  \frac{C}{A ^{\nicefrac{2}{3}}} \mcb{\mrb{\frac{A}{2}}^{\nicefrac{5}{3}} +
  \mrb{\frac{A}{2}} ^{\nicefrac{5}{3}}} \propto A
\]
che \textit{contribuisce al termine di volume}.

Calcoliamo la derivata seconda:
\[
  \mpdv[2]{E _\text{tot}}{Z} = \frac{10}{9} \frac{C}{A ^{\nicefrac{2}{3}}}
  \mcb{Z ^{-\nicefrac{1}{3}} + \mrb{A - Z}^{-\nicefrac{1}{3}}}
\]
quindi:
\[
  \mpdv[2]{E _\text{tot}}{Z} \mrb{Z = \frac{A}{2}, A} =  \frac{10}{9}
  \frac{C}{A ^{\nicefrac{2}{3}}} \mcb{\mrb{\frac{A}{2}}^{-\nicefrac{1}{3}} +
  \mrb{\frac{A}{2}}^{-\nicefrac{1}{3}}} \propto \frac{1}{A}
\]
In conclusione, quindi:
\[
  E _\text{tot} \mrb{Z, A} \simeq KA + \frac{D}{A} \mrb{Z - \frac{A}{2}}^{2}
\]
dove $KA$ viene riassorbito dal termine di volume e $\frac{D}{A} \mrb{Z -
\frac{A}{2}}^{2}$ costituisce il \textit{termine di asimmetria}.

Quindi dobbiamo aggiungere un ultimo termine correttivo, chiamato
\textit{\textbf{termine di asimmetria}}, dovuto alla \textit{natura fermionica
di protoni e neutroni}. Otteniamo così:
\begin{align*}
  B = a_V A - a_S A ^{\nicefrac{2}{3}} - a_C \frac{Z^2}{A ^{\nicefrac{1}{3}}} -
  a_A \frac{\mrb{A - 2Z}^{2}}{A}
  \\
  \eps = a_V - a_S A ^{-\nicefrac{1}{3}} - a_C \frac{Z^2}{A ^{\nicefrac{4}{3}}}
  - a_A \frac{\mrb{A - 2Z}^{2}}{A^2}
\end{align*}

\begin{note}[]
  Il \textit{termine di asimmetria favorisce configurazioni} con $Z \simeq N$.
\end{note}

\subsubsection{Termine di Pairing}
Il \textit{termine di pairing} riflette l'osservazione sperimentale che $2$
protoni o $2$ neutroni possono formare un sistema particolarmente legato
(\textit{effettto di \quot{shell}} non legato alle caratteristiche specifiche
delle interazioni forti).
Per tenere conto di ciò si introduce:
\[
  B = a_V A - a_S A ^{\nicefrac{2}{3}} - a_C \frac{Z^2}{A ^{\nicefrac{1}{3}}} -
  a_A \frac{\mrb{A - 2Z}^{2}}{A} + \delta \mrb{A, Z}
\]
dove:
\begin{align*}
  \delta \mrb{A, Z} =
  \begin{dcases}
    - \frac{a_P}{A ^{\nicefrac{1}{2}}} & N = \textit{dispari};\ Z =
    \textit{dispari}
    \\
    0 & A = \textit{dispari}
    \\
    \frac{a_p}{A ^{\nicefrac{1}{2}}} & N = \textit{pari};\ Z = \textit{pari}
  \end{dcases}
\end{align*}

\subsubsection{Formula semi-empirica di massa}
Questa è dunque la \textit{formula semi-empirica di massa}:
\[
  \boxed{
    M \mrb{Z, A} c^2 = Z m_p c^2 + \mrb{A - Z} m_n c^2 - B \mrb{A, Z}
  }
\]
dove $M$ è la \textit{massa nucleare a riposo}, $m_p$ la \textit{massa del
protone}, $m_n$ la \textit{massa del neutrone} e $B \mrb{Z, A}$:
\[
  \boxed{
    \begin{alignedat}{2}
      B \mrb{Z, A} = & + a_V A && \qquad \textit{Termine di Volume}
      \\
      & - a_S A ^{\nicefrac{2}{3}} && \qquad \textit{Termine di Superficie}
      \\
      & - a_C \frac{Z^2}{A ^{\nicefrac{1}{3}}} && \qquad \textit{Termine di Coulomb}
      \\
      & - a_A \frac{\mrb{A - 2Z}^{2}}{A} && \qquad \textit{Termine di Asimmetria}
      \\
      & \begin{dcases}
        - \frac{a_P}{A ^{\nicefrac{1}{2}}} \quad N = \textit{pari}, Z =
        \textit{disp.}
        \\
        0 \quad A = \textit{disp.}
        \\
        \frac{a_P}{A ^{\nicefrac{1}{2}}} \quad N = \textit{pari}, Z =
        \textit{disp.}
      \end{dcases} && \qquad \textit{Termine di Pairing}
    \end{alignedat}
  }
\]

La formula si può scrivere anche in termini della \textit{masssa atomica a
riposo} (piuttosto che la massa del nucleo), quindi indicando con $\mu$ la
\textit{massa atomica}:
\[
  \mu \mrb{A, Z} c^2 = Z m_p c^2 + \mrb{A - Z} m_n c^2 + Z m_e c^2 - B
  \mrb{A, Z} - \cancelto{0}{B _\text{atomic}}
\]
perché nell'atomo dobbiamo tener conto della massa degli elettroni e sottrarre
la \textit{binding energy atomica}, che tiene conto delle energie di legame
degli elettroni, che però, essendo dell'ordine dell'$\si{\eV}$, può essere
trascurata.
Quindi possiamo riscrivere la massa atomica, considerando che $m_p c^2 + m_e
c^2 = m_H c^2$ è la \textit{massa a riposo dell'atomo di idrogeno}:
\[
  \boxed{
    \mu \mrb{A, Z} c^2 \simeq Z m_H c^2 + \mrb{A - Z} m_n c^2 - B \mrb{A, Z}
  }
\]
