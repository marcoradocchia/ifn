%LTeX: language=it
\chapter{2022-05-19}
\section{Formula semi-empirica di massa (reprise)}
\subsection{Nuclei stabili}
Consideriamo che l'\textit{impulso di fermi} di protoni e neutroni è
rispettivamente:
\begin{align}
  p _{F,p}^{3} &= Z \mcb{\frac{8}{3} \pi \frac{V}{\mrb{2 \pi}^{3}
  \hbar^3}}^{-1}
  \\
  p _{F,n}^{3} &= \mrb{A - Z} \mcb{\frac{8}{3} \pi \frac{V}{\mrb{2 \pi}^{3}
  \hbar^3}}^{-1}
\end{align}
\begin{note}[]
  Per nuclei tali che $Z = N = \frac{A}{2}$ abbiamo che l'\textit{impulso di
  fermi}:
  \begin{equation}
    p_F
    \equiv  p _{F, n}
    \equiv  p _{F, p}
    = \mrb{\frac{A}{2} \frac{3 \pi^2 \hbar^3}{V}}^{\frac{1}{3}}
    = \mrb{\frac{\cancel{A}}{2} \frac{3 \pi^{\cancel{2}} \hbar^3}{\frac{4}{3}
      \cancel{\pi} R_0^3 \cancel{A}}}
    = \mrb{\frac{9 \pi}{8}}^{\frac{1}{3}} \frac{\hbar}{R_0}
    \simeq \qty{250}{\MeV \per c}
  \end{equation}
\end{note}

Otteniamo dunque che l'\textit{energia totale}:
\begin{equation}
  E _\text{tot} = \frac{8 \pi}{5} \frac{V}{2 m_N} \frac{1}{\mrb{2 \pi}^{3}
  \hbar^3} \mcb{\frac{8}{3} \pi \frac{V}{\mrb{2 \pi}^{3}
  \hbar^3}}^{-\nicefrac{5}{3}} \msb{Z ^{\nicefrac{5}{3}} + \mrb{A - Z}
  ^{\nicefrac{5}{3}}}
\end{equation}
dove abbiamo confuso le masse di protone e neutrone con la \textit{massa del
generico nucleone} $m_N$. Quindi:
\begin{equation}
  E _\text{tot} = K \frac{\mrb{2 \pi}^{3} \hbar^2}{2 m_N} \frac{1}{V
  ^{\nicefrac{2}{3}}} \mcb{Z ^{\nicefrac{5}{3}} + \mrb{A -
  Z}^{\nicefrac{5}{3}}}
\end{equation}
quindi, raccogliendo ulteriori costanti:
\begin{equation}
  E_\text{tot} = \frac{C}{A ^{\nicefrac{2}{3}}} \mcb{Z ^{\nicefrac{5}{3}} +
  \mrb{A - Z}^{\nicefrac{5}{3}}}
\end{equation}
Calcoliamo la derivata di $E _\text{tot}$ rispetto a $Z$, ottenendo il minimo
(\textit{estremale}) dell'energia cinetica in funzione di $A$:
\begin{equation}
  \mpdv{E _\text{tot}}{Z} = \frac{5}{3} \frac{C}{A ^{\nicefrac{2}{3}}} \mcb{Z
  ^{\nicefrac{2}{3}} + \mrb{A - Z}^{\nicefrac{5}{3}}}
\end{equation}
quindi:
\begin{equation}
  \mpdv{E _\text{tot}}{Z} = 0
  \mthen
  \boxed{
    Z = \frac{A}{2}
  }
\end{equation}
Sostituendo nell'espressione dell'energia totale, otteniamo l'energia
\textit{totale minima}:
\begin{equation}
  \mrb{E _\text{tot}}_\text{min} = E _\text{tot} \mrb{Z = \frac{A}{2}, A} =
  \frac{C}{A ^{\nicefrac{2}{3}}} \mcb{\mrb{\frac{A}{2}}^{\nicefrac{5}{3}} +
  \mrb{\frac{A}{2}} ^{\nicefrac{5}{3}}} \propto A
\end{equation}
che \textit{contribuisce al termine di volume}.

Calcoliamo la derivata seconda:
\begin{equation}
  \mpdv[2]{E _\text{tot}}{Z} = \frac{10}{9} \frac{C}{A ^{\nicefrac{2}{3}}}
  \mcb{Z ^{-\nicefrac{1}{3}} + \mrb{A - Z}^{-\nicefrac{1}{3}}}
\end{equation}
quindi:
\begin{equation}
  \mpdv[2]{E _\text{tot}}{Z} \mrb{Z = \frac{A}{2}, A} =  \frac{10}{9}
  \frac{C}{A ^{\nicefrac{2}{3}}} \mcb{\mrb{\frac{A}{2}}^{-\nicefrac{1}{3}} +
  \mrb{\frac{A}{2}}^{-\nicefrac{1}{3}}} \propto \frac{1}{A}
\end{equation}
In conclusione, quindi:
\begin{equation}
  E _\text{tot} \mrb{Z, A} \simeq KA + \frac{D}{A} \mrb{Z - \frac{A}{2}}^{2}
\end{equation}
dove $KA$ viene riassorbito dal termine di volume e $\frac{D}{A} \mrb{Z -
\frac{A}{2}}^{2}$ costituisce il \textit{termine di asimmetria}.

Quindi dobbiamo aggiungere un ultimo termine correttivo, chiamato
\textit{\textbf{termine di asimmetria}}, dovuto alla \textit{natura fermionica
di protoni e neutroni}. Otteniamo così:
\begin{equation}
  B = a_V A - a_S A ^{\nicefrac{2}{3}} - a_C \frac{Z^2}{A ^{\nicefrac{1}{3}}} -
  a_A \frac{\mrb{A - 2Z}^{2}}{A}
\end{equation}
Quindi
\begin{equation}
  \eps = a_V - a_S A ^{-\nicefrac{1}{3}} - a_C \frac{Z^2}{A ^{\nicefrac{4}{3}}}
  - a_A \frac{\mrb{A - 2Z}^{2}}{A^2}
\end{equation}

\begin{note}[]
  Il \textit{termine di asimmetria favorisce configurazioni} con $Z \simeq N$.
\end{note}

\subsubsection{Termine di Pairing}
Il \textit{termine di pairing} riflette l'osservazione sperimentale che $2$
protoni o $2$ neutroni possono formare un sistema particolarmente legato
(\textit{effettto di \quot{shell}} non legato alle caratteristiche specifiche
delle interazioni forti).
Per tenere conto di ciò si introduce:
\begin{equation}
  B = a_V A - a_S A ^{\nicefrac{2}{3}} - a_C \frac{Z^2}{A ^{\nicefrac{1}{3}}} -
  a_A \frac{\mrb{A - 2Z}^{2}}{A} + \delta_p \mrb{A, Z}
\end{equation}
dove:
\begin{align*}
  \delta_p \mrb{A, Z} =
  \begin{dcases}
    - \frac{a_P}{A ^{\nicefrac{1}{2}}},
    & N = \textit{dispari},\ Z = \textit{dispari}
    \\
    0, & A = \textit{dispari}
    \\
    \frac{a_p}{A ^{\nicefrac{1}{2}}}, & N = \textit{pari},\ Z = \textit{pari}
  \end{dcases}
\end{align*}

\begin{note}[]
  I nuclei $N = \textit{dispari}$ e $Z = \textit{displari}$ sono molto rari in
  natura, infatti quelli noti sono solo quattro:
  $_{1}^{2}\ch{H},\ ^{6}_{3}Li,\ _{5}^{10}B,\ _{7}^{14}N$.
\end{note}

\begin{note}[]
  Questo termine si può studiare sperimentalmente misurando la
  \quot{\textit{neutron separation energy}}, ovvero l'energia necessaria per
  liberare un neutrone dal nucelo con energia cinetica nulla.
\end{note}

% TODO: aggiungi figura B / A

\subsubsection{Formula semi-empirica di massa}
Questa è dunque la \textit{formula semi-empirica di massa}:
\begin{equation}
  \boxed{
    M \mrb{Z, A} c^2 = Z m_p c^2 + \mrb{A - Z} m_n c^2 - B \mrb{A, Z}
  }
\end{equation}
dove $M$ è la \textit{massa nucleare a riposo}, $m_p$ la \textit{massa del
protone}, $m_n$ la \textit{massa del neutrone} e $B \mrb{Z, A}$:
\begin{equation}
  \boxed{
    \begin{alignedat}{2}
      B \mrb{Z, A} = & + a_V A && \qquad \textit{Termine di Volume}
      \\
      & - a_S A ^{\nicefrac{2}{3}} && \qquad \textit{Termine di Superficie}
      \\
      & - a_C \frac{Z^2}{A ^{\nicefrac{1}{3}}} && \qquad \textit{Termine di Coulomb}
      \\
      & - a_A \frac{\mrb{A - 2Z}^{2}}{A} && \qquad \textit{Termine di Asimmetria}
      \\
      & \begin{dcases}
        - \frac{a_P}{A ^{\nicefrac{1}{2}}} \quad N = \textit{pari}, Z =
        \textit{disp.}
        \\
        0 \quad A = \textit{disp.}
        \\
        \frac{a_P}{A ^{\nicefrac{1}{2}}} \quad N = \textit{pari}, Z =
        \textit{disp.}
      \end{dcases} && \qquad \textit{Termine di Pairing}
    \end{alignedat}
  }
\end{equation}

La formula si può scrivere anche in termini della \textit{masssa atomica a
riposo} (piuttosto che la massa del nucleo), quindi indicando con $\mu$ la
\textit{massa atomica}:
\begin{equation}
  \mu \mrb{A, Z} c^2 = Z m_p c^2 + \mrb{A - Z} m_n c^2 + Z m_e c^2 - B
  \mrb{A, Z} - \cancelto{0}{B _\text{atomic}}
\end{equation}
perché nell'atomo dobbiamo tener conto della massa degli elettroni e sottrarre
la \textit{binding energy atomica}, che tiene conto delle energie di legame
degli elettroni, che però, essendo dell'ordine dell'$\si{\eV}$, può essere
trascurata.
Quindi possiamo riscrivere la massa atomica, considerando che $m_p c^2 + m_e
c^2 = m_H c^2$ è la \textit{massa a riposo dell'atomo d'idrogeno}:
\begin{equation}
  \boxed{
    \mu \mrb{A, Z} c^2 \simeq Z m_H c^2 + \mrb{A - Z} m_n c^2 - B \mrb{A, Z}
  }
\end{equation}

\begin{note}[]
  La \textit{formula semi-empirica di massa} (modello a goccia del nucleo)
  funziona piuttosto bene per $A > 20$, ma non riesce a riprodurre i picchi
  nell'energia di legame per nucleone che si osservano sperimentalmente in
  prossimità dei cosiddetti \textit{numeri magici}.
  Una trattazione più accurata sarà il \textit{modello a shell}.
\end{note}

\begin{note}[]
  I valori dei parametri che riproducono al meglio gli andamenti sperimentali
  dell'energia di legame per nucleone sono:
  \begin{equation}
    a_V \simeq \qty{15.6}{\MeV},
    \qquad
    a_S \simeq \qty{17.2}{\MeV},
    \qquad
    a_C \simeq \qty{0.7}{\MeV},
    \qquad
    a_A \simeq \qty{23.3}{\MeV}
  \end{equation}
\end{note}
